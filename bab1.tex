%-----------------------------------------------------------------------------%
\chapter{\babSatu}
%-----------------------------------------------------------------------------%

%-----------------------------------------------------------------------------%
\section{Latar Belakang}
%-----------------------------------------------------------------------------%
Bahasa merupakan representasi perilaku kompleks yang dimiliki oleh spesies dengan tingkat kognisi tinggi. Elemen suara, kata-kata, dan pola sintaktis sebuah bahasa berbeda untuk setiap kelompok manusia dan berkembang secara perlahan menjadi bentuk bahasa yang paling efisien \cite{aitchison_2004} . Hal ini disebabkan oleh berkembangnya pola linguistik seiring waktu dan penyebaran penuturnya (Sapir, 1921). Chomsky (1965) juga mengungkapkan bahwa keserupaan dari semua bahasa di dunia adalah aspek kreatifnya, sehingga memungkinkan penutur untuk mengekspresikan pikiran dengan jumlah yang tak terhingga dan bereaksi secara tepat dalam segala bentuk situasi. Salah satu karakteristik ujaran manusia yang paling utama adalah tingkat keteraturan atau organisasi isi tuturan yang sangat tinggi. Semua satuan yang membentuk sebuah tuturan disusun oleh penutur dengan konfigurasi yang sangat spesifik mengikuti aturan tertentu. Aturan-aturan ini membentuk tataran organisasi terpenting dari sebuah bahasa, yaitu sintaksis. Para linguis Jerman terdahulu menerjemahkan kata sintaksis ke dalam bahasa Jerman sebagai Satzlehre atau ?ilmu kalimat? karena obyek dari sintaksis struktural adalah studi tentang kalimat (Tesni�re, 1959).



%-----------------------------------------------------------------------------%
\section{Permasalahan}
%-----------------------------------------------------------------------------%
Pada bagian ini akan dijelaskan mengenai definisi permasalahan 
yang \saya~hadapi dan ingin diselesaikan serta asumsi dan batasan 
yang digunakan dalam menyelesaikannya.


%-----------------------------------------------------------------------------%
\subsection{Definisi Permasalahan}
%-----------------------------------------------------------------------------%
\todo{Tuliskan permasalahan yang ingin diselesaikan. Bisa juga
	berbentuk pertanyaan}


%-----------------------------------------------------------------------------%
\subsection{Batasan Permasalahan}
%-----------------------------------------------------------------------------%
\todo{Umumnya ada asumsi atau batasan yang digunakan untuk 
	menjawab pertanyaan-pertanyaan penelitian diatas.}


%-----------------------------------------------------------------------------%
\section{Tujuan}
%-----------------------------------------------------------------------------%
\todo{Tuliskan tujuan penelitian.}


%-----------------------------------------------------------------------------%
\section{Posisi Penelitian}
%-----------------------------------------------------------------------------%
\todo{Posisi penelitian Anda jika dilihat secara bersamaan dengan 
	peneliti-peneliti lainnya. Akan lebih baik lagi jika ikut menyertakan 
	diagram yang menjelaskan hubungan dan keterkaitan antar 
	penelitian-penelitian sebelumnya}


%-----------------------------------------------------------------------------%
\section{Metodologi Penelitian}
%-----------------------------------------------------------------------------%
\todo{Tuliskan metodologi penelitian yang digunakan.}


%-----------------------------------------------------------------------------%
\section{Sistematika Penulisan}
%-----------------------------------------------------------------------------%
Sistematika penulisan laporan adalah sebagai berikut:
\begin{itemize}
	\item Bab 1 \babSatu \\
	\item Bab 2 \babDua \\
	\item Bab 3 \babTiga \\
	\item Bab 4 \babEmpat \\
	\item Bab 5 \babLima \\
	\item Bab 6 \babEnam \\
	\item Bab 7 \kesimpulan \\
\end{itemize}

\todo{Tambahkan penjelasan singkat mengenai isi masing-masing bab.}

