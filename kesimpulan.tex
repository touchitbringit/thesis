%---------------------------------------------------------------
\chapter{\kesimpulan}
%---------------------------------------------------------------


%---------------------------------------------------------------
\section{Kesimpulan}
%---------------------------------------------------------------

Tinjauan terhadap pengurangan panjang dependensi (DLM) dan pengurangan jarak dependensi (DDM) pada tataran kalimat dalam bahasa Indonesia ragam tulis dan lisan memperlihatkan beberapa temuan yang sesuai dengan hipotesis yang diajukan. Temuan-temuan ini mendukung ide bahwa bahasa manusia memiliki kecenderungan untuk mengatur urutan kata sehingga memiliki panjang dependensi kalimat dan jarak dependensi antarkonstituen yang lebih pendek. Pendekatan ini dapat dilakukan dengan menerapkan beberapa strategi seperti yang telah dibahas pada penelitian-penelitian terdahulu (\citealp{jaeger2006redundancy, gildea2015human}). Dalam ranah sintaksis, strategi pendekatan konstituen-konstituen ini termasuk melalui organisasi urutan kata dan pengurangan sintaktis dalam kalimat (yang mungkin diakibatkan oleh pengurangan unsur yang repetitif ataupun hal yang lain). 

Penelitian ini memanfaatkan dua metode untuk mengukur panjang dan jarak dependensi yaitu konsep panjang dependensi atau \textit{Dependency Length} (DL) dan rata-rata jarak dependensi atau \textit{Mean Dependency Distance} (MDD) untuk meninjau dua korpus yang berisikan teks informatif bahasa Indonesia dalam ragam tulis dan lisan. Kedua pendekatan ini memperlihatkan adaya perbedaan dalam temuan yang didapatkan terutama terkait pengaruh panjang kalimat terhadap panjang dan jarak dependensi. Salah satu bagian analisis penelitian ini adalah tinjauan terhadap hipotesis DLM dengan melakukan percobaan acak menggunakan \textit{Free Word Order Baseline} \citep{futrell2015large} yang menghasilkan 100 kemungkinan struktur yang tidak memiliki aturan urutan kata tertentu untuk setiap kalimat. Hasil dari percobaan ini mendukung hipotesis bahwa terjadi DLM pada kedua korpus data (ragam tulis maupun ragam lisan) secara signifikan (\textit{P} \textless 0,001). 

Analisis kemungkinan pengaruh panjang kalimat terhadap panjang dan jarak dependensi serta konstruksi struktur kalimat menghasilkan temuan yang menunjukkan indikasi adanya kondisi tertentu di mana panjang kalimat berpengaruh secara signifikan dan kondisi lain di mana pengaruh tersebut tidak terlalu jelas. Kedua pendekatan DL dan MDD menunjukkan signifikansi pada kategori kalimat pendek, terutama pada jumlah konstituen 5 atau kurang, dan pada kalimat panjang, terutama pada jumlah konstituen antara 21 hingga 30. Namun, perbedaan yang didapat dengan menerapkan pendekatan DL tidak signifikan selain kedua kategori tersebut. Bahasa Indonesia ragam lisan memperlihatkan kecenderungan yang kuat untuk mengurangi panjang kalimat dalam menghasilkan panjang dan jarak dependensi yang pendek. Dari hasil tinjauan yang lebih dalam dan hasil analisis kualitatif terhadap struktur sintaktis beberapa kalimat contoh, pengurangan panjang dan jarak dependensi pada ragam lisan ini diakibatkan terutama oleh reduksi elemen sintaktis yang termasuk didalamnya adalah reduksi atau pengurangan valensi kata. Temuan ini juga sempat disinggung oleh \cite{jaeger2006redundancy} yang meneliti reduksi sintaktis  pada ujaran spontan dalam bahasa Inggris. Dalam bahasa Indonesia, hal ini mungkin berkaitan dengan karakteristik bahasa Indonesia yang cukup bebas terkait penerapan tata bahasa dalam tuturan nyata \citep{sneddon2010indonesian} dan urutan kata yang bebas (\citealp{sneddon2010indonesian, kridalaksana2002struktur}). 

Meskipun bahasa Indonesia ragam tulis menunjukkan distribusi panjang kalimat yang lebih merata, ragam ini menghasilkan panjang dan jarak dependensi yang lebih pendek pada jumlah konstituen yang lebih banyak (terindikasi mulai dari 6 konstituen, tapi menunjukkan signifikansi tinggi pada jumlah konstituen lebih dari 20). Tinjauan lebih dekat dan hasil analisis kualitatif terhadap struktur sintaktis beberapa kalimat contoh mendukung temuan tersebut dan bahkan memperlihatkan manifestasi karakteristik sintaktis yang mungkin menjadi keunikan atau kekhasan masing-masing ragam. Hal ini berarti, dilihat dari konstruksi kalimatnya, kalimat pada ragam tulis dan lisan memiliki ciri-ciri tersendiri. Berdasarkan analisis yang dilakukan, konstruksi kalimat lisan lebih tidak konsisten dalam aspek posisi akar dalam kalimat dan sistematika percabangannya sebagaimana tervisualisasikan melalui bank pohon struktur. Selain itu, karena ragam lisan memiliki karakter percabangan yang menyebar, maka tingkat kedalaman dependensi ragam ini jrata-rata auh lebih dangkal. Oleh sebab itu, reduksi konstituen merupakan strategi yang lebih diutamakan pada ragam lisan. Keseluruhan tinjauan terkait panjang kalimat ini menunjukkan adanya indikasi bahwa tata bahasa dalam bahasa Indonesia yang diterapkan dalam teks dengan karakter yang lebih formal seperti ragam tulis mendukung hipotesis DLM dan DDM dengan kuat. 

Dalam teori struktur frasa, bahasa Indonesia memiliki banyak hubungan di mana kata kepala berada di posisi sebelum kata yang memodifikasinya. Namun, hubungan ini umumnya hanya menjelaskan frasa yang terdiri dari kata-kata yang saling berdampingan. Dependensi melihat hubungan induk dengan konstituen terikat tidak hanya pada tingkat tersebut, tapi juga hingga pada tingkat argumen utama. Analisis direksionalitas induk yang dilakukan penelitian ini juga menunjukkan kesesuaian hierarki hubungan dalam struktur frasa tersebut baik pada level kata-kata yang berdampingan hingga pada level argumen utama (yang terlihat pada simpai pusat). Temuan ini menjadi bukti empiris adanya kecenderungan terhadap bentuk relasi diawali induk pada bahasa Indonesia, terutama pada ragam tulis dan pada kalimat panjang dalam kedua korpus. Namun, berdasarkan pemetaan dan visualisasi, kecenderungan bahasa Indonesia memanfaatkan strategi percabangan searah atau memanfaatkan strategi menyeimbangkan dependensi positif dan negatif belum terpapar dengan jelas. Penelitian lebih lanjut yang melibatkan pencarian tingkat kesalahan (\textit{margin of error}) dengan fungsi matematis perlu dilakukan untuk menjawab pertanyaan ini. Meskipun begitu, analisis kualitatif terhadap kalimat-kalimat panjang dalam ragam tulis memperlihatkan banyaknya relasi dependensi yang berdampingan sehingga menunjukkan adanya indikasi bahwa kalimat-kalimat panjang dalam bahasa Indonesia ragam tulis menerapkan strategi percabangan searah.

Dari sisi tata bahasa, temuan pada ragam lisan juga memperlihatkan bagaimana ragam ini menyimpang dari tata bahasa yang ada dan struktur sintaktis yang tidak konsisten. Hal ini mungkin disebabkan oleh keterbatasan kognisi manusia untuk merealisasikan kalimat secara verbal, terutama untuk kalimat panjang. Temuan ini juga memiliki keterkaitan terhadap reduksi konstituen dalam upaya untuk menghasilkan panjang dan jarak dependensi yang pendek. Data ragam lisan tidak menunjukkan kecenderungan secara signifikan terhadap bentuk relasi diawali induk dari segi frekuensi, namun memperlihatkan kecenderungan signifikan untuk menekan jauh jarak dependensi yang negatif. Hal ini menimbulkan pertanyaan bahwa apakah strategi terkait direksionalitas induk tidak selalu menjadi preferensi penutur untuk menghasilkan kalimat yang efisien atau ini merupakan salah satu karakteristik ragam lisan yang tidak konsisten.

Pada ragam lisan, banyak kalimat ditemukan tidak memiliki aktor pelaku atau subyek pada argumen utama berdasarkan pada analisis valensi akar verbal pada simpai pusat. Penelitian ini mengambil prinsip dasar dalam konsep kemungkinan pola valensi atau \textit{Probabilistic Valency Pattern} (PVP) \citep{liu2006syntactic} untuk analisis kualitatif. Kalimat seperti ini banyak ditemukan terutama pada panjang kalimat pendek dan menengah. Meskipun jarang ditemukan pada kalimat panjang, terdapat temuan yang cukup menarik, yaitu apabila argumen utama (pada klausa utama) tidak memiliki aktor pelaku, pronomina yang merujuk pada aktor pelaku tersebut juga tidak ditemukan pada klausa terikat. Namun, apabila pronomina ditemukan pada klausa terikat, maka aktor pelaku juga pasti muncul pada klausa utama. 

Secara umum, berdasarkan data yang dikumpulkan, bahasa Indonesia ragam tulis dan lisan memiliki perbedaan karakteristik dalam mengonstruksikan struktur sintaktis untuk menghasilkan kalimat yang efisien dari segi dependensi. Bahasa Indonesia ragam lisan lebih menunjukkan kecenderungan untuk mengurangi konsituen dalam kalimat dan menghasilkan kalimat yang pendek. Sebaliknya, kalimat yang lebih panjang dalam ragam tulis menunjukkan konstruksi sintaktis yang lebih terorganisir dibandingkan ragam lisan. Hal ini mungkin disebabkan oleh keterbatasan memori kerja manusia dalam memproduksi kalimat panjang secara verbal. Untuk mendapatkan pemahaman mengenai konstruksi yang lebih efisien secara akurat dan yang lebih mendekati preferensi pendengar, beberapa penelitian lanjutan perlu dilakukan. Meskipun begitu, penelitian ini telah memberikan dasar awal yang fundamental untuk dapat dikembangkan menjadi penelitian-penelitian linguistik transdisipliner lain yang membahas tentang hubungan struktur sintaktis sebuah kalimat dengan keterbatasan memori kerja dalam kognisi manusia.

%---------------------------------------------------------------
\section{Perkembangan penelitian}
%---------------------------------------------------------------

Temuan yang berlawanan dalam data bahasa Indonesia ragam tulis dan lisan pada kalimat pendek dan panjang memunculkan beberapa pertanyaan untuk penelitian-penelitian lanjutan. Perbedaan struktur sintaktis akibat konstruksi kalimat kedua ragam memicu tinjauan lanjutan untuk mengukur struktur sintaktis secara kuantitatif dan akurat dalam usaha untuk lebih memahami tuntutan sintaktis dalam kognisi manusia. Terkait dengan valensi, penelitian ini belum mengukur secara akurat dan mengkuantifikasi analisis valensi karena kamus atau leksikon valensi dalam bahasa Indonesia belum ada. Sementara, konsep PVP \citep{liu2006syntactic} yang dimanfaatkan dalam penelitian ini serta analisis valensi dalam beberapa bahasa lain telah sampai pada analisis kuantitatif berskala besar dan pembentukan leksikon valensi terutama valensi verbal (\citealp{zabokrtsky2005valency, bielicky2008building, passarotti2016latin, semecky2006constructing}). Oleh karena itu, leksikon valensi terutama untuk kelas kata verba dalam bahasa Indonesia penting untuk dikembangkan dalam upaya untuk mendapatkan analisis valensi yang lebih mendalam terkait kemampuan sebuah konstituen untuk mengikat konstituen lain. 

Penelitian lanjutan yang melibatkan uji persepsi untuk meninjau ada tidaknya preferensi pendengar atau pembaca terhadap struktur sintaktis tertentu juga akan memberikan masukan wawasan linguistik yang sangat bermanfaat dalam mempelajari penampilan bahasa dengan memanfaatkan data bahasa Indonesia. Kajian lain juga dibutuhkan untuk dapat membuktikan apakah perbedaan ini bersifat kontekstual terhadap bahasa Indonesia atau bersifat lintas bahasa. Aspek penting lainnya yang perlu ditinjau adalah kemungkinan adanya perbedaan tingkat pemahaman bahasa antara pendengar (yang mendengar kalimat secara verbal) dan pembaca (yang membaca kalimat secara tertulis) yang dapat mempengaruhi pandangan terhadap struktur sintaktis yang lebih mudah dipahami. 