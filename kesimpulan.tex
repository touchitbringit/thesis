%---------------------------------------------------------------
\chapter{\kesimpulan}
%---------------------------------------------------------------


%---------------------------------------------------------------
\section{Kesimpulan}
%---------------------------------------------------------------

Tinjauan terhadap pengurangan panjang dependensi  jarak dependensi pada tataran kalimat dalam bahasa Indonesia ragam tulis dan lisan memperlihatkan beberapa temuan yang menjawab permasalahan yang diajukan. Uji dan tinjauan yang dilakukan memperlihatkan kondisi-kondisi tertentu yang mendukung hipotesis dalam perumusan masalah. Secara garis besar, temuan-temuan ini mendukung ide umum bahwa bahasa manusia memiliki kecenderungan untuk mendekatkan konstituen-konstituen yang memiliki relasi semantik. Penutur melakukan ini dengan mengatur urutan kata sehingga memiliki panjang dependensi kalimat dan jarak dependensi antarkonstituen yang lebih pendek. Pendekatan ini dapat dilakukan dengan menerapkan beberapa strategi seperti yang telah dibahas pada penelitian-penelitian terdahulu (\citealp{jaeger2006redundancy, gildea2015human}). Dalam ranah sintaksis, strategi pendekatan konstituen-konstituen ini termasuk melalui organisasi urutan kata dan pengurangan sintaktis dalam kalimat (yang mungkin diakibatkan oleh pengurangan unsur yang repetitif ataupun hal yang lain). 

Hipotesis pertama menitikberatkan pada pengaruh panjang kalimat terhadap panjang dan jarak dependensi serta konstruksi struktur kalimat. Temuan dari uji hipotesis ini menunjukkan indikasi adanya kondisi tertentu di mana panjang kalimat berpengaruh secara signifikan dan kondisi lain di mana pengaruh tersebut tidak terlalu jelas. Bahasa Indonesia ragam lisan memperlihatkan kecenderungan yang kuat untuk mengurangi jumlah konstituen dalam upaya mendekatkan konstituen-konstituen lain yang memiliki tautan langsung. Hal ini menyebabkan mayoritas kalimat-kalimat pada korpus ragam lisan tergolong kalimat pendek. Sementara itu, pada kalimat yang lebih panjang, bahasa Indonesia ragam tulis justru menunjukkan panjang dan jarak dependensi yang lebih pendek. Hal ini berarti ragam tulis memiliki struktur sintaktis yang lebih efisien dibandingkan ragam lisan pada kalimat yang lebih panjang. Efisiensi ini terindikasi mulai dari 6 konstituen, tapi menunjukkan signifikansi tinggi pada jumlah konstituen lebih dari 20. 

Hasil analisis kualitatif terhadap struktur sintaktis beberapa kalimat contoh mendukung temuan tersebut dan bahkan memperlihatkan manifestasi karakteristik sintaktis yang mungkin menjadi keunikan atau kekhasan masing-masing ragam. Dilihat dari konstruksi kalimatnya, karakter ini menunjukkan bahwa kalimat pada ragam tulis dan lisan memiliki ciri-ciri tersendiri. Berdasarkan analisis yang dilakukan, konstruksi kalimat lisan lebih tidak konsisten dari segi penempatan posisi akar dalam kalimat dan sistematika percabangannya sebagaimana tervisualisasikan melalui bank pohon struktur. Bentuk percabangan yang menyebar menunjukkan bahwa banyak konstituen-konstituen terikat dalam kalimat bergantung pada induk yang sama sehingga tingkat kedalaman dependensi menjadi dangkal. Sebaliknya, akar pada kalimat-kalimat ragam tulis umumnya tidak memiliki banyak konstituen terikat, dan konstituen terikat tersebut memiliki konstituen terikat lain. Hubungan yang bersifat berlanjut ini menyebabkan tingkat kedalaman dependensi menjadi dalam dan jarak-jarak dependensi yang lebih pendek. Keseluruhan tinjauan terkait panjang kalimat menunjukkan adanya indikasi bahwa tata bahasa dalam bahasa Indonesia yang diterapkan dalam teks dengan karakter yang lebih formal seperti ragam tulis dapat lebih mengurangi panjang dan jarak efisiensi dan menghasilkan struktur sintaktis yang efisien. 

Uji hipotesis kedua dilakukan untuk melihat kecenderungan dalam bahasa Indonesia bahwa induk menempati posisi sebelum konstituen terikatnya. Dalam teori struktur frasa (\citealp{sneddon2010indonesian, kridalaksana2002struktur}), diungkapkan juga bahwa induk cenderung mendahului pewatasnya. Temuan uji hipotesis ini menunjukkan keserupaan hierarki hubungan seperti dalam struktur frasa pada hubungan dua konstituen yang memiliki tautan langsung secara umum. Namun, pada argumen utama dengan akar verbal (yang terlihat pada simpai pusat), tidak terlihat ada kecenderungan induk menempati posisi sebelum konstituen terikatnya. Bentuk relasi diakhiri induk lebih banyak ditemukan pada kalimat dengan jumlah konstituen kurang dari 20. Sebaliknya, pada kalimat yang panjang bentuk relasi diawali induk lebih banyak ditemukan. Hal ini dapat dikaitkan dengan salah satu karakter bahasa Indonesia, yaitu urutan kata bebas (\citealp{stack2005word, postman2004processing}) yang menyebabkan dalam kondisi tertentu urutan tersebut dapat berubah \citep[pp. 209-268]{sneddon2010indonesian}. 

Posisi induk terhadap konstituen terikat juga berkaitan dengan pola percabangan. Berdasarkan pemetaan dan visualisasi, kecenderungan bahasa Indonesia memanfaatkan strategi percabangan searah atau memanfaatkan strategi menyeimbangkan dependensi positif dan negatif belum terpapar dengan jelas. Penelitian lebih lanjut yang melibatkan pencarian tingkat kesalahan (\textit{margin of error}) dengan fungsi matematis perlu dilakukan untuk menjawab pertanyaan ini. Meskipun begitu, analisis kualitatif terhadap kalimat-kalimat panjang dalam ragam tulis memperlihatkan banyaknya relasi dependensi yang berdampingan sehingga menunjukkan adanya indikasi bahwa kalimat-kalimat panjang dalam bahasa Indonesia ragam tulis menerapkan strategi percabangan searah.

Perbedaan penempatan induk pada posisi tertentu terhadap konstituen terikatnya berdampak pada jarak yang terbentuk di antara dua konstituen tersebut. Bentuk relasi diakhiri induk atau induk menempati posisi setelah konstituen terikat cenderung berjarak lebih pendek dibandingkan bentuk relasi sebaliknya. Kecenderungan ini ditemukan mulai jumlah konstituen 6 pada kedua ragam. Hal ini berarti penutur bahasa Indonesia cenderung menghindari bentuk relasi diakhiri induk yang memiliki jarak jauh.

Tinjauan berikutnya untuk menjawab permasalahan terakhir adalah pengaruh penyusunan struktur kalimat yang efisien terhadap valensi akar verbal. Reduksi atau pengurangan valensi kata adalah satu strategi penutur dalam mengurangi jumlah konstituen seperti yang ditemukan dalam ragam lisan. Temuan ini juga sempat disinggung oleh \cite{jaeger2006redundancy} yang meneliti reduksi sintaktis pada ujaran spontan dalam bahasa Inggris. Pada ragam lisan, banyak kalimat ditemukan tidak memiliki aktor pelaku atau subyek pada argumen utama berdasarkan pada analisis valensi akar verbal pada simpai pusat. Kalimat seperti ini banyak ditemukan terutama pada panjang kalimat pendek dan menengah. Meskipun jarang ditemukan pada kalimat panjang, terdapat temuan yang cukup menarik, yaitu apabila argumen utama (pada klausa utama) tidak memiliki aktor pelaku, pronomina yang merujuk pada aktor pelaku tersebut juga tidak ditemukan pada klausa terikat. Namun, apabila pronomina ditemukan pada klausa terikat, maka aktor pelaku juga pasti muncul pada klausa utama. 

Seluruh hasil uji hipotesis dan tinjauan kualitatif memperlihatkan bahwa perbedaan utama terkait pengurangan panjang dan jarak dependensi antara bahasa Indonesia ragam tulis dan lisan mungkin disebabkan oleh keterbatasan memori kerja manusia dalam memproduksi kalimat panjang secara verbal. Untuk mendapatkan pemahaman mengenai konstruksi yang lebih efisien secara akurat dan yang lebih mendekati preferensi pendengar, beberapa penelitian lanjutan perlu dilakukan. Meskipun begitu, penelitian ini telah memberikan dasar awal yang fundamental untuk dapat dikembangkan menjadi penelitian-penelitian linguistik transdisipliner lain yang membahas tentang hubungan tuntutan struktur sintaktis sebuah kalimat dengan keterbatasan memori kerja dalam kognisi manusia.

%---------------------------------------------------------------
\section{Pengembangan penelitian}
%---------------------------------------------------------------

Temuan yang berlawanan dalam data bahasa Indonesia ragam tulis dan lisan pada kalimat pendek dan panjang memunculkan beberapa pertanyaan untuk penelitian-penelitian lanjutan. Perbedaan struktur sintaktis akibat konstruksi kalimat kedua ragam memicu tinjauan lanjutan untuk mengukur struktur sintaktis secara kuantitatif dan akurat dalam usaha untuk lebih memahami tuntutan sintaktis dalam kognisi manusia. Terkait dengan valensi, penelitian ini belum mengukur secara akurat dan mengkuantifikasi analisis valensi karena kamus atau leksikon valensi dalam bahasa Indonesia belum ada. Sementara, konsep PVP \citep{liu2006syntactic} yang dimanfaatkan dalam penelitian ini serta analisis valensi dalam beberapa bahasa lain telah sampai pada analisis kuantitatif berskala besar dan pembentukan leksikon valensi terutama valensi verbal (\citealp{zabokrtsky2005valency, bielicky2008building, passarotti2016latin, semecky2006constructing}). Oleh karena itu, leksikon valensi terutama untuk kelas kata verba dalam bahasa Indonesia penting untuk dikembangkan dalam upaya untuk mendapatkan analisis valensi yang lebih mendalam terkait kemampuan sebuah konstituen untuk mengikat konstituen lain. 

Penelitian lanjutan yang melibatkan uji persepsi untuk meninjau ada tidaknya preferensi pendengar atau pembaca terhadap struktur sintaktis tertentu juga akan memberikan masukan wawasan linguistik yang sangat bermanfaat dalam mempelajari penampilan bahasa dengan memanfaatkan data bahasa Indonesia. Kajian lain juga dibutuhkan untuk dapat membuktikan apakah perbedaan ini bersifat kontekstual terhadap bahasa Indonesia atau bersifat lintas bahasa. Aspek penting lainnya yang perlu ditinjau adalah kemungkinan adanya perbedaan tingkat pemahaman bahasa antara pendengar (yang mendengar kalimat secara verbal) dan pembaca (yang membaca kalimat secara tertulis) yang dapat mempengaruhi pandangan terhadap struktur sintaktis yang lebih mudah dipahami. 