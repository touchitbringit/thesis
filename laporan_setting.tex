%-----------------------------------------------------------------------------%
% Informasi Mengenai Dokumen
%-----------------------------------------------------------------------------%
% 
% Judul laporan. 
\var{\judul}{Pola Pengurangan Jarak Dependensi}
% 
% Tulis kembali judul laporan, kali ini akan diubah menjadi huruf kapital
\Var{\Judul}{Pola Pengurangan Jarak Dependensi}
% 
% Tulis kembali judul laporan namun dengan bahasa Ingris
\var{\judulInggris}{Unknown Title for Final Report/Thesis/Disertation}

% 
% Tipe laporan, dapat berisi Skripsi, Tugas Akhir, Thesis, atau Disertasi
\var{\type}{Tesis}
% 
% Tulis kembali tipe laporan, kali ini akan diubah menjadi huruf kapital
\Var{\Type}{Tesis}
% 
% Tulis nama penulis 
\var{\penulis}{Lalitia Apsari}
% 
% Tulis kembali nama penulis, kali ini akan diubah menjadi huruf kapital
\Var{\Penulis}{Lalitia Apsari}
% 
% Tulis NPM penulis
\var{\npm}{XXXXXXXXXX}
% 
% Tuliskan Fakultas dimana penulis berada
\Var{\Fakultas}{Fakultas Ilmu dan Budaya}
\var{\fakultas}{Fakultas Ilmu dan Budaya}
% 
% Tuliskan Program Studi yang diambil penulis
\Var{\Program}{Linguistik Murni}
\var{\program}{Linguistik Murni}
%
% Tuliskan Kekhususan yang diambil penulis
\Var{\Kekhususan}{Linguistik Murni}
\var{\kekhususan}{Linguistik Murni}
% 
% Tuliskan tahun publikasi laporan
\Var{\bulanTahun}{Juli 2018}
% 
% Tuliskan gelar yang akan diperoleh dengan menyerahkan laporan ini
\var{\gelar}{Sarjana Linguistik}
% 
% Tuliskan tanggal pengesahan laporan, waktu dimana laporan diserahkan ke 
% penguji/sekretariat
\var{\tanggalPengesahan}{XX Januari 2010} 
% 
% Tuliskan tanggal keputusan sidang dikeluarkan dan penulis dinyatakan 
% lulus/tidak lulus
\var{\tanggalLulus}{XX Januari 2010}
% 
% Tuliskan pembimbing 
\var{\pembimbing}{Prof. XXXX}
% 
% Alias untuk memudahkan alur penulisan paa saat menulis laporan
\var{\saya}{Penulis}

%-----------------------------------------------------------------------------%
% Judul Setiap Bab
%-----------------------------------------------------------------------------%
% 
% Berikut ada judul-judul setiap bab. 
% Silahkan diubah sesuai dengan kebutuhan. 
% 
\Var{\kataPengantar}{Kata Pengantar}
\Var{\babSatu}{Pendahuluan}
\Var{\babDua}{Tinjauan Pustaka}
\Var{\babTiga}{Metode Penelitian}
\Var{\babEmpat}{Analisis}
%\Var{\babLima}{Perintah dalam uithesis.sty}
%\Var{\babEnam}{??}
\Var{\kesimpulan}{Kesimpulan}
