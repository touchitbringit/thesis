%-----------------------------------------------------------------------------%
\chapter{\babEmpat} \label{chap:analisis}
%-----------------------------------------------------------------------------%

%-----------------------------------------------------------------------------%
\section{Pendahuluan}
%-----------------------------------------------------------------------------%
Pada bab ini, analisis korpus data jurnalistik Bahasa Indonesia termutakhir ragam tulis dan lisan dijabarkan melalui beberapa tahap. Kedua korpus dianalisis secara terpisah agar temuan analisis dapat disandingkan untuk melihat perbedaan karakter serta signifikansi perbedaan tersebut antara data ragam tulis dan ragam lisan. Tahap pertama merupakan proses penguraian kalimat dengan metode komputasional berdasarkan dependensinya. Merujuk pada penjelasan dalam Bab 3, penguraian kalimat ini dilakukan dengan bantuan sebuah kerangka kerja yang dapat dilatih ulang untuk pengkategorian token, penguraian lemma, dan penguraian kalimat bernama UDPipe \citep{udpipe2017}. Tahap kedua adalah proses penghitungan untuk mendapatkan nilai panjang dependensi kalimat (DL) dengan pendekatan yang diadopsi dari penelitian \cite{gildea2010grammars} dan \cite{futrell2015large} serta nilai rata-rata jarak dependensi antarkonstituen (MDD) dengan pendekatan yang diadopsi dari penelitian \cite{liu2017dependency}. Kemudian, pada tahap ketiga dilakukan percobaan acak dengan mengadopsi prinsip langkah-langkah \textit{Free Word Order Baseline} \citep{futrell2015large} terhadap tabel yang dihasilkan dari tahap-tahap tersebut. Tahap terakhir melibatkan analisis secara kualitatif terhadap kalimat-kalimat di dalam kedua korpus yang mewakili aspek-aspek temuan penelitian berdasarkan analisis kuantitatif.

\begin{center}
\begin{table} \caption{Penggalan penguraian kalimat diambil dari korpus yang dianalisis}\label{tab:penggalan_kalimat}
\begin{tiny}
  \begin{tabulary}{1\textwidth}{| L | L | L | L | L | L | L |}
  \hline
    type & sentence & token & upos & relation & token id & head token id \\ \hline
tulis & Mengingat fungsi dari telepon seluler tersebut sebagai sarana komunikasi yang berdampak pada peningkatan perekonomian & Mengingat & VERB & root & 1 & 0 \\ \hline
tulis & Mengingat fungsi dari telepon seluler tersebut sebagai sarana komunikasi yang berdampak pada peningkatan perekonomian & fungsi & NOUN & obj & 2 & 1 \\ \hline
tulis & Mengingat fungsi dari telepon seluler tersebut sebagai sarana komunikasi yang berdampak pada peningkatan perekonomian & dari & ADP & case & 3 & 4 \\ \hline
tulis & Mengingat fungsi dari telepon seluler tersebut sebagai sarana komunikasi yang berdampak pada peningkatan perekonomian & telepon & NOUN & adl & 4 & 2 \\ \hline
tulis & Mengingat fungsi dari telepon seluler tersebut sebagai sarana komunikasi yang berdampak pada peningkatan perekonomian & seluler & NOUN & compound & 5 & 4 \\ \hline
tulis & Mengingat fungsi dari telepon seluler tersebut sebagai sarana komunikasi yang berdampak pada peningkatan perekonomian & tersebut & DET & det & 6 & 4 \\ \hline
tulis & Mengingat fungsi dari telepon seluler tersebut sebagai sarana komunikasi yang berdampak pada peningkatan perekonomian & sebagai & ADP & case & 7 & 8 \\ \hline
tulis & Mengingat fungsi dari telepon seluler tersebut sebagai sarana komunikasi yang berdampak pada peningkatan perekonomian & sarana & NOUN & adl & 8 & 2 \\ \hline
tulis & Mengingat fungsi dari telepon seluler tersebut sebagai sarana komunikasi yang berdampak pada peningkatan perekonomian & komunikasi & NOUN & compound & 9 & 8 \\ \hline
tulis & Mengingat fungsi dari telepon seluler tersebut sebagai sarana komunikasi yang berdampak pada peningkatan perekonomian & yang & PRON & nsubj & 10 & 11 \\ \hline
tulis & Mengingat fungsi dari telepon seluler tersebut sebagai sarana komunikasi yang berdampak pada peningkatan perekonomian & berdampak & VERB & acl:relcl & 11 & 8 \\ \hline
tulis & Mengingat fungsi dari telepon seluler tersebut sebagai sarana komunikasi yang berdampak pada peningkatan perekonomian & pada & ADP & case & 12 & 13 \\ \hline
tulis & Mengingat fungsi dari telepon seluler tersebut sebagai sarana komunikasi yang berdampak pada peningkatan perekonomian & peningkatan & NOUN & obl & 13 & 11 \\ \hline
tulis & Mengingat fungsi dari telepon seluler tersebut sebagai sarana komunikasi yang berdampak pada peningkatan perekonomian & perekonomian & NOUN & compound & 14 & 13 \\ 
\hline
  \end{tabulary}  
\end{tiny}
\end{table}
\end{center}

Pada tahap pertama, kedua korpus data jurnalistik dipersiapkan untuk dapat diolah dengan metode komputasional. Dalam data ragam lisan ditemukan beberapa kalimat yang tidak lengkap. Karena metode penguraian kalimat belum dapat mengolah kalimat seperti ini, kalimat-kalimat tidak lengkap dikeluarkan dari korpus terlebih dahulu agar tidak mempengaruhi hasil penghitungan kalimat lainnya. Hasil dari tahap penguraian kalimat ini berupa tabel yang berisi anotasi konstituen-konstituen kalimat seperti contoh pada \tab~\ref{tab:penggalan_kalimat}. Berdasarkan penguraian kalimat ini, kedua korpus memiliki total 19.530 kalimat atau 270.409 konstituen dengan data ragam tulis berjumlah 9311 kalimat atau 162.201 konstituen dan data ragam lisan berjumlah 10.219 kalimat atau 108.208 konstituen. Untuk memastikan kualitas penguraian kalimat dan anotasi yang sesuai, dilakukan tahap tambahan yaitu pemeriksaan secara manual. Berdasarkan pemeriksaan ini, kualitas penguraian kalimat untuk data ragam tulis terbilang sudah bagus, namun pada ragam lisan masih diperlukan pengembangan terutama untuk kata-kata informal atau sehari-hari. Contoh hasil penguraian pada \tab~\ref{tab:penggalan_kalimat} memperlihatkan kalimat \textit{Mengingat fungsi dari telepon seluler tersebut sebagai sarana komunikasi yang berdampak pada peningkatan ekonomi} yang telah diekstraksi menjadi kata-kata yang memiliki anotasi untuk kelas kata serta hubungan antarkonstituennya. 

\begin{figure}
	\centering \includegraphics[width=0.8
	\textwidth] {pics/visualisasi_penguraian.jpg} 
	\caption{Contoh visualisasi penguraian kalimat berdasarkan dependensi} 
\label{fig:visualisasi_penguraian} \end{figure}

Untuk memudahkan dalam analisis kualitatif dan melihat secara dekat tautan-tautan dependensi yang terbentuk pada sebuah kalimat, visualisasi seperti pada \pic~\ref{fig:visualisasi_penguraian} dilakukan secara terpisah. Proses visualisasi ini hanya dilakukan pada kalimat-kalimat yang akan dilihat lebih rinci sebagai contoh yang mewakili hasil temuan dengan bantuan kerangka kerja UDPipe \citep{udpipe2017}. Pada kalimat contoh ini, terlihat bahwa kalimat memiliki akar verbal \textit{mengingat} yang simpai pusatnya hanya terdiri dari satu tautan, yaitu \textit{mengingat fungsi} dengan hubungan tautan $obj$ atau obyek. Penelitian ini merupakan penelitian terhadap penggunaan bahasa secara nyata (\textit{real utterance}). Oleh sebab itu, meskipun beberapa kalimat yang ditemukan terlihat seperti klausa terikat saja, atau tampak seperti kalimat yang tidak lengkap, semua kalimat tetap dianalisis tanpa koreksi atau manipulasi.

\begin{center}
\begin{table} \caption{Penggalan penguraian kalimat diambil dari korpus yang dianalisis}\label{tab:dl_mdd}
\begin{tiny}
  \begin{tabulary}{1\textwidth}{| L | L | L | L | L | L | L | L |}
  \hline
type & sentence & dl & sum\textunderscore dl & dl\textunderscore positive & dl\textunderscore negative & mdd \\ \hline
tulis & Mengingat fungsi dari telepon seluler tersebut sebagai sarana komunikasi yang berdampak pada peningkatan perekonomian & 0 & 0 & 0 & 0 & 0.00 \\ \hline
tulis & Mengingat fungsi dari telepon seluler tersebut sebagai sarana komunikasi yang berdampak pada peningkatan perekonomian & 1 & 1 & 1 & 0 & 0.08 \\ \hline
tulis & Mengingat fungsi dari telepon seluler tersebut sebagai sarana komunikasi yang berdampak pada peningkatan perekonomian & -1 & 2 & 1 & -1 & 0.15 \\ \hline
tulis & Mengingat fungsi dari telepon seluler tersebut sebagai sarana komunikasi yang berdampak pada peningkatan perekonomian & 2 & 4 & 3 & -1 & 0.31 \\ \hline
tulis & Mengingat fungsi dari telepon seluler tersebut sebagai sarana komunikasi yang berdampak pada peningkatan perekonomian & 1 & 5 & 4 & -1 & 0.38 \\ \hline
tulis & Mengingat fungsi dari telepon seluler tersebut sebagai sarana komunikasi yang berdampak pada peningkatan perekonomian & 2 & 7 & 6 & -1 & 0.54 \\ \hline
tulis & Mengingat fungsi dari telepon seluler tersebut sebagai sarana komunikasi yang berdampak pada peningkatan perekonomian & -1 & 8 & 6 & -2 & 0.62 \\ \hline
tulis & Mengingat fungsi dari telepon seluler tersebut sebagai sarana komunikasi yang berdampak pada peningkatan perekonomian & 6 & 14 & 12 & -2 & 1.08 \\ \hline
tulis & Mengingat fungsi dari telepon seluler tersebut sebagai sarana komunikasi yang berdampak pada peningkatan perekonomian & 1 & 15 & 13 & -2 & 1.15 \\ \hline
tulis & Mengingat fungsi dari telepon seluler tersebut sebagai sarana komunikasi yang berdampak pada peningkatan perekonomian & -1 & 16 & 13 & -3 & 1.23 \\ \hline
tulis & Mengingat fungsi dari telepon seluler tersebut sebagai sarana komunikasi yang berdampak pada peningkatan perekonomian & 3 & 19 & 16 & -3 & 1.46 \\ \hline
tulis & Mengingat fungsi dari telepon seluler tersebut sebagai sarana komunikasi yang berdampak pada peningkatan perekonomian & -1 & 20 & 16 & -4 & 1.54 \\ \hline
tulis & Mengingat fungsi dari telepon seluler tersebut sebagai sarana komunikasi yang berdampak pada peningkatan perekonomian & 2 & 22 & 18 & -4 & 1.69 \\ \hline
tulis & Mengingat fungsi dari telepon seluler tersebut sebagai sarana komunikasi yang berdampak pada peningkatan perekonomian & 1 & 23 & 18 & -5 & 1.77 \\ \hline
  \end{tabulary}  
\end{tiny}
\end{table}
\end{center}

Tahap selanjutnya melibatkan proses anotasi nilai-nilai yang berkaitan dengan tautan dependensi sebuah kalimat pada tabel hasil penguraian kalimat. \tab~\ref{tab:dl_mdd} merupakan tabel hasil anotasi terhadap salah satu kalimat di dalam korpus data. Anotasi ini dihasilkan melalui penghitungan nilai DL dan penghitungan nilai MDD seperti yang dijabarkan pada Bab 2 dan rumus penghitungan yang dijabarkan pada Bab 3. DL menekankan pada total jumlah dependensi yang didapatkan dari semua tautan dependensi dalam sebuah kalimat, sedangkan MDD menekankan pada rata-rata jarak dependensi antara dua konstituen dalam sebuah kalimat. Meskipun memiliki prinsip mendasar yang serupa, saya berasumsi ada perbedaan informasi yang berarti antara kedua pendekatan ini sehingga keduanya digunakan dalam penelitian ini. Eksplorasi perbedaan pendekatan DL dan MDD serta penentuan pendekatan mana yang lebih optimal dalam mengilustrasikan efisiensi memori kerja melalui struktur sintaksis kalimat bukan merupakan fokus penelitian ini, namun hasil penerapan kedua pendekatan dalam penelitian ini dapat berkontribusi dalam menunjukkan pada kondisi mana kedua pendekatan tersebut berbeda.

\tab~\ref{tab:dl_mdd} memperlihatkan penghitungan nilai DL dan MDD untuk kalimat \textit{Mengingat fungsi dari telepon seluler tersebut sebagai sarana komunikasi yang berdampak pada peningkatan perekonomian}. Dengan jumlah konstituen 14, kalimat ini memiliki nilai DL 23 dan MDD 1,77. Hasil ini menunjukkan semua nilai tautan dependensi (total Jarak Dependensi atau DD antarkonstituen) dalam kalimat tersebut berjumlah 23 dengan rata-rata jarak dependensi antara dua konstituen yang memiliki jarak tautan langsung sejauh 1,77 konstituen. Pada tabel tersebut terlihat bahwa nilai DL juga dipisahkan menjadi dua berdasarkan arahnya. Nilai DL positif menggambarkan kondisi \textit{head-initial} yang berarti konstituen terikat (\textit{dependant}) direalisasikan setelah akar atau induk (\textit{root/head}). Sebaliknya, nilai DL negatif menggambarkan kondisi \textit{head-final} yang berarti konstituen terikat direalisasikan sebelum akar atau induk. Analisis hubungan nilai DL positif dan negatif ini memiliki dua bagian. Bagian pertama melihat tautan antara dua konstituen yang memiliki tautan langsung tanpa memperhitungkan posisi frasa atau klausa kedua konstituen tesebut. Analisis bagian kedua melihat tautan antara dua konstituen di mana salah satu konstituen tersebut adalah akar verbal (simpai pusat) dan menganggap semua simpai cabang konstituen tersebut mengikuti nilainya. Pada kalimat kompleks yang memiliki lebih dari satu klausa, tautan dependensi utama ini dapat membawa satu klausa atau lebih di dalamnya. Sebagai contoh, jika terdapat satu tautan dependensi utama yang negatif, dan dalam simpai cabang tersebut terdapat dua klausa, kedua klausa tersebut harus disimpan di dalam memori kerja terlebih dahulu hingga akar direalisasikan. Oleh sebab itu, posisi simpai cabang menentukan bagaimana seluruh informasi yang dikandungnya diproses. Analisis ini berangkat dari asumsi bahwa Bahasa Indonesia cenderung memilih bentuk tautan \textit{head-initial} yang diambil dari indikasi dari teori struktur frasa yang mengungkapkan banyaknya kondisi di mana induk akan mendahului konstituen terikatnya (\citealp{kridalaksana2002struktur, sneddon2010indonesian}).

\begin{figure}
	\centering \includegraphics[width=0.8
	\textwidth] {pics/percobaan_acak.jpg} 
	\caption{Contoh hasil percobaan acak} 
\label{fig:percobaan_acak} 
\end{figure}

Berikutnya, langkah yang dilakukan setelah mendapatkan tabel hasil penguraian ini adalah percobaan acak dengan mengadopsi prinsip langkah-langkah algoritma acak \textit{Free Word Order Baseline} \citep{futrell2015large}. \textit{Free Word Order Baseline} merupakan pendekatan yang dilakukan untuk \cite{futrell2015large} dalam menguji hipotesis adanya Pengurangan Panjang Dependensi (DLM). Pada tahap ini tidak dilakukan perbandingan untuk menguji hipotesis Pengurangan Jarak Dependensi (DDM) karena variabel pembagi, yang dalam hal ini adalah jumlah tautan, tidak berubah sehingga akan menghasilkan gambaran yang serupa dengan uji hipotesis DLM. Dari kalimat-kalimat yang telah diurai, percobaan ini menghasilkan 100 kombinasi struktur kalimat dengan mempertahankan tautan dependensi sesuai hasil observasi. \pic~\ref{fig:percobaan_acak} memperlihatkan tiga contoh hasil percobaan acak terhadap kalimat \textit{Sama, nanti pecahannya kita lihat} dan perbedaan nilai DL yang didapatkan tiga kombinasi struktur lainnya. Percobaan acak ini dilakukan terhadap semua kalimat yang berada pada kategori jumlah konstituen dengan frekuensi terbanyak. Berdasarkan klasifikasi kalimat terhadap kedua korpus data dan tahap pertama analisis, frekuensi jumlah konstituen terbanyak pada setiap klasifikasi kalimat adalah 10 konstituen (ragam tulis) dan 5 konstituen (ragam lisan) untuk klasifikasi kalimat pendek (\textless= 10 konstituen), 14 konstituen (ragam tulis) dan 12 konstituen (ragam lisan) untuk klasifikasi kalimat menengah (11-20 konstituen), serta 21 konstituen (ragam tulis) dan 22 konstituen (ragam lisan) untuk klasifikasi kalimat panjang (\textgreater 20 konstituen). 

Merujuk pada penjelasan dalam tinjauan pustaka pada Bab 2, bahasa Indonesia memiliki urutan kata yang cenderung bebas, dan beberapa kasus menunjukkan pertukaran kata maupun klausa tidak mengubah makna kalimat \citep{sneddon2010indonesian}. Dalam penelitiannya, \cite{futrell2015large} menunjukkan kedekatan hasil data observasi bahasa Indonesia dengan nilai DL minimum dengan menggunakan pendekatan ini untuk korpus data yang berisi artikel penelitian. Temuan ini memberikan indikasi bahwa penerapan kebebasan urutan kata tidak hanya terjadi pada kalimat yang dianggap gramatikal, tetapi terutama pada penggunaan bahasa secara nyata dalam kehidupan sehari-hari (data penampilan linguistik). Kebebasan terhadap aturan ini dan asumsi bahwa penutur bahasa Indonesia memanfaatkan kebebasan ini merupakan dasar utama pemilihan pendekatan \textit{Free Word Order Baseline}. Hal ini dikarenakan pendekatan tersebut menghasilkan hasil kombinasi struktur kalimat yang paling minimum dan maksimum sehingga sangat menarik untuk melihat posisi \textit{real utterance} yang didapat melalui observasi terhadap kemungkinan-kemungkinan kombinasi struktur yang terbentuk. 

Tahap terakhir dari analisis penelitian ini merupakan analisis kualitatif untuk melihat perilaku valensi konstituen yang difokuskan pada valensi verbal pada simpai pusat kalimat. Verba sebagai akar kalimat ditemukan sebanyak 84,67\% atau sebanyak 7884 kalimat pada data ragam tulis dan 69,26\% atau sebanyak 7078 kalimat pada data ragam lisan sehingga menjadikan bentuk ini mayoritas dalam kedua korpus data. Analisis ini dilakukan untuk melihat adanya strategi yang menyebabkan terjadinya DLM dan DDM pada level paling utama (simpai pusat) serta perbedaan karakter yang mungkin muncul antara kedua jenis ragam bahasa. Analisis kualitatif juga dilakukan pada temuan yang didapat pada tahap-tahap sebelumnya untuk menguraikan struktur kalimat yang mungkin memberikan informasi lebih mengenai kecenderungan penutur dalam membentuk kalimat yang efisien dari segi dependensi.

%-----------------------------------------------------------------------------%
\section{Panjang dependensi kalimat dan rata-rata jarak dependensi antarkonstituen antara data jurnalistik ragam lisan dan tulis}
%-----------------------------------------------------------------------------%
Pada analisis penelitian ini, kedua pendekatan untuk menghitung nilai DL dan nilai MDD digunakan karena diduga masing-masing memberikan informasi yang berbeda atau menunjukkan perbedaan nilai pada kondisi tertentu. Kedua penghitungan ini dilakukan terhadap setiap kalimat dalam korpus data jurnalistik ragam lisan dan tulis. Secara garis besar, ada perbedaan yang cukup terlihat dalam kedua paparan grafik nilai DL dan nilai MDD antara korpus data ragam tulis dan data ragam lisan.

%-----------------------------------------------------------------------------%
\subsection{Perbandingan panjang dependensi kalimat antara data jurnalistik ragam lisan dan tulis}
%-----------------------------------------------------------------------------%

Pada pendekatan penghitungan panjang dependensi kalimat (DL) yang diadopsi dari penelitian \cite{futrell2015large}, seluruh nilai tautan dependensi dalam sebuah kalimat dijumlahkan sehingga nilai DL yang dihasilkan cukup sensitif dan berbanding linear dengan jumlah konstituen dalam kalimat. Pada \pic~\ref{fig:lisantulis_DL} dapat dicermati bahwa terjadi perubahan pada kedua garis regresi. Perbandignan umum ini dilakukan tanpa melihat lebih dalam persebaran data terkait jumlah konstituen dalam kalimat atau tanpa memperhitungkan klasifikasi panjang kalimat. Berdasarkan grafik nilai DL ini, terlihat bahwa mayoritas jumlah konstituen dalam kalimat pada kedua korpus data berada di bawah 40 konstituen. Pada area di bawah 40 konstituen, garis regresi data ragam tulis berada di bawah garis regresi data ragam lisan. Namun, pada area di atas 40 konstituen, garis regresi menunjukkan perbandingan terbalik secara drastis. Secara umum, rata-rata nilai DL adalah 42,61 untuk data ragam tulis dan 24,36 untuk data ragam lisan. 

\begin{figure}
	\centering \includegraphics[width=1
	\textwidth] {pics/lisantulis_DL.png} 
	\caption{Grafik nilai DL data ragam tulis dan lisan} 
\label{fig:lisantulis_DL} 
\end{figure}

Tanpa memperhitungkan klasifikasi panjang kalimat, hasil pada \pic~\ref{fig:lisantulis_DL} menunjukkan adanya kecenderungan yang signifikan bahwa ragam lisan akan menghasilkan jumlah nilai tautan dependensi yang lebih kecil. Hal ini berarti ada indikasi bahwa data ragam lisan menunjukkan efisiensi yang lebih tinggi dibandingkan data ragam tulis dilihat dari segi total nilai dependensi dalam sebuah kalimat. Akan tetapi, seperti yang terlihat pada garis regresi yang terbentuk, hingga panjang kalimat 40 konstituen, garis regresi data ragam tulis terus berada di bawah garis regresi ragam lisan dan terjadi pemusatan data pada panjang kalimat yang mendekati minimum untuk data ragam lisan. Hal ini menunjukkan bahwa dugaan awal ini harus dicermati lebih dalam dengan melihat klasifikasi panjang kalimat yang mungkin akan memberikan informasi tambahan lain yang lebih menentukan keakuratan hasil uji hipotesis. 

%-----------------------------------------------------------------------------%
\subsection{Perbandingan rata-rata jarak dependensi antarkonstituen antara data jurnalistik ragam lisan dan tulis}
%-----------------------------------------------------------------------------%

Berbeda dengan penghitungan nilai DL, rata-rata jarak dependensi antarkonstituen (MDD) didapatkan dengan membagi total nilai tautan dependensi dalam sebuah kalimat dengan jumlah tautan itu sendiri. Penghitungan ini menghasilkan perkiraan rata-rata jarak dependensi satu (1) tautan dalam sebuah kalimat sehingga menggambarkan pada umumnya sejauh apa jarak relasi semantik antar dua konstituen. Seperti \pic~\ref{fig:lisantulis_DL}, \pic~\ref{fig:lisantulis_MDD} merupakan paparan nilai MDD untuk semua kalimat pada kedua korpus data (ragam tulis dan lisan). Nilai MDD, seperti yang dikemukakan oleh \cite{liu2017dependency}, tidak terlalu sensitif terhadap jumlah konstituen karena merupakan nilai rata-rata per tautan. Meskipun begitu, berdasarkan grafik nilai MDD ini, pada data dengan jumlah konstituen di bawah 10 nilai MDD masih menunjukkan hasil yang sangat linier terhadap jumlah konstituen. Hal ini berarti mengindikasikan bahwa pada panjang konstituen yang pendek, MDD memiliki tingkat sensitivitas yang lebih tinggi.

\begin{figure}
	\centering \includegraphics[width=1
	\textwidth] {pics/lisantulis_MDD.png} 
	\caption{Grafik nilai MDD data ragam tulis dan lisan} 
\label{fig:lisantulis_MDD} 
\end{figure}

Apabila nilai DL memberikan gambaran umum mengenai kompleksitas sebuah kalimat karena hanya fokus terhadap total nilai, nilai MDD dapat lebih memberikan ilustrasi kerja kognisi yang tercermin pada relasi dua buah konstituen dengan kondisi agar kedua konstituen dapat diproses, salah satu harus menunggu yang lain untuk direalisasikan. Pada \pic~\ref{fig:lisantulis_MDD}, dapat disimpulkan bahwa garis regresi untuk korpus data ragam tulis juga berada di bawah garis regresi korpus data ragam lisan dengan panjang kalimat di bawah 40 konstituen namun berbanding terbalik setelah 40 konstituen. Temuan ini menunjukkan konsistensi antara nilai DL dan nilai MDD yang sama-sama menunjukkan bahwa hingga jumlah setidaknya 40 konstituen, kalimat-kalimat dalam data ragam lisan menunjukkan efisiensi yang lebih tinggi dari segi dependensi dibandingkan data ragam tulis. Rata-rata nilai MDD yang didapat dari seluruh kalimat adalah 2,05 konstituen untuk data ragam lisan dan 2,35 konstituen untuk data ragam tulis. Seperti pada catatan grafik nilai DL sebelumnya, nilai ini tidak memperhitungkan klasifikasi panjang kalimat dan pemusatan data ragam tulis pada panjang kalimat yang mendekati minimum yang juga patut diperhitungkan.

%%-----------------------------------------------------------------------------%
\section{Perbandingan data hasil observasi dengan hasil percobaan acak}
%%-----------------------------------------------------------------------------%
Setelah mendapatkan gambaran umum mengenai nilai DL dan MDD untuk kedua korpus data, percobaan acak terhadap kedua korpus data dilakukan untuk melihat apakah kalimat-kalimat dalam data hasil observasi, baik ragam tulis maupun lisan, memiliki kecenderungan untuk menghindari nilai tautan dependensi yang lebih besar. Berdasarkan hipotesis DLM, \cite{futrell2015large} mengajukan bahwa jika bahasa berevolusi untuk mendukung komunikasi yang lebih mudah, seharusnya urutan kata yang dimanfaatkan penutur dalam penggunaan bahasa secara nyata sehingga tidak menghasilkan nilai panjang dependensi yang besar. Hal ini terjadi karena nilai panjang dependensi yang besar mencerminkan kompleksitas produksi ataupun pemahaman yang lebih tinggi. Percobaan acak dengan pendekatan \textit{Free Word Order Baseline} menghasilkan 100 bentuk urutan konstituen acak dengan mempertahankan struktur tautan-tautan dependensi yang sama dengan kalimat dalam data observasi. 

Beberapa histogram di bawah menunjukkan perbandingan antara nilai DL yang dihasilkan percobaan acak dan nilai DL dari semua kalimat dalam data observasi ragam tulis (\pic~\ref{fig:trandomobs}) serta ragam lisan (\pic~\ref{fig:lrandomobs}). Sumbu $X$ (horizontal) menandakan nilai-nilai DL yang muncul dan sumbu $Y$ menandakan persentase jumlah kalimat dengan nilai DL tersebut muncul. Perbandingan ini dilakukan dengan mengontrol jumlah konstituen sehingga hasil percobaan acak dapat memberikan nilai yang spesifik. \cite{futrell2015large} pernah menampilkan histogram serupa dengan jumlah konstituen sebanyak 12. Berdasarkan penelitiannya, nilai terkecil dari data bahasa Indonesia hasil observasi sangat mendekati nilai minimum dari hasil percobaan acak.

\begin{figure}
\centering

\begin{subfigure}{.45\textwidth}
  \centering
  \includegraphics[width=1\linewidth]{pics/t10randomobs.png}
  \caption{10 konstituen}
  \label{fig:t10randomobs} 
\end{subfigure}
%
\begin{subfigure}{.45\textwidth}
  \centering
  \includegraphics[width=1\linewidth]{pics/t14randomobs.png}
  \caption{14 konstituen}
  \label{fig:t14randomobs} 
\end{subfigure}
%
\begin{subfigure}{.45\textwidth}
  \centering
  \includegraphics[width=1\linewidth]{pics/t21randomobs.png}
  \caption{21 konstituen}
  \label{fig:t21randomobs} 
\end{subfigure}

\caption{Nilai DL hasil percobaan acak dengan observasi untuk panjang kalimat 10,14, dan 21 konstituen pada data ragam tulis}
\label{fig:trandomobs}
\end{figure}

\pic~\ref{fig:trandomobs} memperlihatkan histogram perbandingan nilai DL antara hasil percobaan acak dan data observasi untuk jumlah konstituen 10, 14, dan 21. Jumlah-jumlah konstituen ini memiliki frekuensi terbanyak yang mewakili setiap kelompok klasifikasi kalimat pada data ragam tulis.  Histogram hasil percobaan acak terlihat jelas membentuk kurva yang lebih rendah dibandingkan dengan kurva yang dibentuk oleh data observasi. Hal ini berarti hasil percobaan acak memiliki distribusi nilai DL yang lebih besar dibandingkan dengan data observasi. Berdasarkan histogram ini, dapat dilihat juga bahwa pada jumlah konstituen yang semakin banyak, irisan antara kurva nilai DL data observasi dan hasil percobaan acak semakin sedikit.

\begin{table}
\begin{center}
\begin{small}
  \caption{Tabel perbandingan rata-rata nilai DL hasil percobaan acak dengan data observasi dan hasil tes signifikansi untuk data ragam tulis}  \label{tab:perbandingan_DL_tulis}
  \begin{tabular}{ | l | l | l | l |}
    \hline
    	Jumlah konstituen & Rata-rata nilai DL (acak) & Rata-rata nilai DL (observasi) & p-value \\ \hline
	10 & 33,01 & 18,14 & \textless 0,001 \\ \hline
	14 & 64,95 & 29,11 & \textless 0,001 \\ \hline
	21 & 146,67 & 51,59 & \textless 0,001 \\ \hline
  \end{tabular}
  \end{small}
\end{center}
\end{table}

\tab~\ref{tab:perbandingan_DL_tulis} menunjukkan perbedaan rata-rata nilai DL yang cukup jauh antara hasil percobaan acak dan data observasi untuk data ragam tulis dengan efek yang signifikan pada semua jumlah konstituen (P \textless 0,001). 

\begin{figure}
\centering

\begin{subfigure}{.45\textwidth}
  \centering
  \includegraphics[width=1\linewidth]{pics/l5randomobs.png}
  \caption{5 konstituen}
  \label{fig:l5randomobs} 
\end{subfigure}
%
\begin{subfigure}{.45\textwidth}
  \centering
  \includegraphics[width=1\linewidth]{pics/l12randomobs.png}
  \caption{12 konstituen}
  \label{fig:l12randomobs} 
\end{subfigure}
%
\begin{subfigure}{.45\textwidth}
  \centering
  \includegraphics[width=1\linewidth]{pics/l22randomobs.png}
  \caption{22 konstituen}
  \label{fig:l22randomobs} 
\end{subfigure}

\caption{Nilai DL hasil percobaan acak dengan observasi untuk panjang kalimat 5,12, dan 22 konstituen pada data ragam lisan}
\label{fig:lrandomobs}
\end{figure}

Sesuai dengan ekspektasi, percobaan acak terhadap data ragam lisan menunjukkan hasil yang serupa dengan percobaan terhadap data ragam tulis. \pic~\ref{fig:lrandomobs} merupakan histogram perbandingan nilai DL antara hasil percobaan acak dan data observasi untuk jumlah konstituen 5, 12, dan 22 yang mewakili setiap kelompok klasifikasi kalimat pada data ragam lisan. Histogram hasil percobaan acak untuk ragam lisan juga membentuk kurva yang secara jelas terlihat lebih rendah dan juga menandakan bahwa  acak memiliki distribusi nilai DL yang lebih besar dibandingkan dengan data observasi. Bahkan untuk jumlah konstituen 5 yang tergolong kalimat sangat pendek, data observasi masih memperlihatkan adanya indikasi DLM secara signifikan dengan nilai P \textless 0,001. Berdasarkan histogram ini juga dapat disimpulkan bahwa pada jumlah konstituen yang semakin banyak, irisan antara nilai DL data observasi dan hasil percobaan acak semakin sedikit.

\begin{table}
\begin{center}
\begin{small}
  \caption{Tabel perbandingan rata-rata nilai DL hasil percobaan acak dengan data observasi dan hasil tes signifikansi untuk data ragam lisan}  \label{tab:perbandingan_DL_lisan}
  \begin{tabular}{ | l | l | l | l |}
    \hline
	 Jumlah konstituen & Rata-rata nilai DL (acak) & Rata-rata nilai DL (observasi) & \textit{p-value} \\ \hline
	 5 & 7,98 & 6,74 & \textless 0,001 \\ \hline
	 12 & 47,7 & 24,78 & \textless 0,001 \\ \hline
	 22 & 160,89 & 58,92 & \textless 0,001 \\ \hline
  \end{tabular}
  \end{small}
\end{center}
\end{table}

Serupa juga dengan percobaan terhadap data ragam tulis, \tab~\ref{tab:perbandingan_DL_lisan} menunjukkan perbedaan rata-rata nilai DL yang cukup jauh antara data observasi dan hasil percobaan acak untuk data ragam lisan dengan efek yang signifikan pada semua jumlah konstituen (P \textless 0,001).

Percobaan acak pada kedua korpus data dengan jumlah konstituen yang berbeda-beda menunjukkan adanya optimasi struktur kalimat pada data observasi sehingga terjadi pengurangan jarak dependensi (DLM) secara menyeluruh. Histogram hasil percobaan menunjukkan bahwa pada semua jumlah konstituen, distribusi nilai DL data observasi jauh lebih kecil mendekati jumlah minimum. Temuan ini sejalan dengan hipotesis penelitian \cite{hawkins2014cross} yang menyebutkan bahwa aturan urutan kata mendasar yang dalam pemahaman penutur akan menghasilkan konstruksi kalimat yang tautan dependensinya lebih pendek dibandingkan alternatif konstruksi yang mungkin ada.

Hipotesis pengurangan panjang atau jarak dependensi (DLM atau DDM) berbicara tentang kecenderungan manusia untuk mendekatkan kata-kata yang memiliki relasi semantik. Pendekatan ini dapat dilakukan dengan menerapkan beberapa strategi seperti yang telah dibahas pada penelitian-penelitian terdahulu (\citealp{jaeger2006redundancy, gildea2015human}). Dalam ranah sintaksis, strategi pendekatan-kata-kata ini termasuk melalui penyusunan urutan kata dan pengurangan kata dalam kalimat (yang mungkin diakibatkan oleh pengurangan unsur yang repetitif ataupun hal yang lain). Berdasarkan tahap pertama analisis, yaitu percobaan acak dengan menggunakan \textit{Free Word Order Baseline} \citep{futrell2015large} dihasilkan 100 kemungkinan struktur kalimat yang tidak memiliki aturan urutan kata tertentu. Percobaan ini mendukung hipotesis bahwa terjadi DLM pada kedua korpus data (ragam tulis maupun ragam lisan) secara signifikan (P \textless 0,001). 

Temuan DLM ini juga berkaitan dengan pengurangan jarak dependensi (DDM) karena percobaan acak tidak mengubah jumlah konstituen dalam kalimat yang sama sehingga variabel pembagi nilai DL sebuah kalimat tidak berubah. Temuan ini menandakan bahwa struktur kalimat-kalimat hasil observasi pada kedua ragam memperlihatkan struktur yang optimal dari segi dependensi dalam menghasilkan nilai DL yang lebih kecil dibandingkan kemungkinan struktur lain yang terbentuk. Perlu ditekankan bahwa struktur yang optimal ini dinilai dari segi dependensi dan bukan dari segi tata bahasa. Hal ini berarti terlepas dari standar gramatikal dan keberterimaannya, variasi struktur kalimat yang ada dalam kedua korpus rata-rata cukup mengoptimalkan memori kerja dan mengindikasikan kemudahan untuk berkomunikasi. Penentuan apakah sebuah struktur optimal dan juga gramatikal atau serta mudah dipahami berada di luar batasan penelitian ini. Metodologi lain seperti uji persepsi perlu dilakukan untuk mendapatkan temuan yang akurat. Penilaian dan pengukuran struktur yang lebih optimal dari segi produksi maupun pemahaman berada di luar batasan penelitian ini, sehingga perlu diadakan penelitian lebih lanjut yang melibatkan metodologi transdisipliner dengan ilmu kognitif serta uji persepsi.

%%-----------------------------------------------------------------------------%
\section{Persebaran kalimat dan kaitannya terhadap panjang dependensi dan rata-rata jarak dependensi}
%%-----------------------------------------------------------------------------%

Pada \pic~\ref{fig:lisantulis_DL}  dan \pic~\ref{fig:lisantulis_MDD} dalam pembahasan sebelumnya, terlihat jelas adanya perbedaan persebaran kalimat pada kedua korpus data terkait dengan panjang kalimat atau jumlah konstituen. Gambaran umum yang didapat dan garis regresi yang ditunjukkan pada kedua gambar tersebut menimbulkan pertanyaan yang hanya dapat dijawab melalui analisis dengan memperhitungkan panjang kalimatnya. Berdasarkan korpus data yang terkumpul, persebaran kalimat pada kedua ragam berbeda secara signifikan. 

\begin{table}
\begin{center}
\begin{small}
   \caption{Persentase kalimat pada tabel klasifikasi jumlah konstituen}  \label{tab:presentase_kalimat}
  \begin{tabular}{ |p{3cm} | p{3cm} | p{3cm} | p{3cm} |}
    \hline
 & Kalimat pendek \newline (\textless=10 konstituen) & Kalimat menengah (11-20 konstituen) & Kalimat panjang (\textgreater20 konstituen) \\ \hline
Data Ragam Tulis & 23,33\% & 45,54\% & 31,13\% \\ \hline
Data Ragam Lisan & 59,49\% & 29,16\% & 11,35\% \\ \hline
  \end{tabular}
  \end{small}
\end{center}
\end{table}

Pada data ragam tulis, sebanyak 45,54\% kalimat berada pada klasifikasi kalimat menengah (\tab~\ref{tab:presentase_kalimat}). Klasifikasi kalimat pendek dan panjang pada ragam tulis memiliki jumlah kalimat yang tidak jauh berbeda. \todo{Sedangkan}, sebanyak 59,49\% kalimat pada data ragam lisan berada pada klasifikasi kalimat pendek dan menurun secara drastis pada klasifikasi kalimat menengah dan panjang.  

\begin{figure}
	\centering \includegraphics[width=1
	\textwidth] {pics/Jumlah_kata.png} 
	\caption{Diagram perbandingan jumlah konstituen data ragam lisan dan tulis} 
	\label{fig:jumlah_kata} 
\end{figure}

Seperti yang terlihat pada \pic~\ref{fig:jumlah_kata}, terdapat kepadatan yang sangat tinggi pada panjang kalimat di bawah 10 konstituen untuk data ragam lisan, sementara kepadatan tertinggi pada data ragam lisan terletak pada kalimat dengan panjang menengah. Hal ini memperlihatkan bahwa persebaran panjang kalimat pada kedua korpus tidak boleh dihiraukan dalam menganalisis kaitannya dengan panjang dan jarak dependensi. Pada Panjang kalimat dengan frekuensi terbanyak pada data ragam tulis adalah 14 konstituen. \todo{Sedangkan} pada ragam lisan, sedangkan frekuensi terbanyak yang ditemukan pada ragam tulis adalah 5 konstituen.

%%-----------------------------------------------------------------------------%
\subsection{Pemusatan data ragam lisan pada klasifikasi kalimat pendek (\textless= 10 konstituen)}
%%-----------------------------------------------------------------------------%
Pemusatan data pada klasifikasi kalimat pendek (\textless= 10 konstituen) pada data ragam lisan cukup besar hingga melebihi 50\% dari total jumlah kalimat dalam korpus tersebut. Hal ini berdampak pada perbandingan rata-rata nilai DL serta MDD antara data ragam tulis dan lisan pada kelompok klasifikasi tersebut (\tab~\ref{tab:DL_MDD_pendek}).

\begin{table}
\begin{center}
\begin{small}
\caption{Perbandingan rata-rata nilai DL dan MDD pada klasifikasi kalimat pendek }\label{tab:DL_MDD_pendek}
  \begin{tabular}{ | p{3.2cm} | p{3.2cm} | p{3.2cm} | p{2cm} |}
    \hline
 & Kalimat pendek \newline (\textless =10 konstituen) \newline Data ragam tulis & Kalimat pendek \newline (\textless =10 konstituen) \newline Data ragam lisan & \textit{p-value} \\ \hline
 Rata-rata nilai DL & 11,85 & \textbf{8,76} & \textless 0,001 \\ \hline
 Rata-rata nilai MDD & 1,792 & \textbf{1,687} & \textless 0,001 \\ \hline
 Rata-rata jumlah konstituen & 7,356 & \textbf{5,708} & \textless 0,001 \\ \hline
 Jumlah kalimat & 2220 & 6073 & -- \\ \hline
   \end{tabular}
   \end{small}
\end{center}
\end{table}

Berdasarkan \tab~\ref{tab:DL_MDD_pendek}, Nilai DL dan MDD data ragam lisan lebih kecil dibandingkan data ragam tulis pada klasifikasi kalimat pendek secara signifikan. Berbeda dengan \pic~\ref{fig:lisantulis_DL}  dan \pic~\ref{fig:lisantulis_MDD} yang tidak memperhitungkan klasifikasi panjang kalimat, terutama pada nilai DL yang bersifat cukup linier terhadap panjang kalimat, rata-rata nilai DL pada data ragam lisan pada klasifikasi ini adalah 8,76. Nilai ini lebih rendah dibandingkan dengan nilai yang didapatkan dari data ragam tulis karena pemusatan jumlah konstituen yang mendekati minimum pada data ragam lisan akan mempengaruhi hasil yang didapat. Hal ini dapat dilihat dari perbandingan rata-rata jumlah konstituen dan jumlah kalimat yang muncul pada klasifikasi ini dengan perbedaan yang signifikan. Meskipun jumlah kalimat data ragam lisan pada klasifikasi ini jauh lebih banyak, rata-rata jumlah konstituen dalam satu kalimat lebih sedikit dibandingkan data ragam tulis.  Berkaitan dengan pemusatan data tersebut, rata-rata nilai MDD data ragam lisan juga menjadi lebih kecil dibandingkan data ragam tulis. Temuaan ini memberikan indikasi bahwa penggunaan kalimat pendek menjadi salah satu preferensi (dan mungkin strategi)  untuk meningkatkan efisiensi kalimat dalam percakapan lisan. Dalam beberapa kasus di korpus data ragam lisan, ditemukan kalimat-kalimat yang sebenarnya merupakan klausa terikat yang berdiri sendiri sebagai sebuah kalimat utuh dan lepas dari klausa pada kalimat sebelumnya. Seperti contoh, kalimat-kalimat ini banyak ditemukan ditandai dengan konjungsi \textit{dan} pada posisi pertama. 

\begin{table}
\begin{center}
\begin{small}
\caption{Perbandingan DL pada batas panjang kalimat 5 konstituen pada klasifikasi kalimat pendek}\label{tab:DL_5}
  \begin{tabular}{ | p{3.2cm} | p{3.2cm} | p{3.2cm} | p{2cm} |}
    \hline
Rentang & Data ragam tulis & Data ragam lisan & \textit{p-value} \\ \hline
\textless= 5 konstituen & 4,54 & \textbf{3,92} & \textless 0,001 \\ \hline
6-10 konstituen & 13,75 & 13,6 & \textgreater 0,05 \\ \hline
   \end{tabular}
   \end{small}
\end{center}
\end{table}

\begin{table}
\begin{center}
\begin{small}
\caption{Perbandingan MDD pada batas panjang kalimat 5 konstituen pada klasifikasi kalimat pendek}\label{tab:MDD_5}
  \begin{tabular}{ | p{3.2cm} | p{3.2cm} | p{3.2cm} | p{2cm} |}
    \hline
Rentang & Data ragam tulis & Data ragam lisan & \textit{p-value} \\ \hline
\textless= 5 konstituen & 1,443 & \textbf{1,402} & \textless 0,05 \\ \hline
6-10 konstituen & \textbf{1,883} & 1,972 & \textless 0,001 \\ \hline
   \end{tabular}
   \end{small}
\end{center}
\end{table}

Kepadatan data ragam lisan yang sangat tinggi pada klasifikasi ini memunculkan analisis tambahan untuk melihat perbandingan yang terjadi di dalam klasifikasi ini. Sebelumnya, \cite{wang2017effects} menemukan bahwa nilai MDD ragam teks yang lebih formal lebih kecil dibandingkan ragam teks yang ditujukan untuk ujaran lisan mulai dari panjang kalimat sebanyak 6 konstituen. Dari temuan ini, analisis tambahan dilakukan untuk meninjau perbandingan DL dan MDD pada panjang kalimat kurang dari dan lebih dari 5 konstituen. \tab~\ref{tab:DL_5} memperlihatkan bahwa perbedaan yang signifikan terkait total panjang dependensi sebuah kalimat hanya ditemukan pada klasifikasi panjang kalimat kurang dari atau sama dengan 5 konstituen. Namun, \tab~\ref{tab:MDD_5} memperlihatkan perbedaan yang signifikan terkait rata-rata jarak dependensi antara dua konstituen dalam kalimat pada kedua sub-klasifikasi, namun berbanding terbalik. 

\begin{figure}
\centering

\begin{subfigure}{.3\linewidth}
  \centering
  \includegraphics[width=1\linewidth] {pics/ts662.jpg} 
	\caption{T4}
	\label{fig:ts662} 
\end{subfigure}
%
\begin{subfigure}{.35\linewidth}
  \centering
  \includegraphics[width=1\linewidth]{pics/ls1102.jpg} 
	\caption{L4}
	\label{fig:ls1102} 
\end{subfigure}

\caption{Contoh perbandingan struktur sintaktis pada panjang kalimat dengan 4 konstituen}
\label{fig:struktur4}
\end{figure}

Panjang dan jarak dependensi ditemukan signifikan pada klasifikasi kalimat pendek, terutama pada panjang kalimat kurang dari atau sama dengan 5 konstituen. Pada contoh kalimat T4 (\pic~\ref{fig:ts662}) yang diambil dari ragam tulis dan kalimat L4 (\pic~\ref{fig:ls1102}) yang diambil dari ragam lisan, perbedaan yang jelas terlihat adalah tidak adanya aktor pelaku pada kalimat L4. Tingkat atau level dependensi yang semakin ke bawah dan antara dua kalimat tersebut berbeda di mana level dependensi T4 lebih banyak. DL dan MDD pada kalimat T4 (\pic~\ref{fig:ts662}) adalah 5 dan 1,67, sementara pada kalimat L4 (\pic~\ref{fig:ls1102}) adalah 4 dan 1,33. Namun, karena jumlah konstituen yang sangat sedikit, perbedaan struktur sintaktis tidak terlalu signifikan secara visual sehingga kurang bisa diasumsikan bahwa ada perbedaan karakteristik yang mendasar antara kedua ragam. 

%%-----------------------------------------------------------------------------%
\subsection{Perbedaan signifikansi panjang dan jarak dependensi pada klasifikasi kalimat menengah (11-20 konstituen)}
%%-----------------------------------------------------------------------------%
Pada kelompok klasifikasi kalimat menengah dengan panjang kalimat sebanyak 11 hingga 20 konstituen, terlihat adanya perbedaan dalam perbandingan nilai-nilai diantara kedua ragam. Pada klasifikasi ini ditemukan perbandingan terbalik antara rata-rata nilai DL dan MDD pada kedua korpus data (\tab~\ref{tab:DL_MDD_menengah}). Untuk nilai DL, data ragam lisan memiliki rata-rata yang sedikit lebih rendah, yaitu 33,05 dibandingkan data ragam tulis (33,32) namun perbedaan ini tidak signifikan (\textit{P} \textgreater 0,05). Hal ini berarti pemusatan kalimat ragam lisan yang banyak mendekati kalimat pendek atau persebaran kalimat ragam tulis di kelompok klasifikasi ini yang cukup merata mungkin tidak mempengaruhi panjang dependensi. Meskipun begitu, temuan yang menarik adalah bahwa rata-rata nilai MDD data ragam tulis (2,31) lebih rendah dibandingkan dengan data ragam lisan (2,404) secara signifikan. Jumlah kalimat kedua korpus data tidak jauh berbeda dan selisih rata-rata jumlah konstituen juga tidak sebesar pada klasifikasi kalimat pendek, namun perbedaannya signifikan. Hal ini berarti pada klasifikasi kalimat menengah, masih terlihat kecenderungan untuk menggunakan kalimat yang lebih pendek pada ragam lisan.  

\begin{table}
\begin{center}
\begin{small}
 \caption{Perbandingan rata-rata nilai DL dan MDD pada klasifikasi kalimat menengah}\label{tab:DL_MDD_menengah}  
 \begin{tabular}{| p{3.2cm} | p{3.2cm} | p{3.2cm} | p{2cm} |}
    \hline
 & Kalimat menengah \newline (11-20 konstituen) \newline Data ragam tulis & Kalimat menengah \newline (11-20 konstituen) \newline Data ragam lisan & \textit{p-value} \\ \hline
 Rata-rata nilai DL & 33,32 & 33,05 & \textgreater 0,05  \\ \hline
 Rata-rata nilai MDD & \textbf{2,31} & 2,404 & \textless 0,001 \\ \hline
 Rata-rata jumlah konstituen & 15,24 & \textbf{14,56} & \textless 0,001 \\ \hline
 Jumlah kalimat & 4210 & 2976 & -- \\ \hline
   \end{tabular}
   \end{small}
\end{center}
\end{table}

Terlihat pada (\tab~\ref{tab:DL_MDD_menengah}), rata-rata nilai DL dan MDD kedua korpus data yang berbanding terbalik menunjukkan adanya indikasi bahwa pada panjang kalimat tertentu, jumlah konstituen yang lebih sedikit tidak selalu berfungsi sebagai strategi untuk menghasilkan kalimat yang efisien dari segi dependensi. Indikasi pada kelompok klasifikasi ini menarik karena memperlihatkan bahwa meskipun jumlah nilai yang dihasilkan semua tautan dependensi lebih kecil, tidak berarti menggambarkan jarak antarkonstituen yang lebih kecil juga.

%%-----------------------------------------------------------------------------%
\subsection{Tidak ada kecenderungan jumlah konstituen pada klasifikasi kalimat panjang (\textgreater20 konstituen)}
%%-----------------------------------------------------------------------------%

Berbeda dengan kedua klasifikasi panjang kalimat sebelumnya, rata-rata nilai DL dan MDD pada data ragam tulis lebih kecil dibandingkan data ragam lisan (\tab~\ref{tab:DL_MDD_panjang}) dan perbedaan ini bersifat signifikan. Hal ini berarti terlepas dari persebaran kalimatnya yang cukup banyak (31,13\%) dari keseluruhan korpus data ragam tulis), kalimat dengan jumlah konstituen yang lebih banyak pada data ragam tulis cenderung memiliki struktur yang lebih efisien dibandingkan dengan data ragam lisan. Nilai MDD data ragam tulis lebih kecil dibandingkan data ragam lisan secara signifikan mulai dari jumlah konstituen sebanyak 6. \todo{maksudnya? :Temuan tersebut dan temuan pada klasifikasi ini menunjukkan bahwa mulai panjang kalimat tertentu, meskipun kompleksitas kalimat semakin tinggi (nilai DL makin tinggi), data ragam tulis menunjukkan struktur yang menyebabkan tautan antarkonstituen lebih pendek sehingga mendukung efisiensi kalimat. }

\begin{table}
\begin{center}
\begin{small}
\caption{Perbandingan rata-rata nilai DL dan MDD pada klasifikasi kalimat panjang}  \label{tab:DL_MDD_panjang}
\begin{tabular}{ | p{3.2cm} | p{3.2cm} | p{3.2cm} | p{2cm} |}
    \hline
 & Kalimat panjang \newline (\textgreater20 konstituen) \newline Data ragam tulis & Kalimat panjang \newline (\textgreater20 konstituen) \newline Data ragam lisan & \textit{p-value} \\ \hline
 Rata-rata nilai DL & \textbf{79,94} & 83,84 & \textless 0,01 \\ \hline
 Rata-rata nilai MDD & \textbf{2,844} & 3,018 & \textless 0,001 \\ \hline
 Rata-rata jumlah konstituen & 28,39 & 28,36 & \textgreater 0,05 \\ \hline
 Jumlah kalimat & 2876 & 1158 & -- \\ \hline
   \end{tabular}
   \end{small}
\end{center}
\end{table}

Berdasarkan tab~\ref{tab:DL_MDD_panjang}, jumlah kalimat antara keduanya cukup jauh berbeda. Tetapi, selisih rata-rata jumlah konstituen sangat kecil dan perbedaannya tidak signifikan. Hal ini berarti pada klasifikasi kalimat panjang berdasarkan data yang dikumpulkan, tidak ada kecenderungan atau preferensi terkait panjang kalimat pada kedua ragam. Signifikansi temuan ini konsisten hingga jumlah konstituen sebanyak 40 dalam arti hingga panjang kalimat tersebut, data ragam tulis menunjukkan efisiensi kalimat yang lebih tinggi dibandingkan data ragam lisan baik dari segi panjang dependensi keseluruhan kalimat maupun rata-rata jarak antara dua konstituten. Signifikansi ini melemah pada panjang kalimat di atas 40 konstituen. Hal ini mungkin disebabkan oleh persebaran jumlah kalimat kedua korpus pada rentang ini sangat sedikit dibandingkan rentang panjang kalimat yang lain.  

Untuk memberikan ilustrasi terhadap perbandingan struktur kalimat kelompok klasifikasi kalimat panjang, berikut adalah beberapa contoh kalimat dengan jumlah konstituen sebanyak 22 dan 31 yang diambil dari kedua korpus data, yaitu kalimat T22 (\pic~\ref{fig:ts3901}), T31a (\pic~\ref{fig:ts2079}) dan T31b (\pic~\ref{fig:ts2081}) yang diambil dari ragam tulis serta L22 (\pic~\ref{fig:ls6521}), , L31a (\pic~\ref{fig:ls1716}), L31b (\pic~\ref{fig:ls16}), dan L31c (\pic~\ref{fig:ls114}) yang diambil dari ragam lisan. Kelima kalimat berikut menunjukkan adanya karakter utama yang secara cukup jelas membedakan karakteristik struktur kalimat antara kedua ragam. 

\begin{figure}
	\centering \includegraphics[width=0.8
	\textwidth] {pics/ts3901.jpg} 
	\caption{Kalimat T22 pada data ragam tulis} 
	\label{fig:ts3901} 
\end{figure}

Konstituen \textit{sepakat} pada kalimat T22 (\pic~\ref{fig:ts3901}) merupakan akar yang memiliki tautan-tautan dependensi terhadap konstituen-konstituen terikatnya. Induk pada simpai cabang terdekat ini cenderung memiliki jumlah konstituen sedikit meskipun mengandung beberapa klausa terikat. Pada kalimat T22 ini terlihat banyak tautan dependensi dari dua konstituen yang berdampingan seperti pada relasi simpai-simpai antara konstituen \textit{para}, \textit{pihak}, dan \textit{sepakat}, antara konstituen \textit{meniadakan}, \textit{pungutan}, dan \textit{yang}, serta beberapa tautan lainnya. Berdasarkan diagram pohon dependensi ini juga, terlihat ada percabangan utama yang bersifat berlanjut antara \textit{sepakat}, \textit{meniadakan},\textit{pungutan}, \textit{proses}, \textit{perekrutan}, dan \textit{masuk}. Relasi berlanjut ini mencerminkan adanya percabangan searah meskipun bukan terhadap dua konstituen yang berdampingan. Kalimat T22 (\pic~\ref{fig:ts3901}) memiliki nilai DL sebesar 35 dan nilai MDD sebesar 1,667. Selisih antara nilai DL dari tautan-tautan dependensi positif dan negatif sebesar 19 dalam arti jumlah tautan dengan bentuk relasi induk sebelum konstituen terikat lebih banyak dan/atau berjarak lebih jauh dibandingkan bentuk relasi yang sebaliknya.


\begin{figure}
	\centering \includegraphics[width=1
	\textwidth] {pics/ts2079.jpg} 
	\caption{Kalimat T31a pada data ragam tulis} 
	\label{fig:ts2079} 
\end{figure}

Akar adjektiva \textit{penting} pada kalimat T31a (\pic~\ref{fig:ts2079}) mengikat verba \textit{memperbaiki} yang dihubungkan dengan konjungsi \textit{untuk} dalam kaidah dependensi. Induk pada simpai cabang terdekat ini juga cenderung memiliki jumlah konstituen sedikit meskipun mengandung beberapa klausa terikat. Pola ini terlihat berulang pada simpai-simpai cabang berikutnya dan mengakibatkan banyak simpai cabang lain yang tidak memiliki tautan langsung dengan akar dan banyak tautan antara dua konstituen yang berdampingan. Seperti kalimat T22, relasi antara simpai-simpai konstituen \textit{penting}, \textit{memperbaiki}, \textit{kinerja}, \textit{mendorong}, \textit{konsistensi}, \textit{berusaha}, dan \textit{dilakukan} bersifat berlanjut mencerminkan adanya percabangan searah meskipun bukan terhadap dua konstituen yang berdampingan. Kalimat T31a (\pic~\ref{fig:ts2079}) memiliki nilai DL sebesar 68 dan nilai MDD sebesar 2,267. Selisih antara nilai DL dari tautan-tautan dependensi positif dan negatif sebesar 16 yang juga berarti jumlah tautan dengan bentuk relasi induk sebelum konstituen terikat lebih banyak dan/atau berjarak lebih jauh dibandingkan bentuk relasi yang sebaliknya.

\begin{figure}
	\centering \includegraphics[width=1
	\textwidth] {pics/ts2081.jpg} 
	\caption{Kalimat T31b pada data ragam tulis} 
	\label{fig:ts2081} 
\end{figure}

Seperti kalimat T22 dan T31a, konstituen \textit{ditingkatkan} pada kalimat T31b (\pic~\ref{fig:ts2081}) juga merupakan akar yang memiliki beberapa tautan dependensi terhadap konstituen terikatnya. Pada kalimat ini, akar verbal \textit{ditingkatkan} mengikat verba lain \textit{mempercepat} yang juga dihubungkan dengan konjungsi tujuan \textit{untuk}. Juga seperti kalimat T31a, banyak simpai-simpai cabang pada kalimat T31b tidak memiliki tautan langsung dengan akar mengakibatkan ada percabangan searah. Kalimat T31b (\pic~\ref{fig:ts2081}) memiliki nilai DL 67 dan nilai MDD 2,233. Meskipun struktur kalimat T31b terlihat serupa dengan kalimat T31a dan nilai DL serta MDD cukup dekat, selisih nilai DL dari tautan-tautan dependensi positif dan negatif sangat kecil, yaitu hanya sebesar 3. Hal ini berarti jumlah bentuk relasi induk sebelum konstituen terikat dan jumlah bentuk relasi induk setelah konstituen terikat dalam kalimat tersebut cukup seimbang. 

Pada ketiga kalimat, terutama kalimat T22 dan T31a, terlihat pergerakan simpai dan percabangan yang semakin menurun ke arah sesudah akar atau ke arah kanan. Percabangan menurun ke kanan ini juga menunjukkan indikasi adanya strategi untuk menghasilkan dependensi positif (konstituen terikat sesudah induk) dan relasi antarkonstituen utama yang bersifat \textit{head-initial}. Struktur kalimat dengan karakteristik seperti pada \pic~\ref{fig:ts3901}, \pic~\ref{fig:ts2079} dan \pic~\ref{fig:ts2081} ini cukup umum dalam korpus data ragam tulis klasifikasi kalimat panjang. Namun, seperti yang terlihat pada perbandingan selisih nilai DL antara kalimat T31a dan T31b, dominasi percabangan searah pada kalimat T31a tidak selalu menghasilkan nilai lebih kecil dibandingkan kalimat T31b yang memiliki keseimbangan antara kedua bentuk relasi induk dan konstituen terikat. Hal ini mungkin disebabkan karena kalimat T31b memiliki jumlah tautan dependensi antara dua konstituen yang berdampingan lebih banyak dibandingkan T31a.

\begin{figure}
	\centering \includegraphics[width=1
	\textwidth] {pics/ls6521.jpg} 
	\caption{Kalimat L22 pada data ragam tulis} 
	\label{fig:ls6521} 
\end{figure}

Kalimat L22 (\pic~\ref{fig:ls6521}) memiliki akar verba \textit{melakukan} dan mengikat banyak konsituen terikat. Pada kalimat ini, terlihat bagaimana tautan dependensi dan percabangan jauh lebih menyebar dibandingkan contoh-contoh kalimat pada ragam tulis. Percabangan yang menyebar ini menyebabkan tautan-tautan dependensi berjarak jauh karena tautan dependensi antara dua konstituen yang berdampingan sedikit dibandingkan contoh-contoh pada ragam tulis. Dua percabangan utama terlihat pada simpai \textit{dilakukan} dan simpai pusat \textit{melakukan}. Penyebaran simpai-simpai ini juga menjadikan tingkat dependensi lebih dangkal dibandingkan contoh-contoh pada ragam tulis. Akar \textit{melakukan} juga tidak memiliki aktor pelaku seperti pada contoh kalimat L4 (pic~\ref{fig:ls1102}). Pada kedua kasus ini, aktor pelaku terhadap verba-verba tersebut mungkin baru akan dimengerti dengan memahami relasi pada tataran diskursus. Kalimat L22 ini memiliki nilai DL sebesar 57 dan nilai MDD sebesar 2,714. 

\begin{figure}
	\centering \includegraphics[width=1
	\textwidth] {pics/ls1716.jpg} 
	\caption{Kalimat L31a pada data ragam lisan} 
	\label{fig:ls1716} 
\end{figure}

Pada kalimat L31a (\pic~\ref{fig:ls1716}), simpai pusat dan simpai cabang pada verba \textit{dipaksa} memiliki banyak tautan yang cukup seimbang dibandingkan dengan contoh kalimat pada ragam tulis. Kalimat ini memperlihatkan adanya pembagian informasi secara besar seperti pada kalimat L22. Relasi akar verbal \textit{berdatangan} dengan verba \textit{dipaksa} memiliki tipe dependensi $parataxis$. Tipe ini menandakan relasi yang menyerupai koordinasi antar diskursus, dalam arti seluruh informasi yang dibawa oleh simpai konstituen terikat ini bisa menjadi kalimat terpisah. Kalimat L31a (\pic~\ref{fig:ls1716}) memiliki nilai DL sebesar 86 dan nilai MDD sebesar 2,867. 

\begin{figure}
	\centering \includegraphics[width=1
	\textwidth] {pics/ls16.jpg} 
	\caption{Kalimat L31b pada data ragam lisan}
	\label{fig:ls16} 
\end{figure}

Pada kalimat L31b (\pic~\ref{fig:ls16}), akar berada di posisi tengah kalimat. Seluruh informasi yang dibawa klausa tujuan dengan induk \textit{memastikan} direalisasikan sebelum akar verbal \textit{menanyakan}. Akar verbal ini juga mengikat verba \textit{dibantah} yang dihubungkan dengan konjungsi \textit{dan}. Kalimat ini tidak memiliki kandungan informasi yang bersifat diskursus seperti kalimat L31b, namun jumlah simpai-simpai cabang di bawah \textit{memastikan}, \textit{menanyakan}, dan \textit{dibantah} tersebar cukup merata. Kalimat L31b (\pic~\ref{fig:ls16}) memiliki nilai DL 67 dan nilai MDD 2,233. Seperti kalimat T31b, selisih nilai DL dari tautan-tautan dependensi positif dan negatif sangat kecil (sebesar 3) sehingga juga menandakan jumlah bentuk relasi induk sebelum konstituen terikat dan jumlah bentuk relasi induk setelah konstituen terikat yang cukup seimbang. 

\begin{figure}
	\centering \includegraphics[width=1
	\textwidth] {pics/ls114.jpg} 
	\caption{Kalimat L31c pada data ragam lisan}
	\label{fig:ls114} 
\end{figure}

Kalimat L31c (\pic~\ref{fig:ls114}) memiliki akar cenderung di posisi akhir seperti pada kalimat L22. Hal ini berarti semua simpai-simpai cabang di bawah \textit{hemat} dan \textit{Pak} harus disimpan di dalam memori kerja sebelum akar verbal \textit{kira} direalisasikan. Di bawah simpai \textit{Pak}, terlihat adanya indikasi percabangan searah, yaitu relasi antara \textit{Pak} dengan \textit{melakukan} dan \textit{melakukan} dengan \textit{peresmian}, dan seterusnya. Namun seluruh relasi ini berada di bawah tautan utama antara \textit{ada} dengan \textit{Pak}, sehingga terjadi percabangan yang bersifat kombinasi antara searah dan beda arah. Kalimat L31c(\pic~\ref{fig:ls114}) memiliki nilai DL 100 dan nilai MDD 3,333. Selisih nilai DL dari tautan-tautan dependensi positif dan negatif sangat besar yaitu sebesar -40. Nilai negatif ini menandakan jumlah bentuk relasi induk setelah konstituen terikat yang jauh lebih besar dibandingkan sebaliknya. Tampak pada \pic~\ref{fig:ls114}, nilai negatif yang besar ini terutama disebabkan oleh tautan-tautan dependensi pada simpai pusat yang sangat jauh.

Kalimat L22 (\pic~\ref{fig:ls6521}), L31a (\pic~\ref{fig:ls1716}), L31b (\pic~\ref{fig:ls16}), dan L31c(\pic~\ref{fig:ls114}) sedikit berbeda satu sama lain, namun memiliki karakter utama yang serupa yang membedakannya dengan contoh-contoh kalimat pada data ragam tulis. Salah satunya adalah perbedaan posisi akar pada \pic~\ref{fig:ls6521}, \pic~\ref{fig:ls1716}, \pic~\ref{fig:ls16}, dan \pic~\ref{fig:ls114}. Berdasarkan diagram-diagram pohon tersebut, perbedaan yang paling terlihat antara contoh-contoh pada ragam lisan dibandingkan dengan contoh-contoh pada ragam tulis adalah percabangannya yang terlihat lebih tidak konsisten. Meskipun tidak semua kalimat memiliki struktur seperti \pic~\ref{fig:ts2079} dan \pic~\ref{fig:ts2081}, mayoritas kalimat panjang pada data ragam tulis menunjukkan karakter yang lebih konsisten dalam hal pergerakan percabangannya. Pada data ragam lisan, terlihat banyak kombinasi antara percabangan searah dan beda arah serta letak posisi akar yang menyebar dari awal hingga akhir kalimat. Percabangan utama yang tersebar pada ketiga kalimat tersebut menunjukkan indikasi hubungan klausa yang tidak bersifat berlanjut, namun paralel. Kalimat L22 (\pic~\ref{fig:ls6521}), L31b (\pic~\ref{fig:ls16}) dan L31c (\pic~\ref{fig:ls114}) memiliki percabangan utama sebelum akar sehingga secara kolektif membentuk relasi dependensi utama yang negatif. Hal ini berarti seluruh informasi yang dibawa oleh klausa-klausa pada simpai dan percabangan sebelum akar harus disimpan dalam memori kerja dan menunggu realisasi akar yang mengandung informasi utama untuk dapat diproses oleh kognisi manusia \citep{hawkins2014cross}.

%%-----------------------------------------------------------------------------%
\subsection{Pengaruh panjang kalimat terhadap panjang dependensi kalimat dan rata-rata jarak dependensi antarkonstituen}
%%-----------------------------------------------------------------------------%
Berdasarkan percobaan acak yang dilakukan, langkah analisis selanjutnya adalah melihat aspek-aspek yang mungkin mempengaruhi struktur kalimat sehingga memungkinkan terjadinya DLM dan DDM. Secara garis besar, analisis-analisis lanjutan untuk melihat aspek-aspek ini menghasilkan temuan strategi yang berkaitan dengan panjang kalimat, pengaturan urutan kata terkait pendekatan dan penempatan posisi, serta pengurangan valensi. Ketiga aspek besar ini menjawab pertanyaan-pertanyaan penelitian yang diajukan pada Bab 1. Langkah analisis pada bagian ini mencoba mengeksplorasi bagaimana pengaruh panjang kalimat atau jumlah konstituen terhadap nilai DL dan MDD serta struktur kalimat yang terbentuk kedua ragam kalimat. Temuan utama yang muncul adalah dalam kondisi tertentu, perbandingan nilai DL dan MDD antara kedua korpus data memiliki karakteristik yang berbeda. Rata-rata jumlah konstituen pada data ragam tulis berada antara 17 hingga 18 (\textit{mean} = 17,423) dan antara 10 hingga 11 konstituen (\textit{mean} = 10,859) pada ragam lisan dengan perbedaan yang signifikan (\textit{P} \textless 0,001). Perbedaan ini menunjukkan bahwa ada kecenderungan penutur memilih jumlah konstituen yang lebih sedikit saat merealisasikan kalimat secara lisan. Klasifikasi kalimat pendek, menengah, dan panjang dilakukan berdasarkan penelitian terdahulu dari \cite{gildea2015human}, \cite{futrell2015large}, \cite{wang2017effects}, serta \cite{liu2017dependency}. Berdasarkan data masing-masing penelitian tersebut, temuan-temuan yang didapat menyebutkan bahwa ada perbedaan perilaku dan karakteristik sintaksis yang dipengaruhi panjang kalimat. Berdasarkan klasifikasi panjang kalimat yang dilakukan bertitik tolak dari temuan-temuan tersebut, penelitian ini juga menemukan hasil yang serupa. 

Pada klasifikasi kalimat pendek (maksimal 10 konstituen dalam sebuah kalimat), perbedaan nilai rata-rata DL dan MDD antara ragam tulis dan lisan tampak dipengaruhi oleh pemusatan data ragam lisan. Kalimat-kalimat dalam kedua korpus sama-sama mengalami optimasi urutan kata yang menyebabkan terjadinya DLM dan DDM.  Kecenderungan untuk menggunakan kalimat pendek menyebabkan data ragam lisan memiliki rata-rata nilai DL dan MDD yang lebih rendah secara signifikan terutama hingga panjang kalimat sebanyak 5 konstituen. Namun, pada pengamatan yang lebih dalam, ditemukan bahwa mulai jumlah konstituen sebanyak 6, data ragam tulis menunjukkan nilai MDD yang lebih rendah secara signifikan. Tinjauan secara kualitatif mengindikasikan adanya perbedaan kedalaman level dependensi antara kedua ragam pada contoh-contoh kalimat dengan jumlah konstituen sebanyak 4. Karakteristik struktur yang unik yang dapat membedakan keduanya secara signifikan tidak ditemukan. Struktur sintaksis dan pengaturan klausa bebas serta terikat pada kedua ragam juga tidak berbeda secara signifikan. Keserupaan ini mungkin disebabkan oleh keterbatasan jumlah konstituen sehingga banyak ditemukan kalimat dengan hanya satu klausa bebas saja menjadikan kompleksitas kalimat-kalimat tersebut masih rendah. Meskipun begitu, terlihat indikasi adanya pengurangan atau reduksi konstituen pada data ragam lisan. Tinjauan lebih dalam mengenai ini akan dibahas pada tahap analisis berikutnya.

Terdapat temuan yang menarik pada klasifikasi kalimat menengah, yaitu rata-rata nilai DL dan MDD antara kedua korpus data yang berbanding terbalik. Rata-rata nilai DL yang dihasilkan korpus data ragam lisan lebih kecil dibandingkan ragam tulis namun tidak berbeda secara signifikan. Sebaliknya, rata-rata nilai MDD yang didapatkan korpus data ragam tulis lebih kecil dibandingkan ragam lisan dan berbeda secara signifikan. Asumsi perbedaan kedua nilai ini merupakan alasan mengapa kedua pendekatan digunakan pada penelitian ini dan perbedaan ini dapat memberikan temuan awal untuk melihat kesesuaian kedua pendekatan dalam mengukur efisiensi memori kerja yang tercermin pada struktur sintaksis sebuah kalimat. Rata-rata nilai DL dan MDD yang berbanding terbalik ini mengindikasikan bahwa meskipun jumlah total nilai dependensi lebih besar, struktur kalimat yang lebih efisien dapat menghasilkan rata-rata jarak dependensi antarkonstituen yang lebih kecil. Dalam hal ini, data ragam tulis sudah mulai memperlihatkan struktur kalimat yang lebih efisien dibandingkan ragam lisan. Rata-rata jumlah konstituen antara kedua korpus data berbeda secara signifikan yang berarti pada klasifikasi ini, penutur masih memiliki kecenderungan atau preferensi untuk mengurangi panjang kalimat saat merealisasikan kalimat secara lisan. Berdasarkan temuan ini, muncul asumsi bahwa nilai DL berguna untuk menggambarkan kompleksitas sebuah kalimat secara umum dan nilai MDD berguna untuk mewakili bagaimana efisiensi struktur kalimat melalui tautan dependensi antarkonstituennya.

Pada klasifikasi kalimat panjang, kedua rata-rata nilai DL dan MDD pada korpus data ragam tulis lebih kecil dibandingkan ragam lisan dan berbeda secara signfikan. Hal ini berarti ada indikasi bahwa struktur kalimat-kalimat ragam tulis lebih efisien dilihat baik dari segi kompleksitas kalimat maupun jarak antarkonstituen yang dapat mewakili bentuk strukturnya. Perbedaan rata-rata jumlah konstituen antara kedua korpus tidak signifikan yang berarti paka klasifikasi ini, tidak ada kecenderungan atau preferensi penutur terhadap panjang kalimat terkait kedua ragam. Tinjauan lebih dekat dilakukan untuk mengamati struktur-struktur kalimat yang umum ditemukan pada kedua korpus data. Berdasarkan analisis yang bersifat kualitatif terhadap struktur kalimat-kalimat \pic~\ref{fig:ts3901}, \pic~\ref{fig:ts2079}, \pic~\ref{fig:ts2081}, \pic~\ref{fig:ls6521}, \pic~\ref{fig:ls1716}, \pic~\ref{fig:ls16}, dan \pic~\ref{fig:ls114}, terlihat ada perbedaan yang cukup mendasar antara kedua ragam. 

Pada struktur kalimat panjang ragam tulis, banyak ditemukan kalimat yang hanya memiliki satu simpai cabang terdekat meskipun mengandung banyak klausa terikat sehingga kompleksitas kalimat tergolong tinggi. Hal ini juga terlihat pada berbedaan level dependensi antara kedua ragam. Pengaturan urutan kata pada bentuk struktur sintaksis seperti pada ragam tulis mencerminkan percabangan searah dan mengakibatkan hubungan-hubungan antarklausanya bersifat berlanjut. Contoh-contoh pada ragam tulis juga menunjukkan jumlah tautan dependensi antara dua konstituen yang berdampingan yang lebih banyak dibandingkan pada ragam lisan. Berkaitan dengan teori dependensi \citep{tesniere1959elements, hawkins2014cross, gildea2010grammars}, hubungan seri ini dilakukan untuk menambahkan informasi secara bertahap sehingga tautan dependensinya lebih pendek. 

Struktur sintaksis yang ditemukan pada kalimat-kalimat ragam tulis ini juga memilki posisi akar relatif paling awal, sehingga seiring jalannya kalimat direalisasikan, penambahan informasi yang bertahap dapat langsung diproses secara kognitif karena akar (konstituen yang mengandung informasi utama) sudah direalisasikan sejak awal. Berbeda dengan ragam tulis, struktur-struktur kalimat dalam data ragam lisan ditemukan tidak konsisten dan berbeda satu sama lain. Posisi akar dapat berada di depan, tengah, maupun akhir dan banyak ditemukan kalimat dengan banyak simpai cabang pada level tautan dependensi utama. Relasi $parataxis$ dengan akar juga ditemukan yang berarti simpai cabang memiliki kualitas diskursus dan dapat menjadi kalimat lepas namun bergabung menjadi satu kalimat. Simpai-simpai cabang ini juga mengakibatkan klausa-klausa terikat tersebar secara paralel yang mungkin juga menyebabkan level dependensi pada ragam lisan ini lebih dangkal dibandingkan ragam tulis. Pada beberapa kalimat ditemukan kombinasi percabangan yang bersifat searah maupun beda arah, tetapi kemunculannya juga tidak konsisten. Terutama pada akar yang berada di akhir kalimat, klausa-klausa yang terikat pada simpai-simpai cabang sebelumnya harus disimpan dulu di memori kerja hingga akar yang mengandung informasi utama direalisasikan dalam kalimat. 

Secara umum, perbandingan analisis pada ketiga klasifikasi memberikan informasi yang berarti. Perbedaan yang ditemukan pada ketiganya menandakan bahwa panjang kalimat berpengaruh terhadap nilai DL dan MDD. Namun, perbedaan karakter struktur sintaksis pada korpus data ragam lisan dan tulisan baru terlihat jelas pada kalimat dengan kompleksitas lebih tinggi jumlah konstituen yang lebih banyak. \cite{wang2017effects} menemukan dalam penelitiannya bahwa data ragam tulis yang ditujukan untuk percakapan lisan (sebagai contoh dialog film) memiliki nilai MDD yang lebih kecil dibandingkan teks yang lebih formal. \cite{wang2017effects} menindaklanjuti kajian yang dilakukan \cite{hiranuma1999syntactic} dan \cite{liu2009chinese} mengajukan asumsi bahwa pada teks yang semakin formal (berdasarkan data yang digunakan) lebih sulit secara sintaktis. Dalam penelitian ini, temuan serupa terjadi pada klasifikasi kalimat pendek (\textless= 5 konstituen) di mana ragam lisan memiliki nilai DL dan MDD yang lebih kecil. Temuan serupa didapatkan oleh \cite{wang2017effects}, yang mengindikasikan bahwa kalimat-kalimat ragam lisan pada klasifikasi ini lebih mudah dimengerti secara sintaktis. Namun, asumsi ini tidak berlaku pada kalimat yang lebih panjang. 

Berdasarkan data yang dikumpulkan, pada klasifikasi menengah dan panjang, teks yang lebih formal (ragam tulis) justru menunjukkan efisiensi kalimat yang lebih tinggi mulai dari jumlah konstituen sebanyak 6 hingga 40. Merujuk pada \pic~\ref{fig:lisantulis_DL} dan \pic~\ref{fig:lisantulis_MDD}, kedua grafik menunjukkan gerak garis regresi yang berkebalikan antara ragam lisan dan tulis pada titik sekitar panjang kalimat sebanyak 40 konstituen. Namun, signifikansi yang melemah dan cenderung berbalik pada panjang kalimat di atas 40 konstituen mungkin disebabkan oleh persebaran data kedua korpus yang sangat sedikit pada rentang tersebut, terutama pada data ragam lisan. Penelitian lanjutan untuk rentang panjang kalimat tersebut membutuhkan data dengan jumlah yang cukup besar dan seimbang pada kedua korpus, namun kalimat yang sangat panjang cukup jarang ditemukan dalam ragam lisan. Hal ini dapat dikaitkan dengan kecenderungan pada ragam lisan untuk menggunakan kalimat-kalimat yang pendek. Analisis kualitatif terhadap perbedaan karakteristik sintaksis pada klasifikasi kalimat panjang memperlihatkan secara jelas bagaimana perbedaan struktur kalimat pada kedua ragam, terutama terkait kedalaman level dependensi, tautan dependensi dari dua konstituen yang berdampingan, dan reduksi sintaktik atau reduksi kata pada ragam lisan yang mungkin menyebabkan jumlah konstituen menjadi sedikit.

%%-----------------------------------------------------------------------------%
\section{Direksionalitas induk dan jumlah valensi}
%%-----------------------------------------------------------------------------%
Pada pembahasan temuan-temuan di atas, sudah dibahas sedikit mengenai struktur kalimat terkait posisi akar dan pola percabangannya. Seperti yang dijelaskan sebelumnya, belum ada konvensi mengenai arah dependensi terkait hierarki dan pengolahan dalam memori kerja. Nilai dependensi positif dihasilkan dari bentuk relasi induk sebelum konstituen terikat (\textit{head-initial}), dan sebaliknya, nilai dependensi negatif dihasilkan dari bentuk relasi induk setelah konstituen terikat (\textit{head-final}). Bahasa Indonesia memiliki urutan kata yang cenderung bebas \citep{sneddon2010indonesian}. Dalam aturan tata bahasa yang mengadopsi prinsip struktur frasa, banyak urutan kata dalam aturan terkait frasa yang memiliki kepala (\textit{head}) mendahului pewatas (\textit{modifier}) \citep{sneddon2010indonesian}. Akan tetapi, teori struktur frasa berbeda dengan dependensi dan belum ada penelitian yang melihat relasi induk dan konstituen terikat terkait arah dan posisi induknya (\textit{head directionality}). 

Hipotesis yang digunakan pada beberapa penelitian yang melihat Pengurangan Panjang Dependensi (DLM) \citep{futrell2015large} serta Pengurangan Jarak Dependensi (DDM) \citep{liu2017dependency} menyebutkan bahwa kebebasan urutan kata ini mungkin dimanfaatkan oleh penutur menjadi strategi utama dalam menghasilkan kalimat yang efisien terutama mengenai peletakan konstituen dalam urutan sebuah kalimat. Pada pembahasan ini, penghitungan nilai DL tautan dependensi yang bernilai positif (konstituen terikat setelah akar atau induk) dipisahkan dengan yang bernilai negatif (konstituen terikat sebelum akar atau induk). Pada bagian pertama analisis nilai DL positif dan negatif, kedua jenis nilai didapatkan dengan memisahkan total nilai dependensi positif dan negatif pada setiap kalimat. Nilai DL positif dan negatif ini memberikan gambaran kasus tautan akar atau induk dengan konstituen terikat secara general dan berkontribusi untuk membentuk asumsi kecenderungan karakteristik bahasa dilihat dari posisi induknya \citep{wang2017effects}. 

\begin{figure}
\centering

\begin{subfigure}{.8\linewidth}
  \centering
  \includegraphics[width=1\linewidth] {pics/tulis_DLposneg.png} 
	\caption{Data ragam tulis}
	\label{fig:tulis_DLposneg} 
\end{subfigure}
%
\begin{subfigure}{.8\linewidth}
  \centering
  \includegraphics[width=1\linewidth]{pics/lisan_DLposneg.png} 
	\caption{Data ragam lisan}
	\label{fig:lisan_DLposneg} 
\end{subfigure}

\caption{Grafik nilai DL positif dan negatif data ragam tulis dan lisan}
\label{fig:DL_posneg}
\end{figure}


Berdasarkan \pic~\ref{fig:tulis_DLposneg} dan \pic~\ref{fig:lisan_DLposneg}, kedua korpus data menunjukkan garis regresi tautan dependensi positif yang jauh berada di atas garis regresi tautan dependensi negatif dan pada kalimat dengan jumlah konstituen yang semakin banyak, jarak antara keduanya semakin besar. Seperti yang dijelaskan sebelumnya, nilai DL cukup sensitif terhadap panjang kalimat dan keduanya memiliki hubungan linear. Dalam sebuah kalimat, perbandingan jumlah bentuk relasi induk sebelum dan sesudah konstituen terikat saling berkaitan. Hal ini berarti ada pola kecenderungan preferensi terhadap hubungan akar atau induk sebelum konstituen terikat (\textit{head-initial}) dan menandakan kesesuaian data observasi dengan tata bahasa yang berlaku. Namun, pada grafik nilai DL ragam lisan (\pic~\ref{fig:lisan_DLposneg}), garis regresi tautan dependensi negatif menunjukkan adanya kurva parabola terbuka ke bawah sehingga muncul asumsi adanya nilai tertinggi atau ambang batas (\textit{threshold}) terhadap nilai DL negatif pada korpus data ragam lisan. Berdasarkan korpus data yang terkumpul, dugaan ambang batas tautan dependensi negatif ini menimbulkan asumsi bahwa mulai panjang kalimat tertentu, total tautan dependensi negatif dalam sebuah kalimat mungkin tidak akan melebihi nilai tertentu juga. 

\begin{table}
\begin{center}
\begin{small}
\caption{Frekuensi tautan dependensi positif dan negatif}  \label{tab:tautanposneg}
\begin{tabular}{ | p{2cm} | p{2cm} | p{2cm} | p{2cm} | p{2cm} |}
    \hline
Jumlah konstituen & Ragam tulis (+) & Ragam tulis (-) & Ragam lisan (+) & Ragam lisan (-) \\ \hline
\textless= 5 & 677 & 700 & 2841 & 2764 \\ \hline
6 - 10 & 7023 & 5715 & 7649 & 6918 \\ \hline
11 - 20 & 35597 & 24426 & 15554 & 12898 \\ \hline
21 - 30 & 31036 & 18163 & 7564 & 5939 \\ \hline
31 - 40 & 12904 & 6968 & 3232 & 2414 \\ \hline
\textgreater 40 & 6583 & 3240 & 1928 & 1426 \\ \hline
   \end{tabular}
   \end{small}
\end{center}
\end{table}

\tab~\ref{tab:tautanposneg} merupakan tabel yang berisi frekuensi tautan dependensi positif dan negatif dalam kedua korpus data. Frekuensi ini hanya menggambarkan jumlah kemunculan tautan dan tidak memiliki nilai jarak dependensi. Pada tabel ini, tautan-tautan dependensi tidak dikelompokkan per kalimat sehingga dapat memberikan gambaran jumlah tautan yang ada dalam keseluruhan korpus. Kedua ragam menunjukkan bahwa mulai dari panjang kalimat 6 konstituen, frekuensi tautan dependensi negatif lebih sedikit dibandingkan tautan dependensi positif. Hal ini menandakan bahwa pada kalimat yang semakin panjang, bentuk relasi induk sebelum konstituen terikat lebih banyak digunakan oleh penutur bahasa Indonesia. Seiring bertambahnya jumlah konstituen, rasio frekuensi tautan dependensi positif dan negatif semakin besar pada kedua ragam. Namun, perbedaan rasio dari setiap kategori jumlah konstituen pada ragam lisan tidak sejauh ragam tulis. Pada kategori kalimat terpanjang dalam ragam tulis, frekuensi tautan dependensi positif dapat mencapai dua kali tautan dependensi negatif. Sementara pada ragam lisan, perbandingan terbesar hanya mencapai 1,35 banding 1. Temuan pada \tab~\ref{tab:tautanposneg} menunjukkan indikasi bahwa kecenderungan bentuk relasi \textit{head-initial} lebih terlihat pada data ragam tulis.

\begin{table}
\begin{center}
\begin{footnotesize}
\caption{Tabel deskriptif antara dua konstituen}  \label{tab:deskriptif-konstituen}
\begin{tabular}{| c | llll | llll | l |}
\hline
 & \multicolumn{4}{c |}{+} & \multicolumn{4}{c |}{-} & \\  \cline{2-9}  
Panjang kalimat & min & med	& max & mean & min & med & max & mean & \\ \cline{1-10}  
\textless= 5 	& 1 & 1 & 4 & 1,409 	& 1 & 1 & 4 & 1,606 & \multirow{6}{*}{Tulis}\\
6 - 10 		& 1 & 1 & 9 & 1,866 	& 1 & 1 & 9 & 1,948 & \\
11 - 20 		& 1 & 1 & 19 & 2,468 & 1 & 1 & 19 & 2,147 & \\
21 - 30 		& 1 & 1 & 29 & 2,971 & 1 & 1 & 29 & 2,241 & \\ 
31 - 40 		& 1 & 1 & 39 & 3,597 & 1 & 1 & 37 & 2,299 & \\
\textgreater 40 & 1 & 1 & 68 & 4,22 	& 1 & 1 & 55 & 2,282 & \\ 
\hline
\textless= 5 	& 1 & 1 & 4 & 1,466 	& 1 & 1 & 4 & 1,534 & \multirow{6}{*}{Lisan}\\
6 - 10 		& 1 & 1 & 9 & 1,933 	& 1 & 1 & 9 & 2,071 & \\
11 - 20 		& 1 & 2 & 19 & 2,535 & 1 & 1 & 19 & 2,321 & \\
21 - 30 		& 1 & 2 & 28 & 3,263 & 1 & 1 & 27 & 2,5 & \\ 
31 - 40 		& 1 & 2 & 37 & 3,718 & 1 & 1 & 32 & 2,657 & \\
\textgreater 40 & 1 & 2 & 66 & 4,067 	& 1 & 1 & 46 & 2,438 & \\ 
\hline
   \end{tabular}
   \end{footnotesize}
\end{center}
\end{table}



\begin{table}
\begin{center}
\begin{footnotesize}
\caption{Tabel deskriptif MDD}  \label{tab:deskriptif-mdd}
\begin{tabular}{| c | llll | llll | l |}
\hline
 & \multicolumn{4}{c |}{+} & \multicolumn{4}{c |}{-} & \\  \cline{2-9}  
Panjang kalimat & min 	& med	& max 	& mean 	& min 	& med 	& max 	& mean 	& \\ \cline{1-10}  
\textless= 5 	& 1 		& 1 		& 3,5	 	& 1,372 	& 1 		& 1,5 	& 4	 	& 1,545 	&\multirow{6}{*}{Tulis}\\
6 - 10 		& 1 		& 1,667	& 5,2 	& 1,839 	& 1 		& 1,667 	& 9	 	& 1,895 	& 	\\
11 - 20 		& 1 		& 2,25 	& 7,25 	& 2,432 	& 1 		& 1,875 	& 10	 	& 2,117 	& 	\\
21 - 30 		& 1 		& 3,765 	& 10,278 	& 2,97 	& 1 		& 2 		& 12,75	& 2,212 	& 	\\ 
31 - 40 		& 1,458 	& 3,368 	& 9,933	& 3,595 	& 1 		& 2 		& 8		& 2,272 	& 	\\
\textgreater 40 	& 1,929 	& 3,756	& 10,567 	& 4,154 	& 1 		& 2 		& 7,111	& 2,258 	& 	\\ 
\hline
\textless= 5 	& 1 		& 1 		& 4	 	& 1,378 	& 1 		& 1,33 	& 4		& 1,456 	& \multirow{6}{*}{Lisan}\\
6 - 10 		& 1 		& 1,714	& 5,6 	& 1,867 	& 1 		& 1,8		& 9		& 2,003 	& \\
11 - 20 		& 1 		& 2,286 	& 9,625 	& 2,479 	& 1 		& 2 		& 6,818	& 2,255 	& \\
21 - 30 		& 1,4 	& 3	 	& 8,412 	& 3,252	& 1 		& 2,188	& 8,636	& 2,447 	& \\ 
31 - 40 		& 1,556 	& 3,512 	& 7,25	& 3,719 	& 1 		& 2,27	& 8,803	& 2,612 	& \\
\textgreater 40 	& 2,174 	& 3,865	& 6,895 	& 3,957 	& 1,059 	& 2,294	& 4,2		& 2,363 	& \\ 
\hline
   \end{tabular}
   \end{footnotesize}
\end{center}
\end{table}

\todo{analisis}






\begin{figure}
\centering

\begin{subfigure}{.4\linewidth}
  \centering
  \includegraphics[width=1\linewidth] {pics/ls1436.jpg} 
	\caption{L\textit{akar}a}
	\label{fig:ls1436} 
\end{subfigure}
%
\begin{subfigure}{.58\linewidth}
  \centering
  \includegraphics[width=1\linewidth]{pics/ls1460.jpg} 
	\caption{L\textit{akar}b}
	\label{fig:ls1460} 
\end{subfigure}

\caption{Contoh perbedaan penempatan klausa pada data ragam lisan}
\label{fig:akarcontoh}
\end{figure}


\pic~\ref{fig:ls1436} dan \pic~\ref{fig:ls1460} merupakan perbandingan kedua kalimat yang ditemukan pada korpus data ragam lisan. Kedua diagram pohon dependensi ini memberikan ilustrasi perbedaan letak akar dan klausa serta dampaknya terhadap nilai dependensi. Klausa terikat \textit{kalau boleh tahu} pada kalimat L\textit{akar}a (\pic~\ref{fig:ls1436}) dan L\textit{akar}b (\pic~\ref{fig:ls1460}) berada pada posisi yang berbeda namun tidak mengubah makna kalimat. Contoh serupa juga sempat dijabarkan oleh \cite{sneddon2010indonesian} dalam buku \textit{Indonesian Reference Grammar} Pada kalimat L\textit{akar}b, konstituen-konstiuen yang membentuk klausa \textit{kalau boleh tahu} secara kolektif memiliki relasi dependensi utama negatif terhadap akar \textit{ordernya}. Sesuai dengan bentuk relasi induk dan konstituen terikat berdasarkan teori dependensi yang dijabarkan oleh \cite{tesniere1959elements}, klausa terikat pada posisi awal kalimat ini harus disimpan dalam memori kerja terlebih dahulu hingga akar direalisasikan. Hal in berkebalikan dengan L\textit{akar}a yang dapat diproses langsung karena akar sudah direalisasikan terlebih dahulu. 

\begin{figure}
\centering

\begin{subfigure}{.8\linewidth}
  \centering
  \includegraphics[width=1\linewidth] {pics/tulisroot_DLposneg.png} 
	\caption{Data ragam tulis}
	\label{fig:tulisroot_DLposneg} 
\end{subfigure}
%
\begin{subfigure}{.8\linewidth}
  \centering
  \includegraphics[width=1\linewidth]{pics/lisanroot_DLposneg.png} 
	\caption{Data ragam lisan}
	\label{fig:lisanroot_DLposneg} 
\end{subfigure}

\caption{Grafik nilai DL positif dan negatif pada simpai pusat data ragam tulis dan lisan}
\label{fig:rootDL_posneg}
\end{figure}

Bagian kedua analisis nilai DL positif dan negatif berkaitan dengan valensi sebuah kata sehubungan dengan contoh penempatan klausa terikat \textit{kalau boleh tahu} di atas (\pic~\ref{fig:akarcontoh}). Sesuai dengan batasan penelitian ini, bagian analisis ini difokuskan hanya pada simpai pusat dengan kelas kata akar berupa kata kerja atau verba. \pic~\ref{fig:tulisroot_DLposneg} dan \pic~\ref{fig:lisanroot_DLposneg} merupakan grafik nilai DL positif dan negatif yang didapat hanya pada simpai pusat kalimat dengan akar berupa verba. Kalimat dengan akar verbal ditemukan sebanyak 84,67\% atau sebanyak 7879 kalimat pada data ragam tulis dan 69,26\% atau sebanyak 7070 kalimat pada data ragam lisan. Mayoritas frekuensi kemunculan akar verbal ini dianalisis untuk melihat bagaimana penutur menyusun informasi utama dengan melihat tautan-tautan dependensi utama (simpai pusat). Oleh sebab itu, semua klausa yang mungkin terikat pada simpai cabang dianggap mengikuti induknya seperti pada contoh uraian relasi klausa \textit{kalau boleh tahu} dan akarnya di atas (\pic~\ref{fig:akarcontoh}). \pic~\ref{fig:rootDL_posneg} menunjukkan adanya konsistens pada level tautan dependensi utama dan keseluruhan kalimat. Grafik ini memperlihatkan bahwa penutur juga cenderung memilih untuk menekan penggunaan bentuk relasi induk setelah konstituen terikatnya. Serupa dengan \pic~\ref{fig:DL_posneg}, terutama pada data ragam lisan, garis regresi nilai DL negatif mengindikasikan adanya ambang batas pada nilai tertentu.

\begin{table}
\begin{center}
\begin{small}
\caption{Frekuensi tautan dependensi positif dan negatif pada simpai pusat akar verbal}  \label{tab:tautanpusatposneg}
\begin{tabular}{ | p{2cm} | p{2cm} | p{2cm} | p{2cm} | p{2cm} |}
    \hline
Jumlah konstituen & Ragam tulis (+) & Ragam tulis (-) & Ragam lisan (+) & Ragam lisan (-) \\ \hline
\textless= 5 & 126 & 225 & 849 & 1517 \\ \hline
6 - 10 & 1355 & 1777 & 2114 & 3126 \\ \hline
11 - 20 & 5383 & 5602 & 3352 & 3928 \\ \hline
21 - 30 & 3737 & 3144 & 1318 & 1237 \\ \hline
31 - 40 & 1357 & 912 & 381 & 325 \\ \hline
\textgreater 40 & 569 & 300 & 222 & 182 \\ \hline
   \end{tabular}
   \end{small}
\end{center}
\end{table}

Berbeda dengan tabel frekuensi tautan dependensi positif dan negatif secara keseluruhan (\tab~\ref{tab:tautanposneg}), \tab~\ref{tab:tautanpusatposneg} menunjukkan frekuensi tautan dependensi positif dan negatif hanya pada simpai pusat yang memiliki akar verbal. Sebagai contoh, simpai pusat kalimat L4 \textit{Hasil surveynya cukup mengejutkan} pada \pic~\ref{fig:ls1102} memiliki 3 konstituen yang saling bertautan, yaitu \textit{mengejutkan} dengan \textit{hasil} dan \textit{mengejutkan} dengan \textit{cukup}. Kedua tautan memiliki bentuk relasi induk setelah konstituen terikat, sehingga frekuensi dalam kalimat ini adalah 2 tautan dependensi negatif. \tab~\ref{tab:tautanpusatposneg} memperlihatkan bahwa pada simpai pusat akar verbal, kecenderungan untuk menggunakan bentuk relasi \textit{head-initial} baru terlihat pada panjang kalimat mulai 21 konstituen. Pada kalimat-kalimat yang lebih pendek dari itu, tautan dependensi negatif yang merepresentasikan bentuk relasi \textit{head-final} justru lebih banyak. 

\begin{table}
\begin{center}
\begin{small}
\caption{Frekuensi selisih DL positif dan negatif pada simpai pusat akar verbal}  \label{tab:DLpusatposneg}
\begin{tabular}{ | p{2cm} | p{2cm} | p{2cm} | p{2cm} | p{2cm} |}
    \hline
Jumlah konstituen & Ragam tulis (+ \textgreater -) & Ragam tulis (+ \textless -) & Ragam lisan (+ \textgreater -) & Ragam lisan (+ \textless -) \\ \hline
\textless= 5 & 33 & 109 & 277 & 711 \\ \hline
6 - 10 & 342 & 555 & 462 & 956 \\ \hline
11 - 20 & 1266 & 1412 & 680 & 942 \\ \hline
21 - 30 & 862 & 675 & 271 & 205 \\ \hline
31 - 40 & 295 & 159 & 85 & 60 \\ \hline
\textgreater 40 & 126 & 37 & 43 & 21 \\ \hline
   \end{tabular}
   \end{small}
\end{center}
\end{table}

\tab~\ref{tab:DLpusatposneg} berisi jumlah-jumlah kalimat dengan melihat apakah kalimat tersebut memiliki nilai DL postif yang lebih besar atau sebaliknya. Hal ini berarti kalimat yang memiliki nilai DL positif dan negatif sama besar tidak diperhitungkan. Seperti pada \tab~\ref{tab:tautanpusatposneg}, kalimat yang nilai DL positifnya lebih besar ditemukan lebih banyak pada kalimat-kalimat panjang (mulai 21 konstituen). 

%%-----------------------------------------------------------------------------%
\subsection{Percabangan searah dan keseimbangan antara tautan dependensi positif dan negatif}
%%-----------------------------------------------------------------------------%
Pada tinjauan pustaka, pergerakan panah yang merepresentasikan arah dependensi dibahas menjadi dua tipe, yaitu percabangan searah (\textit{same-brancing}) dan percabangan beda arah (\textit{mixed-branching}) \citep{hawkins1994performance}. Namun, dengan jumlah konstituen yang semakin banyak dan kalimat yang semakin kompleks, arah dependensi tidak bisa digolongkan secara biner (hanya percabangan searah saja atau hanya percabangan beda arah saja) \citealp{dryer1992greenbergian, temperley2008dependency}. \cite{gildea2010grammars} menjelaskan mengenai konsep keseimbangan antara tautan dependensi positif dan negatif dalam mendapatkan nilai DL terkecil yang juga dianalisis oleh \citep{dryer1992greenbergian}. Berdasarkan data penelitian ini, temuan yang didapatkan mendukung asumsi yang dikemukakan \cite{dryer1992greenbergian} bahwa arah dependensi tidak bisa digolongkan secara biner pada tataran kalimat, terutama pada kalimat yang semakin panjang. Namun, berdasarkan analisis dengan pemetaan densitas selisih antara nilai DL positif dan negatif, belum ditemukan kecenderungan terhadap bentuk keseimbangan tersebut. 

Pada ragam tulis, pemetaan densitas sangat menyebar sehingga tidak membentuk pola tertentu, sedangkan pada ragam lisan, pemetaan densitas terlalu memusat pada kalimat dengan jumlah konstituen sedikit sehingga juga tidak membentuk pola tertentu. Untuk mencari kecenderungan pada data seperti ini, perlu dilakukan pencarian fungsi matematis untuk menguji tingkat kesalahan (\textit{error rate}) yang lebih kecil antara kedua kemungkinan (percabangan searah atau keseimbangan dependensi positif dan negatif). Namun, pencarian fungsi matematis tersebut berada di luar batasan penelitian ini dan di luar ranah linguistik sehingga perlu dilakukan analisis lebih lanjut dari segi statistika. Asumsi sementara yang dapat diambil bertitik tolak dari grafik perbandingan nilai DL positif dan negatif di atas (\pic~\ref{fig:DL_posneg} dan \pic~\ref{fig:rootDL_posneg}) yang dapat memberikan gambaran adanya kecenderungan untuk menghindari tautan dependensi negatif, terutama pada kalimat yang semakin panjang.

%%-----------------------------------------------------------------------------%
\subsection{Posisi akar dan klausa utama di awal kalimat tanpa pengurangan valensi}
%%-----------------------------------------------------------------------------%
Penempatan akar dan/atau klausa bebas utama di awal kalimat tanpa mengurangi valensi verbal cukup umum ditemukan pada kedua korpus data. Penempatan akar verbal pada posisi pertama seperti kalimat T\textit{akar}a \pic~ref{fig:ts770} dan T\textit{akar}b \pic~\ref{fig:ts7458} ditemukan pada korpus data ragam tulis klasifikasi kalimat pendek dan menengah. Sebaliknya, penempatan akar di posisi pertama tanpa mengurangi valensi verbal tersebut sangat jarang ditemukan pada data ragam lisan.

\begin{figure}
\centering

\begin{subfigure}{.42\linewidth}
  \centering
  \includegraphics[width=1\linewidth] {pics/ts770.jpg} 
	\caption{T\textit{akar}a}
	\label{fig:ts770} 
\end{subfigure}
%
\begin{subfigure}{.57\linewidth}
  \centering
  \includegraphics[width=1\linewidth]{pics/ts7458.jpg} 
	\caption{T\textit{akar}b}
	\label{fig:ts7458} 
\end{subfigure}
\caption{Contoh penempatan akar verbal dan klausa bebas utama di awal kalimat tanpa mengurangi valensi verbal pada data ragam tulis}
\label{fig:rootnovalensi}
\end{figure}

Pada kalimat T\textit{akar}a \pic~ref{fig:ts770}, akar verbal \textit{dibuka} tetap mengikat konstituen terikat obyek \textit{kelas}. Akar verbal \textit{dibuka} pada kalimat tersebut tergolong verba statif (keadaan) sehingga tidak diperlukan kemunculan aktor pelaku dalam kalimat terebut \cite{sneddon2010indonesian}. Serupa dengan kalimat T\textit{akar}a, akar verbal \textit{menggunakan} pada kalimat T\textit{akar}b \pic~ref{fig:ts7458} tetap mempertahankan valensi terhadap \textit{komuter} (aktor pelaku) dan \textit{apa} (pengganti obyek) namun konstruksinya tidak umum seperti halnya aktor pelaku mendahului verba dalam bahasa Indonesia. Contoh ini memberikan indikasi pemanfaatan kebebasan urutan kata dalam Bahasa Indonesia untuk menghindari bentuk relasi induk setelah konstituen terikat dan tetap merealisasikan akar seawal mungkin. Penempatan klausa bebas utama di awal kalimat lebih banyak ditemukan pada data ragam tulis dibandingkan ragam lisan. Penempatan klausa bebas utama di awal kalimat ini menyebabkan akar verbal secara otomatis tetap berada di awal tanpa mengurangi valensi akar verbal tersebut. Namun, karena jumlah konstituen yang panjang dan struktur kalimat dengan klausa-klausa yang lebih kompleks, posisi akar verbal umumnya tidak pada posisi pertama. Pada kalimat seperti ini, konstituen terikat berupa aktor pelaku umumnya direalisasikan sebelum akar verbal. Contoh strategi ini dapat dilihat pada kalimat T31b (\pic~\ref{fig:ts2081}) untuk data ragam tulis dan L31a (\pic~\ref{fig:ls1716}) untuk data ragam lisan.

%%-----------------------------------------------------------------------------%
\subsection{Posisi akar dan klausa utama di awal kalimat dengan pengurangan valensi}
%%-----------------------------------------------------------------------------%
Sejumlah simpai pusat dalam kedua korpus data mengalami pengurangan valensi verbal yang mengakibatkan pengurangan panjang serta jarak dependensi. Kalimat seperti ini sangat jarang ditemui pada data ragam tulis, namun banyak ditemukan pada data ragam lisan, terutama pada klasifikasi kalimat pendek dan menengah. Strategi pengurangan valensi verbal ini banyak ditemukan dengan penempatan akar di awal kalimat seperti pada temuan sebelumnya. Hal ini juga berkaitan dengan temuan pengaruh panjang kalimat di atas yang menandakan kecenderungan jumlah konstituen yang lebih sedikit pada data ragam lisan.  

\begin{figure}
\centering

\begin{subfigure}{.49\linewidth}
  \centering
  \includegraphics[width=1\linewidth] {pics/ls4820.jpg} 
	\caption{L\textit{akar}c}
	\label{fig:ls4820} 
\end{subfigure}
%
\begin{subfigure}{.41\linewidth}
  \centering
  \includegraphics[width=1\linewidth]{pics/ls1435.jpg} 
	\caption{L\textit{akar}d}
	\label{fig:ls1435} 
\end{subfigure}
\caption{Contoh penempatan akar verbal dan klausa bebas utama di awal kalimat tanpa mengurangi valensi verbal pada data ragam tulis}
\label{fig:rootvalensi}
\end{figure}

Pada kalimat L\textit{akar}c (\pic~\ref{fig:ls4820}, akar verbal \textit{membutuhkan} umumnya memiliki valensi sebanyak dua konstituen, yaitu aktor pelaku dan obyek atau \textit{X membutuhkan Y}. Kata ini mengalami pengurangan valensi akar verbal sehingga yang muncul hanya obyek nomina \textit{waktu}. Begitu pula dengan kalimat L\textit{akar}d (\pic~\ref{fig:ls1435} yang juga mengalami pengurangan valensi akar verbal berupa aktor pelaku. Kalimat L\textit{akar}e (\pic~\ref{fig:ls1265}) merupakan contoh lain yang mengalami pengurangan valensi verbal berupa aktor pelaku dan/atau pronomina pada klausa bebas utama maupun klausa terikatnya. Kalimat ini tidak memunculkan pelengkap pada klausa \textit{X mengganti alat} dan/atau pada klausa \textit{supaya X tidak berebut ikan}.

\begin{figure}
	\centering \includegraphics[width=0.9
	\textwidth] {pics/ls1265.jpg} 
	\caption{Kalimat L\textit{akar}e pada data ragam lisan}
	\label{fig:ls1265} 
\end{figure}

Strategi kalimat seperti ini muncul dalam wacana-wacana di mana konstituen yang direduksi dari valensi verbal sudah terlebih dahulu direalisasikan di kalimat-kalimat sebelumnya atau sudah dimengerti bersama tanpa harus direalisasikan dalam kalimat tertentu. Batasan penelitian ini hanya pada tataran kalimat sehingga tidak menganalisis hubungan dependensi yang terbentuk antarkalimat. Namun, asumsi ini menimbulkan pertanyaan untuk analisis dependensi lanjutan pada tataran wacana yaitu bagaimana dan sejauh mana tautan dependensi yang terbentuk antarkalimat sehingga memungkinkan adanya pengurangan valensi verbal pada kalimat tertentu?

%%-----------------------------------------------------------------------------%
\subsection{Kecenderungan posisi induk terhadap konstituen terikat dalam kalimat}
%%-----------------------------------------------------------------------------%
Bagian analisis ini mencoba menjawab pertanyaan terkait arah tautan dependensi (\textit{head directionality}) dan pengurangan valensi akar verbal pada simpai pusat. Pengurangan konstituen dalam kalimat atau preferensi terhadap kalimat pendek terutama pada ragam lisan tidak selalu berarti menandakan adanya pengurangan valensi sebuah konstituen. Pada penelitian-penelitian terdahulu banyak dilakukan eksplorasi terhadap reduksi leksikal lain seperti konjungsi, partikel, ataupun kata-kata lain yang bersifat berulang atau \textit{repetitive} (\citealp{jaeger2006redundancy, gildea2015human}). Penelitian ini memfokuskan pengurangan valensi pada simpai pusat dengan akar verbal yang merupakan tipe akar yang paling banyak ditemukan pada kedua korpus data. 

Seperti yang dibahas pada bab sebelumnya, arah tautan dependensi dapat menyebabkan nilai dependensi tersebut menjadi positif atau negatif. Nilai positif yang menandakan bentuk relasi induk sebelum konstituen terikat (\textit{head-initial}). Berdasarkan aturan struktur frasa dalam tata bahasa yang ada (\citealp{kridalaksana2002struktur, sneddon2010indonesian}), terdapat indikasi bahwa bahasa Indonesia tergolong bahasa yang memilih bentuk \textit{head-initial} dibandingkan \textit{head-final}. Namun, belum ada penelitian dengan skala cukup besar yang dapat memberikan informasi ini dengan memanfaatkan data ujaran nyata (\textit{real utterance}) karena penerapannya dalam ujaran nyata dapat bersifat tidak gramatikal. Bentuk \textit{head-intial} tidak selalu menjadi preferensi pada semua bahasa karena beberapa bahasa memiliki aturan tata bahasa yang menuntut induk muncul setelah konstituen terikat atau \textit{head-final}. Meskipun belum ada konvensi dan bukti empiris yang menunjukkan bahwa bentuk ini lebih memudahkan proses memori kerja, \cite{futrell2015large} menunjukkan bukti empiris bahwa bahasa-bahasa yang cenderung menggunakan bentuk relasi ini memiliki tingkat pengurangan jarak dependensi (DLM) lebih tinggi dibandingkan bahasa-bahasa yang cenderung menggunakan bentuk relasi induk setelah konstituen terikat (\textit{head-final}). 

Didukung oleh hasil penelitian ini, bahasa Indonesia menunjukkan kecenderungan terhadap bentuk \textit{head-initial} pada simpai pusat maupun keseluruhan kalimat (antarkonstituen yang memiliki tautan langsung maupun antarkonstituen pada simpai pusat) yang ditandai oleh konsistensi nilai DL positif yang semakin tinggi terutama pada kalimat yang memiliki jumlah konstituen lebih banyak. Temuan ini menandakan konsistensi untuk menekan nilai DL negatif pada level struktur yang berbeda dan mengindikasikan preferensi head-initial di segala kondisi. Secara otomatis pergerakan nilai DL positif yang semakin tinggi mengakibatkan penekanan bentuk \textit{head-final} yang ditandai oleh nilai DL negatif. Bahkan, pada ragam lisan ditemukan indikasi adanya ambang batas terhadap frekuensi bentuk \textit{head-final}. Temuan ini memunculkan asumsi bahwa dalam bahasa Indonesia, kalimat dengan kecenderungan bentuk relasi induk sebelum konstituen terikat lebih banyak digunakan penutur dan menunjukkan indikasi bahwa bentuk ini lebih mudah dimengerti. Penelitian lebih lanjut dibutuhkan untuk menunjukkan bukti empiris untuk mendukung hipotesis ini dan mengukur seberapa bentuk relasi \textit{head-initial} memudahkan proses memori kerja. 

Analisis kualitatif untuk melihat lebih dalam bagaimana strategi yang mendukung preferensi \textit{head-initial} ini dilakukan dengan meninjau simpai-simpai pusat kalimat yang memiliki akar verbal pada kedua korpus. Secara umum, ditemukan dua strategi terkait arah dan valensi akar verbal yaitu, penempatan akar dan klausa bebas utama di awal kalimat dengan atau tanpa mengurangi valensi akar verbal. Penempatan akar dan klausa bebas utama di awal kalimat sangat umum ditemukan pada semua klasifikasi dalam korpus data ragam tulis. Namun, posisinya sedikit bergeser seiring dengan semakin banyaknya jumlah konstituen. Hal ini dikarenakan penempatan klausa bebas utama yang hampir selalu di posisi awal pada data ragam tulis. Sementara, pengurangan valensi akar verbal hampir tidak ditemukan pada data ragam tulis yang mungkin disebabkan karena karakter wacananya lebih formal dibandingkan dengan ragam lisan. Pengurangan valensi ini secara langsung berakibat pada pengurangan konstituen secara keseluruhan, sehingga dapat dikaitkan dengan temuan adanya preferensi untuk kalimat dengan jumlah konstituen yang lebih sedikit pada ragam lisan pada klasifikasi kalimat pendek dan menengah. Pengurangan valensi akar verbal ini sering disertai dengan penempatan posisi akar di awal kalimat ragam lisan namun hampir tidak ditemukan pada klasifikasi kalimat panjang. 

Temuan yang menarik adalah pada klasifikasi kalimat pendek dan menengah data ragam tulis maupun lisan, penempatan akar verbal yang aktor pelaku dalam valensinya sering berada pada posisi pertama baik dengan atau tanpa pengurangan valensi akar verbal (terutama aktor pelaku). Namun, mulai pada panjang kalimat tertentu, penempatan akar verbal yang mengikat aktor pelaku dalam valensinya tidak ditemukan pada posisi pertama dan aktor pelaku selalu direalisasikan sebelum akar verbal. Pengurangan valensi berupa aktor pelaku yang berupa kata penuh pada simpai dengan akar verbal tidak ditemukan apabila klausa utama didahului klausa lain, namun pengurangan valensi aktor pelaku yang berupa pronomina pengganti pada simpai dengan induk verbal ditemukan apabila klausa utama juga mengalami pengurangan valensi aktor pelaku. Untuk meninjau lebih dalam signifikansi temuan pengurangan valensi pada kondisi tersebut memerlukan pendekatan kuantitatif yang menuntut adanya sumber daya kamus atau leksikon valensi. Tinjauan lebih lanjut mengenai temuan-temuan ini dapat memberikan gambaran lebih dalam mengenai kognisi manusia terkait struktur kalimat yang dikonstruksi dan dapat memberikan masukan terhadap penggunaan tata bahasa yang memudahkan komunikasi serta proses memori kerja.

%%-----------------------------------------------------------------------------%
%\section{thesis.tex}
%%-----------------------------------------------------------------------------%
%Berkas ini berisi seluruh berkas Latex yang dibaca, jadi bisa dikatakan sebagai 
%berkas utama. Dari berkas ini kita dapat mengatur bab apa saja yang ingin 
%kita tampilkan dalam dokumen.
%
%
%%-----------------------------------------------------------------------------%
%\section{laporan\_setting.tex}
%%-----------------------------------------------------------------------------%
%Berkas ini berguna untuk mempermudah pembuatan beberapa template standar. 
%Anda diminta untuk menuliskan judul laporan, nama, npm, dan hal-hal lain yang 
%dibutuhkan untuk pembuatan template. 
%
%
%%-----------------------------------------------------------------------------%
%\section{istilah.tex}
%%-----------------------------------------------------------------------------%
%Berkas istilah digunakan untuk mencatat istilah-istilah yang digunakan. 
%Fungsinya hanya untuk memudahkan penulisan.
%Pada beberapa kasus, ada kata-kata yang harus selalu muncul dengan tercetak 
%miring atau tercetak tebal. 
%Dengan menjadikan kata-kata tersebut sebagai sebuah perintah \latex~tentu akan 
%mempercepat dan mempermudah pengerjaan laporan. 
%
%
%%-----------------------------------------------------------------------------%
%\section{hype.indonesia.tex}
%%-----------------------------------------------------------------------------%
%Berkas ini berisi cara pemenggalan beberapa kata dalam bahasa Indonesia. 
%\latex~memiliki algoritma untuk memenggal kata-kata sendiri, namun untuk 
%beberapa kasus algoritma ini memenggal dengan cara yang salah. 
%Untuk memperbaiki pemenggalan yang salah inilah cara pemenggalan yang benar 
%ditulis dalam berkas hype.indonesia.tex.
%
%
%%-----------------------------------------------------------------------------%
%\section{pustaka.tex}
%%-----------------------------------------------------------------------------%
%Berkas pustaka.tex berisi seluruh daftar referensi yang digunakan dalam 
%laporan. 
%Anda bisa membuat model daftar referensi lain dengan menggunakan bibtex.
%Untuk mempelajari bibtex lebih lanjut, silahkan buka 
%\url{http://www.bibtex.org/Format}. 
%Untuk merujuk pada salah satu referensi yang ada, gunakan perintah \bslash 
%cite, e.g. \bslash cite\{latex.intro\} yang akan akan memunculkan 
%\cite{latex.intro}
%
%
%%-----------------------------------------------------------------------------%
%\section{bab[1 - 6].tex}
%%-----------------------------------------------------------------------------%
%Berkas ini berisi isi laporan yang Anda tulis. 
%Setiap nama berkas e.g. bab1.tex merepresentasikan bab dimana tulisan tersebut 
%akan muncul. 
%Sebagai contoh, kode dimana tulisan ini dibaut berada dalam berkas dengan nama 
%bab4.tex. 
%Ada enam buah berkas yang telah disiapkan untuk mengakomodir enam bab dari 
%laporan Anda, diluar bab kesimpulan dan saran. 
%Jika Anda tidak membutuhkan sebanyak itu, silahkan hapus kode dalam berkas 
%thesis.tex yang memasukan berkas \latex~yang tidak dibutuhkan;  contohnya 
%perintah \bslash include\{bab6.tex\} merupakan kode untuk memasukan berkas 
%bab6.tex kedalam laporan.
%
