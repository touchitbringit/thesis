%
% Halaman Abstrak
%
% @author  Andreas Febrian
% @version 1.00
%

\chapter*{Abstrak}

\vspace*{0.2cm}

\noindent \begin{tabular}{l l p{10cm}}
	Nama&: & \penulis \\
	Program Studi&: & \program \\
	Judul&: & \judul \\
\end{tabular} \\ 

\vspace*{1cm}

Jarak dependensi antara dua konstituen dalam kalimat yang terhubung oleh tautan sintaktis dan panjang dependensi yang didapatkan dari seluruh jumlah jarak tersebut dalam sebuah kalimat adalah tema yang sedang marak dalam kajian sintaksis dan linguistik transdisipliner. Dalam bahasa Indonesia, penelitian terkait dilakukan sebatas pembuatan perangkat komputasional untuk pengolahan bahasa secara digital. Bahasa Indonesia memiliki urutan kata bebas di mana pertukaran posisi antara konstituen atau bahkan klausa mungkin tidak banyak berpengaruh terhadap makna sehingga memungkinkan penutur menghasilkan ujaran yang sangat bervariasi. Karakter ini menimbulkan asumsi adanya perbedaan yang signifikan antara bahasa Indonesia ragam tulis dan lisan, terutama mengenai kecenderungan ujaran spontan untuk memanfaatkan kalimat pendek dan pergeseran dari aturan tata bahasa terkait posisi linear dalam kalimat atau bahkan reduksi kata. Hal ini berkaitan dengan hipotesis bahwa bahasa manusia akan mendekatkan kata-kata yang memiliki relasi semantis sebagaimana dirangkum dalam hipotesis pengurangan panjang (\textit{Dependency Length Minimization} atau DLM) dan jarak dependensi (\textit{Dependency Distance Minimization} atau DDM). Penelitian ini menggunakan pendekatan kuantitatif dan kualitatif dalam melihat panjang dan jarak dependensi pada tataran kalimat dalam bahasa Indonesia. Dengan menggunakan korpus data ragam tulis dan lisan, luaran penelitian mengindikasikan perbedaan signifikan antara kedua ragam serta perbedaan tipe konstruksi kalimat yang mungkin menjadi keunikan masing-masing ragam. Terdapat juga beberapa bukti bagaimana tata bahasa Indonesia diterapkan dengan lebih baik pada konteks yang lebih formal untuk mendukung hipotesis tersebut. Hasil penelitian ini dapat dikembangkan untuk tujuan pemahaman keterkaitan kompleksitas sintaktis dengan tuntutan kognisi manusia.

\vspace*{0.5cm}

\noindent Kata Kunci: 
\newline
\noindent Dependensi, Sintaksis, Bahasa Indonesia, Kognisi Manusia, Efisiensi Bahasa

\newpage