%
% Halaman Abstract
%
% @author  Andreas Febrian
% @version 1.00
%

\chapter*{ABSTRACT}

\vspace*{0.2cm}

\noindent \begin{tabular}{l l p{11.0cm}}
	Name&: & \penulis \\
	Program&: & \programinggris \\
	Title&: & \judulInggris \\
\end{tabular} \\ 

\vspace*{1cm}

Dependency distance, the distance between constituents linked in a syntactic dependency relation, and dependency length of all the distances combined in a sentence, have been key measures of interest in the studies of syntax and transdisciplinary linguistics. Previous works utilizing real utterances in Indonesian are performed to produce methods and tools for digital Indonesian language processing. Indonesian is considered a free word order language, and that the exchange of constituents or even clauses might not have much effect on the meaning allowing speakers to create limitless variations of utterances. Due to this free word order nature, there are assumptions that written and spoken language in Indonesian might exhibit significant differences, particularly on how spontaneous speech might prefer shorter sentences and deviate from grammar rules in positioning or even syntactical reduction. This relates to the hypothesis that human language tend to organize word orders so semantically related words can be close in the linear order of a sentence which results in the minimization of dependency length (DLM) and dependency distance (DDM). This study uses quantitative methods of measuring dependency distance and length to investigate these hypothesis in Indonesian language. Using evidence in both written and spoken Indonesian language, the outcomes indicates differences between both speech modes and exhibitis manifestations of unique sentence structure type that might be exclusive to each speech mode. There are also proofs how Indonesian grammar rules might apply much better in a more formal nature like written language in support of DLM and DDM. The findings in this study pose further research questions to find out whether they are language-specific or cross-language in order to better understand the correlation between syntactic complexity and cognitive demands.

\vspace*{0.5cm}

\noindent Keywords: 
\newline
\noindent Dependency, Syntax, Indonesian Language, Cognition, Language Efficiency

\newpage