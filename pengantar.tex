%-----------------------------------------------------------------------------%
\chapter*{\kataPengantar}
%-----------------------------------------------------------------------------%

Sebagai satu-satunya mahasiswa pada angkatan ini yang datang dari luar ranah linguistik maupun sastra, saya bersyukur dapat mengenal lingkungan linguistik di Universitas Indonesia yang menyambut saya dengan tangan terbuka. Segenap jajaran dosen maupun teman-teman seangkatan dari Linguistik Murni dan Bahasa dan Budaya telah memperlihatkan pada saya bagaimana semangat yang dimiliki untuk mengembangkan ranah linguistik di Indonesia. Saya tidak pernah menganggap berkuliah di Indonesia lebih inferior dibandingkan luar negeri karena teman-teman yang saling mendukung satu sama lain telah menunjukkan semangat untuk dapat bertukar ide dan keberanian untuk mengeksplorasi ranah topik yang baru.

Saat masuk ke dalam program pascasarjana ini, saya sudah memiliki tujuan untuk mendedikasikan diri pada ranah linguistik transdisipliner, terutama kaitannya dengan ilmu kognitif. Meskipun saya mengambil jurusan Arsitektur sebelumnya dan telah berkarya di berbagai ranah lainnya, pada akhirnya seluruh perjalanan tersebut membantu saya untuk memperjelas kekaguman saya terhadap bahasa dan pikiran manusia. Saya sangat berterima kasih kepada pembimbing saya, Pak Totok Suhardijanto, M.Hum., Ph.D yang telah mendukung saya untuk menjelajahi linguistik kuantitatif sejak awal memasuki program ini dan menginspirasi saya untuk menggunakan korpus dalam investigasi bukti linguistik. Pak Totok telah sangat membantu mengoreksi tesis ini tanpa memandang sebelah mata, terutama karena saya sangat tidak mahir menyampaikan konsep ilmiah dalam bahasa Indonesia dan memberi masukan-masukan yang berarti sehingga tesis ini dapat memberikan perspektif utuh dan temuan yang bermakna. 

Penelitian-penelitian dari para peneliti internasional seperti Richard Futrell dan Haitao Liu adalah pemicu utama topik tesis yang saya ajukan: \textbf{Pengurangan panjang dan jarak dependensi pada tataran kalimat dalam bahasa Indonesia}. Investigasi mengenai dependensi dalam konteks linguistik transdisipliner merupakan hal yang baru di ranah bahasa Indonesia, dan saya sangat bersemangat untuk dapat membawa topik ini ke ranah linguistik di Indonesia. Topik ini merupakan persilangan antara beberapa disiplin seperti linguistik, ilmu kognitif, dan ilmu komputer sehingga merupakan tantangan yang sangat berat untuk dipelajari. Meskipun begitu, luaran yang dapat memberikan pemahaman terhadap keterkaitan bahasa dengan kognisi manusia dan mengapa manusia \textit{'say what they say'} selalu menjadi pegangan saya untuk tidak menyerah. Saya percaya ranah ini akan memberikan kesempatan yang baru dan banyak untuk para peneliti lain sehingga dapat memperkaya wawasan linguistik dalam bahasa Indonesia. Saya juga berterima kasih kepada Arno Bosse dan teman-teman Digital Humanities Summer School at Oxford University yang pertama kali menyemangati saya untuk percaya diri terjun ke dalam linguistik kuantitatif. 

Tidak kalah penting, saya juga ingin berterima kasih kepada Pak Dr. Untung Yuwono, Pak Dr. FX Rahyono, SS, M.Hum., Ibu Dr. Myrna Laksman, dan para dosen lainnya yang memberikan banyak masukan kritis terhadap tesis ini, teman-teman Linguistik Murni dan Bahasa dan Budaya angkatan 2016 atas perjuangan bersamanya, Pak Regi Wahyu, Pak Tom Malik dan Pak Susetio Nugroho dari Dattabot yang telah berbaik hati memberikan akses kepada korpus data ragam tulis, teman-teman jurnalis yang telah menyumbangkan rekaman audio yang menjadi korpus data ragam lisan, Fajar, Nabil, Zizah dan teman-teman IYKRA yang meminjamkan kantor agar saya dapat bekerja, Galih, David, Alderina, Anien, Flo, Tommy, Tika, Nona, Tya, dan teman-teman lain yang telah banyak mendengarkan saya berdiskusi, memberikan masukan kritis dan bahan hiburan di tengah tekanan penyelesaian tesis ini, Miss Anindya, Ratih, Stella, Meli, Diona, Lenny, Mutia, dan teman-teman \textit{ballerina} serta Messi yang selalu menjadi pelarian saya untuk melepaskan jenuh, serta teman-teman Master Transkrip yang sudah mengoreksi manual transkripsi data ragam lisan. Untuk keluarga Sindoro 16, untuk Mama, Papa, Mba Kanti, dan untuk Ibu tersayang. Endiyan Rakhmanda, yang berperan vital sebagai \textit{sparring partner} tidak hanya dari segi isi penelitian dan pendekatan komputasional, namun seluruh aspek penyempurnaan penelitian ini, \textit{I owe you my life}. Kepada nama-nama lain yang mungkin terlewat namun kontribusinya tidak kalah penting dalam penyelesaian tesis ini, terima kasih, dan semoga Tuhan membalas kebaikan kalian.

Hasil kerja keras dan buah pikiran ini seluruhnya adalah untuk Saka Guna, sumber kehidupan dan kewarasan saya. Semoga kelak dapat memberikannya inspirasi serta semangat untuk selalu haus akan pengetahuan dan hal-hal baru.

\vspace*{0.1cm}
\begin{flushright}
\today\\[0.1cm]
\vspace*{1cm}
\penulis

\end{flushright}