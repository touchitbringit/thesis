%-----------------------------------------------------------------------------%
\chapter{\babDua}

%-----------------------------------------------------------------------------%
\section{Konteks Penelitian}
%-----------------------------------------------------------------------------%
Kalimat merupakan sebuah susunan terorganisir yang memiliki konstituen berupa unit-unit kata dan tanda baca. Makna semantis sebuah konstituen di dalam ujaran atau kalimat tidak berdiri sendiri seperti kata-kata dan maknanya yang tertulis di dalam kamus. Melainkan, dalam penggunaan bahasa secara nyata (\textit{real utterance}), makna konstituen-konstituen tersebut baru akan terbentuk utuh setelah memiliki relasi dengan konstituen lain dalam sebuah frasa, klausa, kalimat ataupun antarkalimat. Pembentukan relasi dengan konstituen lain ini merupakan salah satu hasil penerapan fungsi konstituen atau kata \citep{tesniere1959elements}. Dalam ruang lingkup dependensi, fungsi yang membentuk relasi antarkonstituen dalam struktur ujaran tersebut terbagi dua: fungsi sintaksis (struktural) dan fungsi semantik (makna) \citep{tesniere1959elements}. Meskipun sintaksis dan semantik adalah dua bidang yang independen dan berbeda, keduanya masih berjalan sejajar dan saling berhubungan dalam teori dependensi \citep{tesniere1959elements}. Hal ini terjadi karena dependensi merupakan tautan struktural langsung yang menghubungkan unit-unit linguistik yang memiliki relasi semantis. Dalam perkembangan teorinya, konsep dependensi ini dapat dilacak jejaknya hingga pada akar tata bahasa Panini \citep{bharati1995natural}, tata bahasa Yunani dan Latin kuno (\citealp{covington1984syntactic, percival1990reflections}), serta bahasa Arab \citep{owens1988foundations}. 

Konsep dependensi yang ditemukan pada beberapa bahasa kuno di atas menunjukkan bahwa kata merupakan konstitituen atau unit utama sintaksis dan konstituen-konstituen dalam ujaran tersebut memiliki relasi struktural secara langsung. Terkait dengan peran kata, kesejajaran, dan hubungan antara sintaksis (struktur) dan semantik (makna), (\citealp{hudson1984word, hudson2007language}) mengembangkan teori \textit{Word Grammar} atau Tata Bahasa Kata. \textit{Word Grammar} mengadopsi konsep dependensi sebagai dasar untuk menelaah struktur kalimat dan melihat bahasa sebagai sebuah jejaring \citep{hudson2007language}. Kata sebagai unit utama sintaksis dan penerapannya dalam teori dependensi juga banyak menjadi bahan diskusi dalam pengembangan metode komputasional. Salah satu teori utama terkait linguistik komputasional yang bertitik tolak dari teori dependensi tersebut adalah \textit{Meaning-Text Theory} atau Teori Makna-Teks (MTT) yang dikembangkan oleh \cite{mel?vcuk1988dependency} dan penguraian kalimat multilingual berdasarkan teori dependensi yang dibangun dari pengumpulan data diagram pohon dependensi dari berbagai bahasa, proyek ini disebut dengan \textit{Universal Dependencies} (\citealp{mcdonald2013universal, nivre2016universal, nivre2017universal}). 

Adanya bukti-bukti dependensi yang ditemukan secara lintas bahasa merupakan indikasi utama bahwa fenomena ini dialami beberapa atau bahkan semua bahasa di dunia (universal). Asumsi bahwa dependensi bersifat universal menandakan keterkaitan konsep ini dengan kerja kognisi manusia \citep{gibson2000dependency}. Salah satu isu utama dalam memahami bagaimana pengetahuan bahasa diimplementasikan di dalam kognisi manusia melibatkan kajian terhadap produksi dan pemahaman bahasa menggunakan data \textit{real utterance}. Pengetahuan di dalam kognisi manusia terkait aturan yang membentuk sebuah bahasa ini disebut dengan kemampuan (\textit{competence}). Sementara itu, aktivitas kebahasaan yang memanfaatkan pengetahuan tersebut secara nyata disebut dengan penampilan (\textit{performance}) \citep{delahuntygarvey2010soundsense}. Beberapa studi berdasarkan bukti-bukti penampilan menunjukkan bahwa dalam mengkonstruksikan sebuah struktur ujaran, penutur melibatkan proses memori kerja secara bertahap (\textit{moment-by-moment}) \citep{gibson2000dependency}. Hubungan memori kerja secara bertahap dengan konstruksi ujaran menggambarkan bagaimana produksi dan pemahaman ujaran dipengaruhi pertimbangan untuk memudahkan proses memori kerja \citep{futrell2015large} Berkaitan dengan asumsi tersebut, muncul hipotesis adanya kecenderungan bahwa penutur akan mendekatkan konstituen-konstituen dalam ujaran kalimat yang memiliki relasi secara semantik (digambarkan melalui tautan dependensi yang menghubungkan kedua konstituen tersebut) (\citealp{futrell2015large, liu2017dependency}). Konstituen-konstituen yang mendekat ini juga dihubungkan dengan hipotesis adanya kesulitan dalam produksi dan pemahaman penutur terhadap ujaran dengan jarak dependensi yang jauh dan struktur yang rumit (\citealp{gibson2000dependency, dillon2011structured}).

Hubungan erat antara struktur ujaran dengan proses dan penyimpanan memori dalam memori kerja menimbulkan pertanyaan mengenai pola dependensi pada ujaran dengan derajat spontanitas yang tinggi seperti pada ragam lisan \citep{abney1991memory}. Pertanyaan ini menjadi dasar  pemilihan kedua korpus data ragam tulis dan lisan untuk analisis dependensi terkait derajat spontanitas ujaran dalam penelitian ini. Selain itu, terdapat beberapa pembahasan yang menyoroti pengaruh faktor karakter bahasa dan ketatabahasaan terhadap pola dependensi (\citealp{hawkins2014cross, jiang2015effects, wang2017effects}). Meskipun tidak membahas faktor karakter bahasa dan ketatabahasaan, dalam penelitian lintas bahasa yang berskala besar, \cite{futrell2015large} menyebutkan bahwa temuan dalam bahasa Indonesia menunjukkan tingkat optimasi ujaran yang paling tinggi dibandingkan 36 bahasa lain dalam penelitiannya. Optimasi yang dimaksud adalah sejauh mana konstituen-konstituen dalam sebuah ujaran didekatkan berdasarkan data bahasa Indonesia. Sebagai contoh, perbandingan temuan optimasi terkait dengan urutan linear kata (\textit{word order}) antara bahasa Indonesia dengan Jerman dalam penelitian \cite{futrell2015large} sangat kontras. Tidak hanya terkait karakter dan ketatabahasaan dalam sebuah bahasa, temuan mengenai analisis dependensi juga mungkin berbeda diakibatkan perbedaan jenis teks yang dianalisis. Berkaitan dengan ini, \cite{wang2017effects} melakukan penelitian lintas bahasa yang menunjukkan adanya perbedaan jarak dependensi terutama antara teks yang bersifat informatif dengan teks imajinatif.

\todo{Pembahasan struktural sintaks berhubungan langsung dengan fungsi struktural (sintaksis), namun dependensi juga sedikit bersinggungan dengan fungsi semantik. Fungsi struktural merupakan ranah di mana ekspresi linguistik dikembangkan dan berkenaan dengan tata bahasa (intrinsik). Sementara itu, fungsi semantik bersifat ekstrinsik dari pembentukan ekspresi kebahasaan karena berkenaan dengan makna dan logika \citep{tesniere1959elements}. Dalam teori dependensi, kedua fungsi tersebut berjalan sejajar karena fungsi struktural bertujuan untuk mengekspresikan apa yang telah dibentuk secara semantik. Kesejajaran fungsi ini juga muncul dalam hubungan antarkata karena hubungan semantik terintegrasikan ke dalam hubungan struktural. \cite{tesniere1959elements} menyebut kesejajaran ini dengan istilah ?struktur mengekspresikan makna? atau \textit{the structural expresses the semantic}. Hubungan ini dapat dinyatakan juga sebagai berikut: makna dari konstituen terikat membawa bersamanya kualitas konstituen induk tempatnya bergantung. Oleh karena itu, hierarki signifikansi kata secara struktural berbanding terbalik dengan semantik.} 


%-----------------------------------------------------------------------------%
\subsection{Kemampuan (\textit{Competence}) dan Penampilan (\textit{Performance})}
%-----------------------------------------------------------------------------%
Intelektualitas untuk berkomunikasi dan berbahasa adalah kemampuan utama yang membedakan antara manusia dengan mahluk yang lain. Meskipun setiap daerah di seluruh dunia memiliki keunikan dalam berkomunikasi dan berbahasa, terdapat asumsi bahwa dibalik perbedaan itu, ada  kesamaan kualitas bahasa yang disebabkan oleh cara kerja kognisi manusia yang serupa \citep{sapir1921intro, chomsky1965syntactic}. \cite{beattie1788theory} merupakan salah satu orang pertama yang menjelaskan keunikan setiap bahasa terkait leksikon dan tata bahasanya. \cite{chomsky1965syntactic} menyanggah prinsip tata bahasa tradisional yang mengatakan bahwa 'susunan alami pikiran' atau \textit{natural order of thoughts} sudah pasti tercermin pada urutan kata dalam sebuah ujaran yang dibahas dalam konvensi tata bahasa. Sebagai contoh, \cite{diderot1751lettre} beranggapan bahwa Perancis merupakan salah satu bahasa yang urutan kata dalam ujaran berkorespondensi terhadap susunan alami pikiran dan ide. Hal ini berarti bagaimanapun susunan yang terbentuk dalam bahasa kuno maupun modern, pikiran dan ide penutur dapat dianalisis secara semantik dengan mengikuti norma-norma sintaksisnya. Namun, salah satu aspek dari teori-teori linguistik tradisional yang masih relevan dalam teori-teori linguistik modern adalah bahwa terdapat setidaknya satu karakter yang mengaitkan seluruh bahasa di dunia, yaitu aspek kreatif dan variasinya \citep{hawkins2014cross}.

Bahasa memiliki sarana terbatas (\texit{finite means}) dan keterbatasan (\textit{constraints}) dalam berbagai aspek \citep{von1972origin}. \cite{von1972origin} menyampaikan ini dalam kalimat \textit{make use of make infinite use of finite means} untuk menggambarkan kemampuan sebuah bahasa. Studi mengenai keterbatasan dan bagaimana bahasa mengakali keterbatasan tersebut telah berkembang selama puluhan tahun terakhir, termasuk di dalamnya adalah kajian berbasis data penampilan bahasa dengan mengintegrasikan ilmu linguistik dan prinsip-prinsip dalam matematika \citep{chomsky1965syntactic}. Semenjak adanya rintisan hasil kerja dari \cite{chomsky1957syntactic}, dunia linguistik mulai mempertimbangkan tata bahasa sebagai perangkat formal berupa deskripsi eksplisit mengenai batasan-batasan produktif sebuah bahasa. Hal ini berarti tata bahasa membatasi berbagai kemungkinan formula yang terbentuk dari relasi antara satu konstituen (seperti kata atau morfem) dengan konstituen lain dalam kategori sintagmatik. Salah satu tujuan utama tata bahasa, menurut \cite{chomsky1965syntactic}, adalah untuk membedakan urutan konstituen gramatikal dengan yang tidak gramatikal\footnote{Pada pembahasan ini, bahasa yang dimaksud Chomsky dipandang sebagai bentuk urutan kata tidak terbatas yang masing-masing memiliki asosiasi relevan dengan deskripsi struktural tata bahasanya.}. Dari sini, \cite{chomsky1965syntactic} mengekspresikan tata bahasa sebagai kemampuan linguistik\footnote{Dalam Kamus Linguistik \citep{kridalaksana2008kamus}, kemampuan linguistik dijabarkan sebagai "kemampuan bahasawan untuk memahami dan menghasilkan kalirnat-kalimat yang belum pernah didengar sebelumnya, yakni kode yang mendasari semua ujaran dalam satu bahasa". Sementara, penampilan dijabarkan sebagai "realisasi kode itu dalam pemakaian bahasa yang sebenarnya, yakni ujaran itu sendiri".} (\textit{linguistic competence}) yang berarti pengetahuan penutur/pendengar terhadap bahasanya. Dalam kemampuan linguistik, istilah 'gramatikal' menjadi salah satu tolok ukur.   \cite{chomsky1965syntactic} menegaskan perbedaan antara istilah tersebut dengan 'dapat diterima' atau \textit{acceptable}.  Berbeda dengan konsep 'gramatikal', keterterimaan (\textit{acceptability}) merupakan konsep dan tolok ukur dalam studi penampilan linguistik (\textit{linguistic performance}) \citep{chomsky1965syntactic}. Studi terhadap penampilan linguistik merupakan studi terhadap penggunaan bahasa oleh penutur dalam situasi yang nyata dengan menggunakan \textit{real utterance}. Dualisme\footnote{Dalam pandangan teknis, Chomsky melihat teori linguistik sebagai hal yang bersifat mentalistik karena berkenaan dengan penjajakan realitas mental yang mendasari perilaku nyata. Chomsky menekankan hal tersebut untuk membedakan dualisme \textit{competence-performance} dengan \textit{langue-parole} yang dikemukakan Saussure \citep{key2017course}. Bagi Chomsky, \textit{langue} hanya mencakup inventori sistematis dari unit-unit linguistik. Dalam konteks realitas mental, kemampuan lebih dekat dengan konsepsi pemikiran dalam proses generatif yang digagas oleh \cite{von1972origin}.} yang digagas  \cite{chomsky1965syntactic} ini berangkat dari hasil analisis yang menunjukkan bahwa kemampuan tidak selalu dapat tercermin secara langsung pada penampilan.

%-----------------------------------------------------------------------------%
\subsection{Efisiensi dan Urutan Kata (\textit{Word Order}) dalam Ranah Ilmu Sintaksis}
%-----------------------------------------------------------------------------%
Hal pertama yang mungkin ditanyakan adalah bagaimana membuat huruf tercetak 
tebal, miring, atau memiliki garis bawah. 
Pada Texmaker, Anda bisa melakukan hal ini seperti halnya saat mengubah dokumen 
dengan OO Writer. 
Namun jika tetap masih tertarik dengan cara lain, ini dia: 

\begin{itemize}
	\item \bo{Bold} \\
		Gunakan perintah \bslash textbf$\lbrace\rbrace$ atau 
		\bslash bo$\lbrace\rbrace$. 
	\item \f{Italic} \\
		Gunakan perintah \bslash textit$\lbrace\rbrace$ atau 
		\bslash f$\lbrace\rbrace$. 
	\item \underline{Underline} \\
		Gunakan perintah \bslash underline$\lbrace\rbrace$.
	\item $\overline{Overline}$ \\
		Gunakan perintah \bslash overline. 
	\item $^{superscript}$ \\
		Gunakan perintah \bslash $\lbrace\rbrace$. 
	\item $_{subscript}$ \\
		Gunakan perintah \bslash \_$\lbrace\rbrace$. 
\end{itemize}

Perintah \bslash f dan \bslash bo hanya dapat digunakan jika package 
uithesis digunakan. 


%-----------------------------------------------------------------------------%
\subsection{Relasi dan Struktur Ujaran}
%-----------------------------------------------------------------------------%
Setiap gambar dapat diberikan caption dan diberikan label. Label dapat 
digunakan untuk menunjuk gambar tertentu. 
Jika posisi gambar berubah, maka nomor gambar juga akan diubah secara 
otomatis. 
Begitu juga dengan seluruh referensi yang menunjuk pada gambar tersebut. 
Contoh sederhana adalah \pic~\ref{fig:testGambar}. 
Silahkan lihat code \latex~dengan nama bab2.tex untuk melihat kode lengkapnya. 
Harap diingat bahwa caption untuk gambar selalu terletak dibawah gambar. 

\begin{figure}
	\centering
	\includegraphics[width=0.50\textwidth]
		{pics/creative_common.png}
	\caption{\license.}
	\label{fig:testGambar}
\end{figure}


%-----------------------------------------------------------------------------%
\subsection{Penelitian Terdahulu dan \textit{Positioning}}
%-----------------------------------------------------------------------------%
Seperti pada gambar, tabel juga dapat diberi label dan caption. 
Caption pada tabel terletak pada bagian atas tabel. 
Contoh tabel sederhana dapat dilihat pada \tab~\ref{tab:tab1}.

\begin{table}
	\centering
	\caption{Contoh Tabel}
	\label{tab:tab1}
	\begin{tabular}{| l | c r |}
		\hline
		& kol 1 & kol 2 \\ 
		\hline
		baris 1 & 1 & 2 \\
		baris 2 & 3 & 4 \\
		baris 3 & 5 & 6 \\
		jumlah  & 9 & 12 \\
		\hline
	\end{tabular}
\end{table}

Ada jenis tabel lain yang dapat dibuat dengan \latex~berikut 
beberapa diantaranya. 
Contoh-contoh ini bersumber dari 
\url{http://en.wikibooks.org/wiki/LaTeX/Tables}

\begin{table}
	\centering
	\caption{An Example of Rows Spanning Multiple Columns}
	\label{row.spanning}
	\begin{tabular}{|l|l|*{6}{c|}}
  		\hline % create horizontal line
  		No & Name & \multicolumn{3}{|c|}{Week 1} & \multicolumn{3}{|c|}{Week 2} \\
  		\cline{3-8} % create line from 3rd column till 8th column
  		& & A & B & C & A & B & C\\
  		\hline
  		1 & Lala & 1 & 2 & 3 & 4 & 5 & 6\\
  		2 & Lili & 1 & 2 & 3 & 4 & 5 & 6\\
  		3 & Lulu & 1 & 2 & 3 & 4 & 5 & 6\\
  		\hline
	\end{tabular}
\end{table}

\begin{table}
	\centering
	\caption{An Example of Columns Spanning Multiple Rows}
	\label{column.spanning}
	\begin{tabular}{|l|c|l|}
		\hline
		Percobaan & Iterasi & Waktu \\
		\hline
		Pertama & 1 & 0.1 sec \\ \hline
		\multirow{2}{*}{Kedua} & 1 & 0.1 sec \\
 		& 3 & 0.15 sec \\ 
 		\hline
		\multirow{3}{*}{Ketiga} & 1 & 0.09 sec \\
 		& 2 & 0.16 sec \\
 		& 3 & 0.21 sec \\ 
 		\hline
	\end{tabular}
\end{table}

\begin{table}
	\centering
	\caption{An Example of Spanning in Both Directions Simultaneously}
	\label{mix.spanning}
	\begin{tabular}{cc|c|c|c|c|}
		\cline{3-6}
		& & \multicolumn{4}{|c|}{Title} \\ \cline{3-6}
		& & A & B & C & D \\ \hline
		\multicolumn{1}{|c|}{\multirow{2}{*}{Type}} &
		\multicolumn{1}{|c|}{X} & 1 & 2 & 3 & 4\\ \cline{2-6}
		\multicolumn{1}{|c|}{}                        &
		\multicolumn{1}{|c|}{Y} & 0.5 & 1.0 & 1.5 & 2.0\\ \cline{1-6}
		\multicolumn{1}{|c|}{\multirow{2}{*}{Resource}} &
		\multicolumn{1}{|c|}{I} & 10 & 20 & 30 & 40\\ \cline{2-6}
		\multicolumn{1}{|c|}{}                        &
		\multicolumn{1}{|c|}{J} & 5 & 10 & 15 & 20\\ \cline{1-6}
	\end{tabular}
\end{table}


%-----------------------------------------------------------------------------%
\section{Konsep Dependensi}
%-----------------------------------------------------------------------------%
Setiap gambar dapat diberikan caption dan diberikan label. Label dapat 
digunakan untuk menunjuk gambar tertentu. 
Jika posisi gambar berubah, maka nomor gambar juga akan diubah secara 
otomatis. 


%-----------------------------------------------------------------------------%
\subsection{Konstituen Induk (\textit{Root/Head}) dan Simpai (\textit{Node})}
%-----------------------------------------------------------------------------%
Setiap gambar dapat diberikan caption dan diberikan label. Label dapat 
digunakan untuk menunjuk gambar tertentu. 
Jika posisi gambar berubah, maka nomor gambar juga akan diubah secara 
otomatis. 


%-----------------------------------------------------------------------------%
\subsection{Panjang Dependensi (\textit{Dependency Length}) dan Jarak Dependensi (\textit{Dependency Distance})}
%-----------------------------------------------------------------------------%
Setiap gambar dapat diberikan caption dan diberikan label. Label dapat 
digunakan untuk menunjuk gambar tertentu. 
Jika posisi gambar berubah, maka nomor gambar juga akan diubah secara 
otomatis. 


%-----------------------------------------------------------------------------%
\subsection{Faktor-faktor yang Mempengaruhi Tautan Dependensi}
%-----------------------------------------------------------------------------%
Setiap gambar dapat diberikan caption dan diberikan label. Label dapat 
digunakan untuk menunjuk gambar tertentu. 
Jika posisi gambar berubah, maka nomor gambar juga akan diubah secara 
otomatis. 

%-----------------------------------------------------------------------------%
\subsubsection{Kognisi Manusia dan Produksi Ujaran}
%-----------------------------------------------------------------------------%
Setiap gambar dapat diberikan caption dan diberikan label. Label dapat 
digunakan untuk menunjuk gambar tertentu. 
Jika posisi gambar berubah, maka nomor gambar juga akan diubah secara 
otomatis. 

%-----------------------------------------------------------------------------%
\subsubsection{Karakter Bahasa dan Ketatabahasaan}
%-----------------------------------------------------------------------------%
Setiap gambar dapat diberikan caption dan diberikan label. Label dapat 
digunakan untuk menunjuk gambar tertentu. 
Jika posisi gambar berubah, maka nomor gambar juga akan diubah secara 
otomatis. 

%-----------------------------------------------------------------------------%
\section{Efisiensi Ujaran dari Segi Dependensi}
%-----------------------------------------------------------------------------%
Setiap gambar dapat diberikan caption dan diberikan label. Label dapat 
digunakan untuk menunjuk gambar tertentu. 
Jika posisi gambar berubah, maka nomor gambar juga akan diubah secara 
otomatis. 

%-----------------------------------------------------------------------------%
\subsection{Pengurangan Panjang dan Jarak Dependensi Terkait Memori Kerja}
%-----------------------------------------------------------------------------%
Setiap gambar dapat diberikan caption dan diberikan label. Label dapat 
digunakan untuk menunjuk gambar tertentu. 
Jika posisi gambar berubah, maka nomor gambar juga akan diubah secara 
otomatis. 

%-----------------------------------------------------------------------------%
\subsection{Perubahan Valensi Konstituen Induk (\textit{Root}) Verbal}
%-----------------------------------------------------------------------------%
Setiap gambar dapat diberikan caption dan diberikan label. Label dapat 
digunakan untuk menunjuk gambar tertentu. 
Jika posisi gambar berubah, maka nomor gambar juga akan diubah secara 
otomatis. 

%-----------------------------------------------------------------------------%
\section{Dependensi dan Urutan Kata dalam Bahasa Indonesia}
%-----------------------------------------------------------------------------%
Setiap gambar dapat diberikan caption dan diberikan label. Label dapat 
digunakan untuk menunjuk gambar tertentu. 
Jika posisi gambar berubah, maka nomor gambar juga akan diubah secara 
otomatis. 

%-----------------------------------------------------------------------------%
\subsection{Urutan Kata dalam Tata Bahasa Indonesia}
%-----------------------------------------------------------------------------%
Setiap gambar dapat diberikan caption dan diberikan label. Label dapat 
digunakan untuk menunjuk gambar tertentu. 
Jika posisi gambar berubah, maka nomor gambar juga akan diubah secara 
otomatis. 

%-----------------------------------------------------------------------------%
\subsection{Simpai Sentral (\textit{Central Node}) dalam Bahasa Indonesia}
%-----------------------------------------------------------------------------%
Setiap gambar dapat diberikan caption dan diberikan label. Label dapat 
digunakan untuk menunjuk gambar tertentu. 
Jika posisi gambar berubah, maka nomor gambar juga akan diubah secara 
otomatis. 
