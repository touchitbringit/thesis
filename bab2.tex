%-----------------------------------------------------------------------------%
\chapter{\babDua}

%-----------------------------------------------------------------------------%
\section{Konteks Penelitian}
%-----------------------------------------------------------------------------%
Kalimat merupakan sebuah susunan terorganisir yang memiliki konstituen berupa unit-unit kata dan tanda baca. Makna semantis sebuah konstituen di dalam ujaran atau kalimat tidak berdiri sendiri seperti kata-kata dan maknanya yang tertulis di dalam kamus. Melainkan, dalam penggunaan bahasa secara nyata (\textit{real utterance}), makna konstituen-konstituen tersebut baru akan terbentuk utuh setelah memiliki relasi dengan konstituen lain dalam sebuah frasa, klausa, kalimat ataupun antarkalimat. Pembentukan relasi dengan konstituen lain ini merupakan salah satu hasil penerapan fungsi konstituen atau kata \citep{tesniere1959elements}. Dalam ruang lingkup dependensi, fungsi yang membentuk relasi antarkonstituen dalam struktur ujaran tersebut terbagi dua: fungsi sintaksis (struktural) dan fungsi semantik (makna) \citep{tesniere1959elements}. Meskipun sintaksis dan semantik adalah dua bidang yang independen dan berbeda, keduanya masih berjalan sejajar dan saling berhubungan dalam teori dependensi \citep{tesniere1959elements}. Hal ini terjadi karena dependensi merupakan tautan struktural langsung yang menghubungkan unit-unit linguistik yang memiliki relasi semantis. Dalam perkembangan teorinya, konsep dependensi ini dapat dilacak jejaknya hingga pada akar tata bahasa Panini \citep{bharati1995natural}, tata bahasa Yunani dan Latin kuno (\citealp{covington1984syntactic, percival1990reflections}), serta bahasa Arab \citep{owens1988foundations}. 

Konsep dependensi yang ditemukan pada beberapa bahasa kuno di atas menunjukkan bahwa kata merupakan konstitituen atau unit utama sintaksis dan konstituen-konstituen dalam ujaran tersebut memiliki relasi struktural secara langsung. Terkait dengan peran kata, kesejajaran, dan hubungan antara sintaksis (struktur) dan semantik (makna), (\citealp{hudson1984word, hudson2007language}) mengembangkan teori \textit{Word Grammar} atau Tata Bahasa Kata. \textit{Word Grammar} mengadopsi konsep dependensi sebagai dasar untuk menelaah struktur kalimat dan melihat bahasa sebagai sebuah jejaring \citep{hudson2007language}. Kata sebagai unit utama sintaksis dan penerapannya dalam teori dependensi juga banyak menjadi bahan diskusi dalam pengembangan metode komputasional. Salah satu teori utama terkait linguistik komputasional yang bertitik tolak dari teori dependensi tersebut adalah \textit{Meaning-Text Theory} atau Teori Makna-Teks (MTT) yang dikembangkan oleh \cite{mel'vcuk1988dependency} dan penguraian kalimat multilingual berdasarkan teori dependensi yang dibangun dari pengumpulan data diagram pohon dependensi dari berbagai bahasa, proyek ini disebut dengan \textit{Universal Dependencies} (\citealp{mcdonald2013universal, nivre2016universal, nivre2017universal}). 

Adanya bukti-bukti dependensi yang ditemukan secara lintas bahasa merupakan indikasi utama bahwa fenomena ini dialami beberapa atau bahkan semua bahasa di dunia (universal). Asumsi bahwa dependensi bersifat universal menandakan keterkaitan konsep ini dengan kerja kognisi manusia \citep{gibson2000dependency}. Salah satu isu utama dalam memahami bagaimana pengetahuan bahasa diimplementasikan di dalam kognisi manusia melibatkan kajian terhadap produksi dan pemahaman bahasa menggunakan data \textit{real utterance}. Pengetahuan di dalam kognisi manusia terkait aturan yang membentuk sebuah bahasa ini disebut dengan kemampuan (\textit{competence}). Sementara itu, aktivitas kebahasaan yang memanfaatkan pengetahuan tersebut secara nyata disebut dengan penampilan (\textit{performance}) \citep{delahuntygarvey2010soundsense}. Beberapa studi berdasarkan bukti-bukti penampilan menunjukkan bahwa dalam mengkonstruksikan sebuah struktur ujaran, penutur melibatkan proses memori kerja secara bertahap (\textit{moment-by-moment}) \citep{gibson2000dependency}. Hubungan memori kerja secara bertahap dengan konstruksi ujaran menggambarkan bagaimana produksi dan pemahaman ujaran dipengaruhi pertimbangan untuk memudahkan proses memori kerja \citep{futrell2015large} Berkaitan dengan asumsi tersebut, muncul hipotesis adanya kecenderungan bahwa penutur akan mendekatkan konstituen-konstituen dalam ujaran kalimat yang memiliki relasi secara semantik (digambarkan melalui tautan dependensi yang menghubungkan kedua konstituen tersebut) (\citealp{futrell2015large, liu2017dependency}). Konstituen-konstituen yang mendekat ini juga dihubungkan dengan hipotesis adanya kesulitan dalam produksi dan pemahaman penutur terhadap ujaran dengan jarak dependensi yang jauh dan struktur yang rumit (\citealp{gibson2000dependency, dillon2011structured}).

Hubungan erat antara struktur ujaran dengan proses dan penyimpanan memori dalam memori kerja menimbulkan pertanyaan mengenai pola dependensi pada ujaran dengan derajat spontanitas yang tinggi seperti pada ragam lisan \citep{abney1991memory}. Pertanyaan ini menjadi dasar  pemilihan kedua korpus data ragam tulis dan lisan untuk analisis dependensi terkait derajat spontanitas ujaran dalam penelitian ini. Selain itu, terdapat beberapa pembahasan yang menyoroti pengaruh faktor karakter bahasa dan ketatabahasaan terhadap pola dependensi (\citealp{hawkins2014cross, jiang2015effects, wang2017effects}). Meskipun tidak membahas faktor karakter bahasa dan ketatabahasaan, dalam penelitian lintas bahasa yang berskala besar, \cite{futrell2015large} menyebutkan bahwa temuan dalam bahasa Indonesia menunjukkan tingkat optimasi ujaran yang paling tinggi dibandingkan 36 bahasa lain dalam penelitiannya. Optimasi yang dimaksud adalah sejauh mana konstituen-konstituen dalam sebuah ujaran didekatkan berdasarkan data bahasa Indonesia. Sebagai contoh, perbandingan temuan optimasi terkait dengan urutan linear kata (\textit{word order}) antara bahasa Indonesia dengan Jerman dalam penelitian \cite{futrell2015large} sangat kontras. Tidak hanya terkait karakter dan ketatabahasaan dalam sebuah bahasa, temuan mengenai analisis dependensi juga mungkin berbeda diakibatkan perbedaan jenis teks yang dianalisis. Berkaitan dengan ini, \cite{wang2017effects} melakukan penelitian lintas bahasa yang menunjukkan adanya perbedaan jarak dependensi terutama antara teks yang bersifat informatif dengan teks imajinatif.

Struktur sebuah ujaran memiliki relasi sintaktis yang diterapkan berkenaan dengan tata bahasanya, namun struktur tersebut juga mengandung relasi dari segi semantik. \cite{tesniere1959elements} menyebutkan tata bahasa yang menjadi ancangan relasi sintaktis ini sebagai sesuatu yang bersifat intrinsik karena merupakan sistematika yang mendasari sebuah ujaran. Relasi sintaktis ini memungkinkan makna sebuah ujaran diproduksi secara utuh dan dipahami oleh penerima informasi. Oleh karena itu, relasi semantis yang tersampaikan kepada dan dipahami pendengar atau pembaca disebut sebagai hal yang ekstrinsik \citep{tesniere1959elements}. Keterkaitan fungsi sintaktis dan semantis ini juga muncul dalam hubungan antarkonstituen karena relasi semantis terintegrasikan ke dalam relasi sintaktis. \cite{tesniere1959elements} menyebut kesejajaran kedua fungsi ini dengan istilah 'struktur mengekspresikan makna' atau \textit{the structural expresses the semantic}. Hubungan ini dapat dinyatakan juga sebagai berikut: makna konstituen terikat membawa bersamanya kualitas induk tempatnya bergantung. Oleh karena itu, hierarki signifikansi konstituen secara struktural berbanding terbalik dengan semantik.

%-----------------------------------------------------------------------------%
\subsection{Dualisme Kemampuan (\textit{Competence}) dan Penampilan (\textit{Performance})}
%-----------------------------------------------------------------------------%
Intelektualitas untuk berkomunikasi dan berbahasa adalah kemampuan utama yang membedakan antara manusia dengan mahluk yang lain. Meskipun setiap daerah di seluruh dunia memiliki keunikan dalam berkomunikasi dan berbahasa, terdapat asumsi bahwa dibalik perbedaan itu, ada  kesamaan kualitas bahasa yang disebabkan oleh cara kerja kognisi manusia yang serupa \citep{sapir1921intro, chomsky1965syntactic}. \cite{beattie1788theory} merupakan salah satu orang pertama yang menjelaskan keunikan setiap bahasa terkait leksikon dan tata bahasanya. \cite{chomsky1965syntactic} menyanggah prinsip tata bahasa tradisional yang mengatakan bahwa 'susunan alami pikiran' atau \textit{natural order of thoughts} sudah pasti tercermin pada urutan kata dalam sebuah ujaran yang dibahas dalam konvensi tata bahasa. Sebagai contoh, \cite{diderot1751lettre} beranggapan bahwa Perancis merupakan salah satu bahasa yang urutan kata dalam ujaran berkorespondensi terhadap susunan alami pikiran dan ide. Hal ini berarti bagaimanapun susunan yang terbentuk dalam bahasa kuno maupun modern, pikiran dan ide penutur dapat dianalisis secara semantik dengan mengikuti norma-norma sintaksisnya. Namun, salah satu aspek dari teori-teori linguistik tradisional yang masih relevan dalam teori-teori linguistik modern adalah bahwa terdapat setidaknya satu karakter yang mengaitkan seluruh bahasa di dunia, yaitu aspek kreatif dan variasinya \citep{hawkins2014cross}.

Bahasa memiliki sarana terbatas (\textit{finite means}) dan keterbatasan (\textit{constraints}) dalam berbagai aspek \citep{von1972origin}. \cite{von1972origin} menyampaikan ini dalam kalimat \textit{make use of make infinite use of finite means} untuk menggambarkan kemampuan sebuah bahasa. Studi mengenai keterbatasan dan bagaimana bahasa mengakali keterbatasan tersebut telah berkembang selama puluhan tahun terakhir, termasuk di dalamnya adalah kajian berbasis data penampilan bahasa dengan mengintegrasikan ilmu linguistik dan prinsip-prinsip dalam matematika \citep{chomsky1965syntactic}. Semenjak adanya rintisan hasil kerja dari \cite{chomsky1957syntactic}, dunia linguistik mulai mempertimbangkan tata bahasa sebagai perangkat formal berupa deskripsi eksplisit mengenai batasan-batasan produktif sebuah bahasa. Hal ini berarti tata bahasa membatasi berbagai kemungkinan formula yang terbentuk dari relasi antara satu konstituen (seperti kata atau morfem) dengan konstituen lain dalam kategori sintagmatik. Salah satu tujuan utama tata bahasa, menurut \cite{chomsky1965syntactic}, adalah untuk membedakan urutan konstituen gramatikal dengan yang tidak gramatikal\footnote{Pada pembahasan ini, bahasa yang dimaksud Chomsky dipandang sebagai bentuk urutan kata tidak terbatas yang masing-masing memiliki asosiasi relevan dengan deskripsi struktural tata bahasanya.}. Dari sini, \cite{chomsky1965syntactic} mengekspresikan tata bahasa sebagai kemampuan linguistik\footnote{Dalam Kamus Linguistik \citep{kridalaksana2008kamus}, kemampuan linguistik dijabarkan sebagai "kemampuan bahasawan untuk memahami dan menghasilkan kalirnat-kalimat yang belum pernah didengar sebelumnya, yakni kode yang mendasari semua ujaran dalam satu bahasa". Sementara, penampilan dijabarkan sebagai "realisasi kode itu dalam pemakaian bahasa yang sebenarnya, yakni ujaran itu sendiri".} (\textit{linguistic competence}) yang berarti pengetahuan penutur/pendengar terhadap bahasanya. Dalam kemampuan linguistik, istilah 'gramatikal' menjadi salah satu tolok ukur.   \cite{chomsky1965syntactic} menegaskan perbedaan antara istilah tersebut dengan 'dapat diterima' atau \textit{acceptable}.  Berbeda dengan konsep 'gramatikal', keterterimaan (\textit{acceptability}) merupakan konsep dan tolok ukur dalam studi penampilan linguistik (\textit{linguistic performance}) \citep{chomsky1965syntactic}. Studi terhadap penampilan linguistik merupakan studi terhadap penggunaan bahasa oleh penutur dalam situasi yang nyata dengan menggunakan \textit{real utterance}. Dualisme\footnote{Dalam pandangan teknis, Chomsky melihat teori linguistik sebagai hal yang bersifat mentalistik karena berkenaan dengan penjajakan realitas mental yang mendasari perilaku nyata. Chomsky menekankan hal tersebut untuk membedakan dualisme \textit{competence-performance} dengan \textit{langue-parole} yang dikemukakan Saussure \citep{key2017course}. Bagi Chomsky, \textit{langue} hanya mencakup inventori sistematis dari unit-unit linguistik. Dalam konteks realitas mental, kemampuan lebih dekat dengan konsepsi pemikiran dalam proses generatif yang digagas oleh \cite{von1972origin}.} yang digagas  \cite{chomsky1965syntactic} ini berangkat dari hasil analisis yang menunjukkan bahwa kemampuan tidak selalu dapat tercermin secara langsung pada penampilan.

Perkembangan diskusi mengenai dualisme \textit{competence-performance} atau kemampuan-penampilan sangat menekankan pada hubungan kedua aspek tersebut. Salah satunya adalah \cite{sagwasow2011pccg} yang menyatakan bahwa teori-teori terkait kemampuan linguistik arus dapat menjadi basis untuk model yang digunakan dalam menganalisis penampilan linguistik. Para linguis yang mendukung paham ini menambahkan bahwa peneliti dalam bidang teori gramatikal atau studi kemampuan linguistik seharusnya mengintegrasikan studi penampilan linguistik yang memanfaatkan \textit{real utterance} sehingga dapat berperan sebagai dasar fakta empiris untuk pengembangan ranah kemampuan linguistik. Keterkaitan antara kemampuan dan penampilan linguistik ini pertama diungkit oleh J.H. Greenberg (\citealp{greenberg1963some, greenberg1966language}) yang memperlihatkan pola korelasi antara temuan penampilan dan tata bahasa terkait evolusi dari aturan bahasa yang mungkin muncul dari penggunaan bahasa dalam keseharian secara reguler. \cite{givon1979syntax} juga melakukan observasi bahwa kecenderungan pilihan (preferensi) dalam sebuah bahasa mungkin berkorelasi dengan persyaratan kategoris (gramatikal). 

Perkembangan teori-teori lain yang berangkat dari dualisme kemampuan-penampilan seperti Teori Optimalitas atau \textit{Optimality Theory} menghasilkan temuan adanya alasan fungsional yang mendasari batasan gramatikal dasar (\citealp{haspelmath1999optimality, aissen1999markedness}). Teori lain seperti Teori Optimalitas Stokastik atau \textit{Stochastic Optimality Theory} (\citealp{bresnan2001soft, manning2003probabilistic}) mengintegrasikan temuan kecenderungan pilihan atau preferensi yang didapatkan dari studi penampilan linguistik  sebagai batasan lunak (\textit{soft constraints}) dan konvensi gramatikal sebagai batasan keras (\textit{hard constraints}).  Gabungan kedua batasan ini menjadi salah satu penerapan praktis dalam upaya untuk mengaitkan kemampuan dan penampilan linguistik. Tidak hanya dalam ruang lingkup bahasa Indonesia, penelitian mengenai keterkaitan kedua aspek dualisme tersebut juga masih terus berkembang secara signifikan dalam ruang lingkup global karena masih banyak aspek yang harus digali terkait mekanisme adaptif\footnote{Mekanisme adaptif yang dimaksud adalah dasar mekanisme sebuah bahasa yang mungkin dimiliki oleh semua bahasa. Termasuk di dalam mekanisme ini adalah fleksibilitas tata bahasa terhadap ujaran yang tidak gramatikal, namun masih dapat diterima (\textit{acceptable}).} \citep{kirby1999function}. Berbagai pertanyaan dari mekanisme adaptif ini patut dipertimbangkan sebagai tujuan akhir dari penelitian-penelitian terkait, seperti pada titik mana konvensi gramatikal dapat terbentuk dari variasi yang dihasilkan penampilan, bagaimana proses model gramatikal dapat mengatur preferensi penampilan, serta tolok ukur pertimbangan untuk penyaringan preferensi penampilan menjadi konvensi tata bahasa.


%-----------------------------------------------------------------------------%
\subsection{Efisiensi dan Urutan Kata (\textit{Word Order}) dalam Ranah Ilmu Sintaksis}
%-----------------------------------------------------------------------------%
Seperti yang dijelaskan sebelumnya, penelitian mengenai keterkaitan dualisme kemampuan-penampilan banyak mencakup pertimbangan terbentuknya preferensi dalam studi penampilan dan konvensi gramatikal. Dalam ruang lingkup sintaksis, hal ini berkaitan dengan kemampuan sebuah bahasa yang dimiliki penutur untuk dapat berekspresi melalui variasi ujaran yang tidak terbatas meskipun dengan struktur yang terbatas (\citealp{von1972origin, plotkin2000language}). Konsep ini menggambarkan hubungan erat antara sintaksis dengan kognisi manusia yang menghasilkan pandangan-pandangan dengan anggapan bahwa seiring waktu, bahasa dibentuk oleh batasan mekanisme kognisi manusia dan tekanan terkait akuisisi dan penggunaan bahasa \citep{plotkin2000language}. Berbagai gagasan yang diajukan terkait tekanan kognitif yang membatasi pembentukan struktur ujaran mencakup keterbatasan memori kerja \citep{slobin1973cognitive}, batasan atas pengolahan bahasa dan persepsi \citep{bever1970cognitive}, dan pertimbangan terhadap komunikasi yang efisien (\citealp{macwhinneybates1989cross, givon1991markedness, zipf1949human}). 

Berdasarkan diskusi mengenai pembentukan struktur ujaran dan pertimbangan terhadap komunikasi yang efisien, \cite{hawkins2004efficiency} menemukan bahwa efisiensi yang dimaksud terintegrasikan dalam proses produksi dan pemahaman bahasa. Temuan ini memperlihatkan bahwa tata bahasa telah mengkonvensionalisasikan struktur ujaran sehingga proporsional terhadap preferensi pada studi performance \citep{hawkins2004efficiency}. Analisis ini didapatkan dengan memetakan pola seleksi dalam korpora dan eksperimen psikolinguistik untuk melihat kemudahan pengolahan bahasa. Kedua sumber data menunjukkan temuan yang serupa, yaitu adanya kecenderungan untuk memilih pronomina persona dalam klausa relatif pada lingkungan yang lebih kompleks sehingga efisiensi diraih dengan meminimalkan ranah \textit{domain minimization} \citep{hawkins2004efficiency}. Dalam konteks urutan konstituen, hal ini berarti frasa dan klausa harus dirakit menjadi kelompok-kelompok yang dapat direpresentasikan dengan diagram struktural, namun diproduksi dalam bentuk ujaran yang linear. Beberapa bentuk urutan dapat mereduksi jumlah konstituen (atau jumlah tautan) yang dibutuhkan agar pendengar/pembaca dapat mengidentifikasi frasa induk serta frasa pelengkap lainnya. Hawkins menegaskan bahwa tata bahasa terkait kemampuan linguistik berbeda dengan penampilan linguistik, mendukung dualisme \cite{chomsky1965syntactic}. Namun, penelitian yang dilakukannya juga memperlihatkan dukungan terhadap usaha dalam mengaitkan kedua aspek. \cite{hawkins2004efficiency} menemukan hasil studi penampilan yang tidak kategoris atau gramatikal untuk diikutsertakan di dalam tata bahasa, tetapi memiliki preferensi yang sangat kuat sehingga dapat diterima. Penelitian \cite{hawkins2004efficiency} yang menunjukkan adanya struktur pokok yang sama antara penampilan dan kemampuan baru ditemukan pada data bahasa dengan urutan kata tetap (\textit{fixed word order}) seperti Bahasa Inggris. 

Teori lain menyangkut efisiensi terkait struktur ujaran yang tengah berkembang adalah Kepadatan Informasi Seragam atau \textit{Uniform Information Density} (UID) (\citealp{jaeger2007speakers, frank2008speaking}) yang menyatakan bahwa setiap konstituen di dalam ujaran  sebaiknya membawa porsi atau jumlah informasi yang kurang lebih sama. Dalam ilmu kognitif, teori ini sangat menarik karena menciptakan keseimbangan antara dua tuntutan komunikasi yang saling berkompetisi, yaitu efisiensi dan ketepatan (akurasi)\footnote{Ketepatan atau akurasi makna berkenaan dengan kesesuaian antara makna sebuah ujaran yang ingin disampaikan oleh penutur dengan makna yang dipahami oleh penerima informasi melalui produksi bahasa.}. Keseimbangan ini berarti penutur dapat mentransmisikan pesannya seefisien mungkin sekaligus menghasilkan sinyal ujaran yang kuat untuk menghadapi gangguan sehingga makna dapat ditangkap secara langsung \citep{jaeger2007speakers}.

\cite{gil2001creoles} melakukan penelitian tentang kompleksitas bahasa dengan menyoroti urutan kata dan beranggapan bahwa urutan kata akan lebih kompleks jika menerapkan semakin banyak aturan. Salah satu obyek yang ditelitinya adalah dialek Indonesia Riau yang memiliki \textit{free word order} sebagai ilustrasi salah satu tata bahasa yang tidak terlalu kompleks \citep{gil2001creoles}. \cite{dryer2007word} juga mengungkapkan bahwa salah satu perbedaan utama dari bahasa-bahasa di dunia adalah karakteristik dalam mengurutkan konstituen atau kata. Dalam penelitiannya, \cite{dryer2007word} menjelaskan bahwa urutan kata dalam sebuah bahasa tidak hanya berkaitan dengan subyek (S), obyek (O), dan predikat (V), namun lebih umum kepada urutan konstituen pada level apapun, baik itu klausa maupun frasa. Secara umum, terdapat enam (6) kombinasi utama dalam mengurutkan kata pada sebuah ujaran (dengan hanya memandang S, V, dan O) yaitu SVO, SOV, VSO, VOS, OVS, dan OSV \cite{dryer2007word}. Prinsip urutan kata SOV dan SVO mencakup 86\% dari variasi urutan kata yang ditemukan berdasarkan tata bahasa pada bahasa-bahasa di seluruh dunia \citep{dryer2005world}. Dalam beberapa literatur, para linguis mengajukan bahwa SOV merupakan urutan yang muncul paling pertama pada bahasa manusia (\citealp{givon1979syntax, gell2011origin, newmeyer2000language}) dan memaparkan bukti yang ditemukan pada banyak bahasa Eropa yang menunjukkan adanya perubahan searah dari SOV menjadi SVO\footnote{Frekuensi relatif atau rasio jumlah kategori yang diobservasi terhadap keseluruhan jumlah observasi untuk SVO pada bahasa-bahasa yang masih aktif di dunia masih didebatkan hingga saat ini.} \citep{newmeyer2000language}. 
 
%-----------------------------------------------------------------------------%
\subsection{Relasi dan Struktur Ujaran pada Struktur Frasa dan Dependensi}
%-----------------------------------------------------------------------------%

Dalam mengevaluasi definisi dependensi yang bertitik tolak dari definisi konstituensi, definisi konstituen itu sendiri perlu diperhatikan. \cite{bloomfield1933language} memberikan ruang untuk mengembangkan definisi konstituen sintaksis sehingga dapat dikembangkan untuk menyempurnakan definisi dependensi. Definisi \cite{bloomfield1933language} atas gagasan konstituen pertama dibahas dalam bab \textit{Morphology} di mana Bloomfield menjabarkan fungsi sebuah morfem. Dalam bab yang membahas sintaksis, tertulis bahwa konstruksi sintaksis merupakan konstruksi di mana tidak ada konstituen langsung yang merupakan bentuk terikat. Bloomfield tidak menjelaskan definisi konstituen secara eksplisit, melainkan definisi ini terbentuk melalui contoh-contoh dari kelas-kelas distribusional. Sebagian besar pokok pembahasan pada bab menyangkut konstituen menitikberatkan pada induk (\textit{head}) dari sebuah konstruksi. \cite{gerdes2013defining} beranggapan bahwa gagasan Bloomfield sebaiknya dilihat sebagai pendahulu gagasan 'koneksi' (yang dalam pengembangan teori berikutnya disebutnya dengan konstruksi) dan 'dependensi'. \cite{chomsky1986barriers} melihat bahwa konstituen hanya hadir di dalam struktur sintaktis sebuah ujaran atau kalimat, namun Chomsky tidak pernah memberikan kriteria pasti dalam mengidentifikasi konstituen. \cite{gleason1961introduction} mengajukan kriteria untuk mendefinisikan konstituen dan membangun struktur konstituen dari bawah (\textit{bottom up}). \cite{gleason1961introduction} menganggap setiap konstituen pada ujaran saling memiliki relasi dengan status tertentu. \cite{haegeman1994introduction} menyatakan bahwa semua konstituen dalam ujaran diatur secara hierarkis menjadi unit yang lebih besar (frasa), namun tidak menyinggung atau mendefinisikan konstituen itu sendiri. Penjelasan \cite{haegeman1994introduction} merupakan salah satu dari berbagai teori sintaksis yang menunjukkan bahwa struktur sintaksis memiliki hierarki. Hal ini berarti tautan di dalam struktur tersebut memiliki arah dan tautan yang memiliki arah atau hierarki pada perkembangan teori berikutnya disebut dengan  dependensi. 

\cite{hudson2010introduction} dalam \textit{Word Grammar} membahas keserupaan struktur frasa dengan struktur dependensi sebagai berikut:
\begin{itemize}
\item Konstituen dikelompokkan menjadi frasa yang memiliki struktur abstrak melebihi urutan linear dan pendampingan linear. Frasa ini secara umum merefleksikan dependensi antarkonstituen di mana setiap frasa dibangun meliputi konstituen atau kata induk (\textit{head-word}) yang menjadi tempat bergantung konstituen terikat lainnya. Relasi ini tidak hanya berorientasi terhadap makna (\textit{meaning}), tetapi juga terhadap struktur ujaran yang tercermin pada urutan konstituen secara linear. Sehingga, sebuah frasa dapat menjadi unit makna dan juga, secara langsung, menjadi sebuah untaian konstituen yang berkelanjutan. Struktur frasa memperlakukan frasa sebagai dasarnya dan memetakan simpai (\textit{node}) terpisah untuk setiap frasa. Dependensi dalam \textit{Word Grammar} dapat diinterpretasikan dalam konteks frasa, yaitu setiap kata yang memiliki setidaknya satu konstituen terikat (\textit{dependent}) adalah induk dari frasa yang terdiri atas konstituen tersebut dan semua konstituen yang bergantung padanya.
\item Setiap frasa bersifat endosentris dalam arti bahwa setiap frasa mengandung satu konstituen yang menjadi induk yang klasifikasinya menentukan distribusi dari keseluruhan frasa. Dalam struktur frasa, induk memiliki label kelas yang sama dengan keseluruhan frasa (seperti contoh nomina dalam frasa nomina). Terkait dengan arah, arah dependensi tidak menuju induk namun menunjuk menjauhi induk dan memperlihatkan konstituen apa yang diikatnya.
\item Setiap konstituen dalam frasa memiliki fungsi gramatikal yang dapat digolongkan sesuai dengan kategori umum seperti 'induk', 'pelengkap', 'subyek', 'keterangan', dan seterusnya. Sistem ini membedakan induk dari konstituen yang bergantung padanya dan menunjukkan bagaimana konstituen yang bergantung tersebut memiliki relasi terhadap keseluruhan frasa (struktur frasa) ataupun induk (struktur dependensi). \cite{hudson2007language} memaparkan bahwa terdapat kesepakatan utama antara keduanya, meskipun bersifat implisit, yaitu relasi dalam ujaran membentuk hierarki dari yang paling umum hingga spesifik. Hierarki ini merupakan bukti jelas terhadap klasifikasi hierarkis atas relasi yang juga merupakan salah satu prinsip utama \textit{Word Grammar}.
\item Struktur internal sebuah frasa terbebas dari relasi di luar frasa tersebut. Dalam struktur frasa, relasi internal ditampilkan di bawah simpai frasa 'X' yang relevan, sedangkan struktur dependensi menunjukkannya dengan panah yang menjauhi kata induk.
\item Bagian dari sebuah frasa dapat bertindak juga sebagai bagian dari frasa lain. Setidaknya dalam struktur frasa yang lebih tradisional, hal ini dapat dilihat melalui melalui jejak (\textit{trace}) \citep{chomsky1986barriers} atau melalui pembagian struktur \citep{pollard1994head}. Dalam struktur dependensi, relasi ganda ini ditunjukkan secara eksplisit melalui panah yang menghubungkan satu konstituen dengan dua konstituen lain (atau lebih).
\end{itemize}

Meskipun terdapat keserupaan yang penting antara struktur frasa dengan struktur dependensi, \cite{hudson2007language} menekankan perbedaan mendasar di antara keduanya. Perbedaan pertama melibatkan perlakuan terhadap relasi gramatikal (atau fungsi gramatikal tradisional). Dalam struktur dependensi, fungsi gramatikal merupakan hal yang mendasar dan digolongkan ke dalam tipe-tipe dependensi yang berbeda. Cara pandang \cite{hudson2007language} terhadap bahasa sebagai jejaring berperan dalam penjabaran ini karena relasi yang terbentuk dalam ujaran selalu diekspresikan langsung oleh tautan antarsimpai. Perbedaan kedua adalah bahwa struktur dependensi tidak memiliki simpai frasa. Frasa dan dependensi merupakan cara-cara alternatif untuk melihat bagaimana kata-kata saling berhubungan.  

Dependensi telah berintegrasi ke dalam berbagai area terkait studi bahasa, terutama linguistik komputasional. Sebagai contoh, evaluasi terhadap berbagai metode penguraian kalimat (\textit{parser}) untuk penerapan praktis menunjukkan bahwa mayoritas metode penguraian kalimat tersebut dibangun berdasarkan konsep dependensi \citep{molla2000answer}. Kumpulan diagram pohon struktur dependensi atau \textit{dependency treebanks} dari korpora hasil analisis dependensi banyak ditemukan dalam berbagai macam bahasa, termasuk bahasa Indonesia (\citealp{marcus1993building, abeille2004enriching, carroll2003parser, lin2003dependency, green2012indonesian}). Salah satu area penelitian yang sedang dieksplorasi adalah pengkondisian metode penguraian kalimat untuk dapat memanfaatkan \textit{treebank} dependensi sebagai 'memori' atau data latihan dalam membantu menghasilkan analisis dengan kualitas yang semakin mendekati \textit{real utterance} (\citealp{nivre2006maltparser, nivre2004incrementality}). Salah satu daya tarik analisis dependensi untuk kerangka kerja komputasional adalah sedikitnya pertentangan mengenai hasil analisis jika dibandingkan dengan variasi yang ditemukan dalam analisis struktur frasa \citep{carroll2003parser}. Hal ini disebabkan karena struktur dependensi telah menunjukkan bahwa bahasa menyimpan karakteristik matematis yang sangat kuat \citep{i2004patterns}. Dalam penelitiannya, \cite{i2004patterns} menemukan bahwa jejaring dependensi cenderung memiliki sistematika koneksi antarsimpai yang kuat dan bukan hanya distribusi acak dari tautan antarsimpai sehingga dapat dikalkulasi lebih akurat dibandingkan struktur frasa.


%-----------------------------------------------------------------------------%
\subsection{Penelitian Terdahulu}
%-----------------------------------------------------------------------------%

Sebelum penelitian ini dilakukan, telah ada beberapa penelitian lain yang menyinggung topik serupa meskipun dari perspektif yang berbeda. Namun, belum ada penelitian serupa yang menitikberatkan pada penerapan dependensi dalam bahasa Indonesia. Penelitian yang memanfaatkan teori dependensi dan menggunakan data bahasa Indonesia sejauh yang ditemukan hanya fokus terhadap pengembangan metode untuk penguraian kalimat. Semua penelitian terdahulu ini menjadi referensi awal dalam menyusun kerangka konseptual dan metodologi dalam penelitian ini.

\cite{jaeger2006redundancy} melakukan penelitian tentang pengulangan dan reduksi sintaktik dalam ujaran spontan pada bahasa Inggris. \cite{jaeger2006redundancy}  menitikberatkan pada kasus-kasus yang mengandung pengurangan sintaktik dengan dengan menganalisis bukti-bukti data korpus berisi ujaran spontan. Ujaran spontan ini diduga mengalami banyak pengurangan sintaktik. Sehubungan dengan metode yang menjadi pendekatan analisis kuantitatif penelitian ini, Jaeger juga menggunakan model regresi statistik modern untuk menjaga substansi studi dari tantangan kajian berbasis korpus pada umumnya. Di akhir disertasinya, Jaeger mengajukan pendekatan untuk mempelajari apa yang digunakan oleh penutur untuk merekam jejak kemungkinan-kemungkinan pengurangan jarak dependensi antarkonstituen. Jaeger juga pernah melakukan penelitian dengan Gildea \citep{gildea2015human} untuk membuktikan bahwa manusia menyusun informasi secara efisien. Dalam penelitian ini, \cite{gildea2015human} memanfaatkan data ragam tulis dan lisan dalam bahasa Inggris serta data ragam tulis dalam bahasa Arab, Ceko, dan Mandarin. Temuan yang didapat menitikberatkan pada densitas informasi dengan menggunakan indikator panjang dependensi (\textit{Dependency Length}) dan bukan jarak dependensi (\textit{Dependency Distance})\footnote{Perbedaan antara panjang dependensi (\textit{Dependency Length}) dan jarak dependensi terletak pada cara penghitungannya. Panjang dependensi menekankan pada jumlah semua niliai tautan dependensi dalam sebuah ujaran, dan jarak dependensi menekankan pada rata-rata jarak antara dua konstituen dalam ujaran. Penjelasan lebih detail akan dijabarkan pada sub-bab berikutnya.}. Sebelumnya, \cite{liu2008dependency} menggunakan indikator jarak dependensi {\textit{Dependency Distance}} untuk melihat adanya efisiensi ujaran pada 20 bahasa. Pengajuan konsep jarak dependensi sebagai ukuran kesulitan produksi atau pemahaman ujaran disempurnakan oleh \cite{liu2017dependency} pada publikasi terbaru di tahun 2017. Studi korpus lintas bahasa secara makro juga beberapa kali dilakukan. Salah satunya adalah \cite{gildea2010grammars} yang menyelidiki rata-rata panjang dependensi dari dua bahasa yang memiliki hubungan sintaksis cukup dekat, yaitu bahasa Inggris dan Jerman. Selain itu, \cite{i2004euclidean} juga melakukan penelitian menggunakan data bahasa Ceko dan Romania, sedangkan \cite{futrell2015large} memanfaatkan data 37 bahasa termasuk bahasa Indonesia.

Penelitian \cite{futrell2015large} memiliki tujuan untuk memaparkan keserupaan temuan secara lintas bahasa terkait efisiensi ujaran. pengurangan panjang dependensi antarkata yang memiliki relasi semantik. Data yang digunakan merupakan korpus berskala besar dari 37 bahasa mencakup teks ragam tulis dari buku, blog, penelitian dan media massa yang dianalisis secara masing-masing dan saling dibandingkan. \cite{futrell2015large} menggunakan metode penguraian kalimat yang dikembangkan oleh para peneliti tiap-tiap bahasa dan penghitungan statistik yang diangkat oleh \cite{gelman2007data}. Untuk bahasa Indonesia, \cite{futrell2015large} memanfaatkan data dan mengadopsi pendekatan yang dilakukan oleh \cite{green2012indonesian}. Hasil yang didapatkan adalah bukti berskala besar adanya efisiensi ujaran yang serupa namun memiliki variasi dalam 37 bahasa. Dalam temuannya, \cite{futrell2015large} menyebutkan bahwa bahasa Indonesia menunjukkan tingkat efisiensi yang terbilang sangat tinggi dibandingkan dengan 36 bahasa lainnya. 

Penelitian \cite{green2012indonesian} yang juga serupa dengan penelitian \cite{kamayani2011dependency} memiliki tujuan untuk menghasilkan pendekatan penguraian kalimat bahasa Indonesia dengan memanfaatkan teori dependensi. \cite{kamayani2011dependency} menggunakan 20 kalimat dalam bahasa Indonesia yang diambil secara acak dan menggunakan sekumpulan pendekatan komputasional yang diangkat oleh beberapa ahli linguis komputasional (\citealp{nivre2006dependency, covington2001fundamental, de2008stanford}). Hasil pendekatan untuk penguraian kalimat berdasarkan relasi semantik dalam bahasa Indonesia melewati pengujian secara sukses pada 20 kalimat yang menjadi obyek penelitian tersebut.

Penelitian \cite{green2012indonesian} lebih menitikberatkan pada pendekatan untuk menghasilkan \textit{treebank} berdasarkan konsep dependensi dalam bahasa Indonesia. \textit{treebank} adalah kerangka kerja berupa diagram pohon yang menggambarkan struktur ujaran dihasilkan dari kalimat-kalimat yang telah diuraikan dan diberi anotasi. Konstituen ujaran dalam diagram tersebut diberi catatan informasi terkait kelas kata dan jenis relasi antarkonstituennya. \textit{Treebank} merupakan sumber daya utama dalam pengolahan dan analisis linguistik komputasional atau \textit{Natural Language Processing} (NLP). Data yang digunakan adalah korpus dalam bahasa Indonesia berskala besar mencakup teks ragam tulis dari kumpulan artikel milik BPPT. \cite{green2012indonesian} menggunakan kumpulan pendekatan komputasional yang diangkat oleh \cite{kubler2009dependency}. Pendekatan yang digunakan \cite{green2012indonesian} ini dikembangkan secara berkelanjutan sehingga dapat dimanfaatkan untuk menjadi salah satu acuan dalam menguraikan kalimat. Kerangka kerja ini memperlihatkan implementasi yang sukses dari penguraian kalimat berdasarkan konsep dependensi.

\cite{irmawati2015dependency} menambahkan skema catatan atau anotasi tambahan pada \textit{treebank} dalam bahasa Indonesia yang dikembangkan \cite{green2012indonesian}. Data yang digunakan adalah 650 kalimat dalam bahasa Indonesia yang diambil secara acak. \cite{irmawati2015dependency} mempertimbangkan kekayaan fenomena morfologis dalam bahasa Indonesia seperti afiksasi dan klausa non-verba dalam mengembangkan skema ini. Makalah ini menguraikan fenomena tersebut dan menjelaskan bagaimana skema yang dikembangkan mengakomodir karakter morfologis dalam bahasa Indonesia. Hal ini berguna sebagai pertimbangan tambahan dalam menentukan tautan dependensi yang terbentuk dari dua konstituen. 

Penelitian-penelitian di atas tidak memiliki tujuan untuk memberikan wawasan dari segi linguistik, namun sangat bermanfaat untuk menjadi metode utama untuk menguraikan kalimat dengan dasar teori dependensi yang kemudian dilanjutkan dengan analisis kuantitatif maupun kualitatif. Berkaitan dengan hal ini, \cite{liu2008dependency} adalah linguis yang banyak melakukan penelitian seputar panjang kalimat, jarak dan arah dependensi, serta struktur sintaksis secara umum. \cite{liu2008dependency} mengajukan penggunaan 'jarak dependensi' atau \textit{dependency distance} sebagai indikator kesulitan pemahaman bahasa dan telah dikembangkannya dalam penelitian \cite{liu2017dependency} sebagai perspektif baru untuk melihat pola sintaktis dalam penggunaan bahasa sehari-hari. Hipotesis Pengurangan Jarak Dependensi atau \textit{Dependency Distance Minimization} (DDM) yang merupakan kerangka untuk melihat efisiensi ujaran melalui nilai rata-rata jarak dependensi digunakan dalam penelitian ini sebagai dasar analisis kuantitatif. Temuan yang memperlihatkan indikasi DDM menyatakan adanya kecenderungan penutur menghasilkan ujaran dengan struktur sintaksis yang meminimalkan jarak dependensi antarkonstituen. Metode kalkulasi jarak dependensi yang dikembangkannya diajukan untuk mengatasi masalah sensitivitas nilai panjang dependensi terhadap panjang kalimat \citep{jiang2015effects}. Dalam kajian ini, \cite{jiang2015effects} melihat pengaruh panjang kalimat terhadap jarak dan arah dependensi dengan memanfaatkan 42 kalimat dalam korpus Bahasa Inggris dan Mandarin. J \cite{jiang2015effects} menggunakan teori dan metode yang diangkat oleh \cite{tesniere1959elements}, \cite{hudson2007language}, dan \cite{nivre2006maltparser}. Melalui analisisnya,  \cite{jiang2015effects} menemukan bahwa pola DDM antara Bahasa Inggris dan Mandarin identik. Namun, terdapat perbedaan dalam karakter pengurangan jarak tersebut seperti arah tautan antarkonstituen kalimat. Hasil temuan menunjukkan bahwa mekanisme kognisi manusia menjadi faktor utama dalam mempengaruhi jarak antarkonstituen, namun faktor linguistik seperti panjang kalimat dapat membentuk pola yang spesifik. Dalam kasus ini, \cite{jiang2015effects} juga berkesimpulan bahwa bahasa Mandarin menuntut daya ingat manusia untuk bekerja lebih banyak dibandingkan bahasa Inggris. 

Liu juga bekerja sama dengan Wang \citep{wang2017effects} untuk memaparkan dampak \textit{genre} atau aliran teks terhadap jarak dan arah tautan dependensi antarkonstituen. Kajian ini menggunakan metode kuantitatif untuk melihat persebaran nilai jarak antarkonstituen yang memiliki relasi semantik dalam korpus bahasa Inggris mencakup 10 aliran teks yang diambil dari British National Corpus. Temuan penelitian menunjukkan bahwa pengaruh genre atau aliran data ragam tulis terhadap jarak dan arah tautan dependensi antarkonstituen sangat kecil, namun Wang dan Liu menemukan bahwa nilai jarak antarkonstituen lebih kecil pada teks ujaran seperti dialog dibandingkan dengan data yang lain serta nilai jarak antarkonstituen pada aliran teks fiksi yang melibatkan imajinasi lebih besar dibandingkan dengan aliran teks yang bersifat informatif.

\todo{masukintabel penelitian terdahulu}

Secara garis besar berdasarkan \todo{tabel}, rangkuman mengenai penelitian-penelitian terdahulu yang berkaitan dengan efisiensi ujaran berdasarkan dependensi antarkonstituen pada tataran kalimat dalam bahasa Indonesia adalah sebagai berikut:
\begin{itemize}
\item Penelitian yang menggunakan teori dependensi dengan korpus data bahasa Indonesia saya temukan memiliki tujuan untuk menghasilkan kerangka kerja penguraian kalimat (\citealp{kamayani2011dependency, green2012indonesian, irmawati2015dependency}).
\item Penelitian yang menggunakan teori dependensi untuk melihat efisiensi ujaran melalui nilai jarak atau panjang dependensi antarkonstituen saya temukan dilakukan dengan korpus bahasa lain seperti Inggris dan Mandarin (\citealp{jiang2015effects, wang2017effects}). Terdapat penelitian yang melibatkan bahasa Indonesia, namun tidak mendetail karena bersifat lintas bahasa berskala besar \citep{futrell2015large}.
\item Dasar pemilahan korpus data, terutama data bahasa Indonesia, pada penelitian-penelitian di atas hanya didasarkan alasan kuantitas sehingga tidak terfokus pada kualitas data. 
\end{itemize}
Berdasarkan tinjauan pustaka di atas, terdapat kerumpangan pada beberapa aspek. Hingga saat ini, belum ada kajian yang mencermati efisiensi ujaran pada bahasa Indonesia dari segi dependensi yang dapat memberikan wawasan dari aspek linguistik, dan belum adanya kajian terkait konsep tersebut yang memanfaatkan ujaran lisan sebagai data penelitian. Oleh karena itu, posisi penelitian tesis ini yang membedakannya dengan penelitian-penelitian sebelumnya adalah sebagai berikut:
\begin{itemize}
\item Penelitian tesis ini menitikberatkan pada uji hipotesis efisiensi ujaran dalam bahasa Indonesia dan pemaparan strategi efisiensi ujaran yang dilihat melalui pengurangan panjang jarak dependensi antarkonstituen terutama pada aspek panjang kalimat, jarak dan arah dependensi, serta perubahan valensi pada simpai pusat yang seluruhnya berdampak pada urutan konstituen dan struktur ujaran.
\item Terkait dengan pemilihan aliran teks, penelitian ini memanfaatkan korpus jurnalistik dalam bahasa Indonesia karena fokus utama adalah melihat efisiensi ujaran. Hal ini berkaitan dengan asumsi bahwa teks informatif akan menunjukkan efisiensi ujaran lebih tinggi dibandingkan teks imajinatif \citep{wang2017effects}. Korpus data ini mencakup ragam tulis dan lisan untuk mendapatkan analisis yang menyeluruh terkait perbedaan karakter yang ada pada ujaran yang lebih spontan.
\item Penelitian ini menggunakan pendekatan kuantitatif untuk mendapatkan gambaran umum terkait pola pengurangan panjang dan jarak dependensi serta pendekatan kualitatif untuk menguraikan struktur ujaran, urutan konstituen dan perubahan valensi kata berdasarkan korpus yang diteliti.
\end{itemize}

%-----------------------------------------------------------------------------%
\section{Konsep Dependensi}
%-----------------------------------------------------------------------------%
Seperti yang telah diungkit sebelumnya, kata merupakan unit utama dari sintaksis dan semua relasi gramatikal menghubungkan satu kata dengan kata lainnya. Relasi ini sebelumnya dinamakan 'koneksi' atau \textit{connection} \citep{tesniere1959elements}. Istilah dependensi itu sendiri merupakan turunan dari 'koneksi' yang merujuk kepada hubungan asimetris antara kata superordinat dan subordinat \citep{hudson1984word}. \cite{tesniere1959elements} mendeskripsikan dependensi pada level mentalitas, yaitu pikiran memahami koneksi (\textit{the mind perceives connections}). Definisi dependensi yang formal dikemukakan pertama kali oleh \cite{lecerf1960programme} serta \cite{gladkij1966lekcii} yang memperlihatkan adanya kemungkinan untuk memetakan hierarki dependensi, mulai dari hierarki konstituen dengan menggunakan \textit{heads}\footnote{Istilah \textit{heads} di sini berbeda dengan \textit{head} (induk) dalam teori dependensi. Pada konteks ini, \textit{heads} merupakan sebutan lain untuk konsep struktur frasa}. Para linguis lain juga turut berkontribusi terhadap penyempurnaan definisi dependensi. \cite{mel'vcuk1988dependency} mengajukan definisi dependensi sebatas fragmen yang mencakup dua kata yang saling berhubungan. \cite{garde1977ordre} memperluas definisi dependensi dengan mempertimbangkan tidak hanya kata, tetapi juga elemen-elemen lain yang memberikan kontribusi pembentukan makna sebuah ujaran. \cite{schubert1987metataxis} mencoba untuk mendefinisikan dependensi sebagai kesertaan terarah (\textit{directed co-occurrence}) dan secara eksplisit mengikutsertakan relasi kesertaan kata-kata yang berjauhan. \cite{schubert1987metataxis} menjabarkan arah kesertaan dengan mengatakan bahwa kesertaan dari kata-kata tertentu yang terikat (\textit{dependent}) dimungkinkan karena hadirnya kata lain yang mengendalikannya (\textit{governor}). Namun, \cite{schubert1987metataxis}, penentuan bentuk sebaiknya tidak menjadi kriteria untuk menetapkan garis kesertaan. Sementara, \cite{hudson1994discontinuous} mengajukan usul untuk mempertahankan tipe dependensi seperti ini.

%-----------------------------------------------------------------------------%
\subsection{Akar (\textit{Root}), Induk (\textit{Head}), Konstituen Terikat (\textit{Dependent}), dan Simpai (\textit{Node})}
%-----------------------------------------------------------------------------%
'Koneksi' yang diungkit oleh \cite{tesniere1959elements} merupakan komponen yang harus ada dalam konstruksi sebuah ujaran. Namun, dalam memahami sebuah ujaran, 'koneksi' ini juga butuh dijabarkan dan ditelaah sehingga kedua konstituen yang berhubungan dapat menempati posisi tertentu dalam konstruksi ujaran. Hubungan kedua konstituen ini menjadi pondasi atau dasar dari sintaksis struktural. Konsep hubungan ini juga tercermin pada kata 'sintaks' itu sendiri yang dalam bahasa Yunani berarti 'pengaturan (dalam sebuah kondisi)'. \cite{tesniere1959elements} mengumpamakan interaksi dalam sebuah ujaran sebagai pertunjukan teater, sehingga dari segi fungsi, konstituen-konstituen dalam sintaksis struktural dikategorikannya menjadi tiga (3), yaitu:
\begin{itemize}
\item \textbf{Proses} (\textit{process}), yang umumnya diekspresikan melalui verba di dalam kalimat.
\item \textbf{Aktor} (\textit{actors/actants}), yang umumnya diekspresikan melalui materi atau mahluk hidup, sehingga diasosiasikan dengan nomina.
\item \textbf{Situasi} (\textit{circumstances/circumstants}), yang berkenaan dengan waktu, tempat, dan kondisi di mana sebuah proses berlangsung.
\end{itemize}
Dari segi struktur, hubungan antara konstituen-konstituen ini memunculkan relasi dependensi antarkonstituen sehingga terbentuk tautan yang menyatukan konstituen yang superior dengan konstituen yang inferior (\citealp{tesniere1959elements, hudson2010introduction, heringer1993dependency}): 
\begin{itemize}
\item Apabila ada tautan dependensi dari A menuju B dan B menuju C, maka A mengatur B dan B mengatur C. A memiliki hierarki yang lebih superior dan bersifat mengendalikan B sehingga disebut sebagai \textbf{pengendali} (\textit{governor}), hubungan yang sama terjadi antara B dan C. Dalam dependensi, pengendali ini terbagi dua:
\begin{itemize} 
\item Jika konstituen tersebut merupakan induk dari keseluruhan ujaran atau kalimat, maka disebut sebagai \textbf{akar} (\textit{root}). Dalam contoh hubungan A, B, dan C. A merupakan akar.
\item Jika konstituen tersebut bukan merupakan induk dari keseluruhan ujaran atau kalimat namun mengikat konstituen lain, maka disebut sebagai \textbf{induk} (\textit{head}). Dalam contoh hubungan A, B, dan C. B merupakan induk.
\end{itemize}
\item B memiliki hierarki yang lebih inferior dan bersifat dikendalikan A sehingga disebut sebagai \textbf{subordinat} (\textit{subordinate}) atau \textbf{konstituen terikat} (\textit{dependent}), hubungan yang sama terjadi antara C dan B. Dalam struktur ujaran yang lebih kompleks, konstituen terikat juga dapat menjadi induk. Dalam contoh hubungan A, B, dan C, B merupakan konstituen terikat dari A sekaligus induk dari C.
\item Setiap kata yang utuh dapat berinteraksi untuk membentuk \textbf{simpai} (\textit{node}) \citep{tesniere1959elements}. Simpai ini dibedakan tergantung dari hierarki konstituen tersebut. Jenis simpai ini juga terbagi dua:
\begin{itemize}
\item Simpai pada akar disebut sebagai \textbf{simpai pusat} (\textit{central node}). Dalam contoh hubungan A, B, dan C, simpai pada A merupakan simpai pusat.
\item Simpai pada induk disebut hanya sebagai \textbf{simpai} (\textit{node}). Simpai ini ditemukan pada B dalam contoh hubungan A, B, dan C.
\end{itemize}
\end{itemize}

Dalam kalimat 'Ibu membeli bunga', kata 'membeli' merupakan verba atau kata kerja dan menjadi akar sehingga 'membeli' membentuk simpai pusat verbal (\textit{verbal central node}). Sementara itu, dalam kalimat 'Ibu membawa pulang lima kucing hitam', 'kucing' yang merupakan nomina yang mengikat kata 'lima' dan 'hitam' serta terikat pada 'membawa' sehingga bukan merupakan akar. Simpai pada kata 'kucing' disebut dengan simpai nomina (\textit{nominal node}). Sebuah ujaran dapat mengandung satu simpai atau lebih, sehingga dalam menelaah struktur ujaran, simpai pusat perlu diidentifikasi terlebih dahulu. Oleh karena itu, ujaran juga dikategorikan oleh \cite{tesniere1959elements} tergantung dari induk pada simpai pusatnya. Dalam bahasa-bahasa yang membedakan jelas antara verba dan nomina seperti kebanyakan bahasa-bahasa Eropa, jenis kalimat verba (memiliki simpai verba sebagai simpai pusatnya) ditemukan paling banyak disusul oleh kalimat nomina (\citealp{tesniere1959elements, hudson2007language}). Perlu saya tekankan bahwa sumber referensi dan diskusi mengenai dependensi dalam ranah linguistik di Indonesia belum banyak ditemukan sehingga belum ada kesepakatan yang cukup jelas terhadap padanan istilah-istilah dependensi dalam Bahasa Indonesia. Oleh karena itu, agar penelitian ini menjadi lebih kontekstual dalam Bahasa Indonesia, saya mengajukan padanan istilah-istilah dependensi dalam bahasa Indonesia seperti yang dipaparkan di atas dengan merujuk pada Kamus Linguistik yang dirumuskan oleh \cite{kridalaksana2008kamus}.

Kriteria hierarki dalam memetakan tautan dependensi antarkonstituen telah diajukan oleh \cite{bloomfield1933language}, \cite{zwicky1985heads}, \cite{garde1977ordre}, dan \cite{mel'vcuk1988dependency}. Kriteria paling umum adalah induk harus merupakan kata yang mengontrol distribusinya dan merupakan kata yang paling sensitif terhadap perubahan dari konteks. Namun, menurut \cite{gerdes2013defining}, pada penggalan ujaran tertentu, induk tidak selalu mengatur distribusi. Dalam contoh kalimat \textit{very little dogs slept}, kata \textit{little} memiliki relasi dengan \textit{very} dan \textit{dogs} memiliki relasi dengan \textit{slept}. Tetapi \textit{little dogs} tidak bisa dipisahkan untuk melihat distribusi \textit{dogs} maupun \textit{little} karena kedua penggalan \textit{very dogs slept} dan \textit{very little slept} tidak berterima. \cite{gerdes2013defining} beranggapan bahwa menentukan induk dari penggalan \textit{little dogs} berarti mengidentifikasi pengendali dari penggalan tersebut. Hal ini  dikarenakan pengendali keseluruhan kalimat atau akar dan induk dapat merupakan kata yang sama atau memiliki hubungan terdekat (jika berada di luar penggalan tersebut). Dalam contoh tersebut, karena \textit{slept} merupakan akar maka \textit{dogs} yang menjadi induk dalam relasi \textit{little dogs}. Hal ini menandakan bahwa untuk mendefinisikan struktur dependensi sebuah ujaran, akan dari keseluruhan ujaran perlu diidentifikasi terlebih dahulu agar induk lain yang ada bisa ditentukan \citep{gerdes2013defining}. 


%-----------------------------------------------------------------------------%
\subsection{Panjang (\textit{Length}), Jarak (\textit{Distance}), dan Arah Dependensi}
%-----------------------------------------------------------------------------%

Berdasarkan penelitian terdahulu, terdapat dua pendekatan yang digunakan dalam melihat efisiensi ujaran dari segi dependensi, yaitu dengan menganalisis \textbf{Panjang Dependensi} atau \textit{Dependency Length} (DL) dan \textbf{Rata-rata Jarak Dependensi} atau \textit{Mean Dependency Distance} (DD).  Hubungan struktural atau dependensi berupa tautan langsung memiliki jarak linear yang membentang antara akar, induk dan konstituen terikatnya. Jarak ini diukur dengan menghitung jumlah kata-kata yang terlibat di antaranya \citep{heringer1980syntax}. Perhitungan terhadap jarak ini pertama dilakukan oleh \cite{heringer1980syntax} dengan menghitung jumlah kata yang memisahkan konstituen terikat dengan induk atau akarnya. Urutan kata (\textit{word order}) dianggap sebagai media yang penting untuk membedakan bahasa-bahasa dilihat dari aspek fitur tipologisnya (\citealp{greenberg1963some, dryer1992greenbergian}. Dalam studi-studi termutakhir, urutan kata direfleksikan juga melalui arah dependensi \citep{hudson2007language}. Arah dependensi ini mengindikasikan apakah tautan dependensi diakhiri atau dimulai oleh induk. Sebagai contoh, \cite{hudson2003psychological} mengungkapkan bahwa bahasa Jepang merupakan salah satu bahasa yang kecenderungan arah dependensinya diakhiri oleh induk. Konsep penghitungan jarak ini berkaitan dengan logika kesulitan pengolahan dalam memori kerja \citep{hudson2007language}. \cite{hudson2007language} menjelaskan bahwa dalam menghasilkan ujaran yang berisi setidaknya dua kata, kedua kata tersebut akan tersimpan secara aktif di dalam memori kerja hingga dependensi antara keduanya terbentuk. Hal ini berarti jarak tersebut diasumsikan dapat merepresentasikan waktu atau usaha yang dibutuhkan untuk menyimpan kata pertama dan kedua dalam memori kerja hingga kata kedua selesai diproses (diujarkan, didengar, dibaca, atau ditulis). Dari konsep ini muncul hipotesis bahwa jarak dependensi yang semakin jauh menuntut usaha pengolahan dalam memori kerja yang semakin berat (\citealp{hudson2007language, gibson1998linguistic}).

Pada semua bahasa, baik yang memiliki urutan kata bebas (\textit{free word order}) ataupun urutan kata tetap (\textit{urutan kata tetap}), akan memiliki kaidah urutan tertentu yang menentukan urutan linear konstituen-konstituen dalam sebuah ujaran \citep{tesniere1959elements}. Hal ini berarti hierarki relasi antarkonstituen, dalam konteks dependensi, memiliki korelasi dengan arah secara linear \citep{greenberg1963some}. Terdapat perbedaan mengenai urutan linear antarkonstituen yang dibahas antara \cite{tesniere1959elements} dengan \cite{greenberg1963some}.  Dalam hal ini, \cite{greenberg1963some} lebih menekankan pada relasi gramatikal dalam sebuah ujaran, sedangkan \cite{tesniere1959elements} berupaya untuk membangun analisis menyeluruh terhadap sebuah ujaran berdasarkan relasi gramatikal tersebut. Penggunaan istilah 'cenderung' atau 'kecenderungan' (\textit{pr{\'e}f{\'e}rence} dalam bahasa Perancis) digunakan \cite{tesniere1959elements} untuk mengklasifikasikan bahasa berdasarkan arah dependensinya.

Hingga saat ini, belum ada konvensi mengenai arah dependensi terkait hierarki dan pengolahan dalam memori kerja. Namun, beberapa studi telah dilakukan untuk memberikan penjelasan tambahan mengenai penerapan aturan tata bahasa. Beberapa penelitian pada awal berkembangnya teori dependensi, telah dilakukan beberapa penelitian yang menemukan bahwa sebuah bahasa cenderung menerapkan arah relasi yang konsisten, baik itu diawali induk (\textit{head-first} atau \textit{head-initial}) maupun diakhiri induk (\textit{head-last} atau \textit(head-final))\footnote{Seperti yang dijelaskan sebelumnya, kata induk atau \textit{head} di sini sedikit berbeda dengan induk/kepala pada teori struktur frasa. Sehingga, konsep relasi yang diawali atau diakhiri induk yang dibahas pada bagian ini terbatas hanya pada penerapannya berdasarkan teori dependensi dan bukan struktur frasa.} (\citealp{hawkins1994performance, radford1997syntactic, vennemann1994linguistic}). Tindakan yang konsisten ini juga diasumsikan sebagai penerapan tata bahasa sebagai strategi untuk meminimalkan jarak antara induk dan konstituen terikatnya (\citealp{hawkins1994performance, frazier1985syntactic}). \cite{liu2010dependency} juga telah berupaya untuk memberikan validitas berdasarkan data mengenai arah dependensi dan kaitannya dengan klasifikasi topologis sebuah bahasa.

\begin{figure}
	\centering \includegraphics[width=1
	\textwidth] {pics/samebranching.png} \caption{Percabangan searah atau \textit{same branching}} 
\label{fig:samebranching} \end{figure}

\begin{figure}
	\centering \includegraphics[width=1
	\textwidth] {pics/mixedbranching.png} \caption{Percabangan beda arah atau \textit{mixed branching}} 
\label{fig:mixedbranching} \end{figure}

Perbandingan percabangan pada \pic~\ref{fig:samebranching} dan \pic~\ref{fig:mixedbranching} mengilustrasikan tautan-tautan yang dibentuk oleh masing-masing tipe percabangan. Percabangan searah atau \textit{same branching} dengan satu relasi yang berkelanjutan tidak mungkin akan secara konsisten terjadi, terutama pada ujaran yang lebih panjang sehingga asumsi ini dipertanyakan oleh \cite{temperley2008dependency} yang menganggap bahwa percabangan searah ini tidak selalu optimal. \cite{temperley2008dependency} meneruskan asumsi ini berdasarkan penelitian \cite{dryer1992greenbergian} terhadap 625 bahasa yang mengungkapkan bahwa ungkapan yang lebih tepat untuk menggambarkan karakter tata bahasa mengenai hal ini adalah bahwa frasa yang mengandung banyak kata cenderung memiliki percabangan searah, sedangkan arah dependensi frasa yang hanya mengandung satu kata ditemukan tidak konsisten, sehingga memunculkan hipotesis adanya gabungan percabangan searah dan percabangan beda arah (\textit{mixed branching}). 

%-----------------------------------------------------------------------------%
\subsection{Faktor-faktor yang Mempengaruhi Tautan Dependensi}
%-----------------------------------------------------------------------------%
Penutur sering menggunakan cara yang berbeda untuk menyampaikan pesan dengan makna yang sama \citep{kroch2001syntactic}. Meskipun penelitian ini hanya menitikberatkan pada variasi sintaktis terkait konsep dependensi, perbedaan karakteristik ujaran ini tidak hanya dalam konteks sintaktis, tetapi juga secara fonetis, fonologis, dan perbedaan leksikal. Keterkaitan perbedaan produksi ujaran secara sintaktis dengan memori kerja manusia telah beberapa kali dikaji terutama di ruang lingkup psikolinguistik (Jay, 2004; Levy, Fedorenko, & Gibson, 2013) maupun oleh peneliti di bidang ilmu kognitif (\citealp{futrell2015large, christiansen2016now, liu2017dependency}). Hal ini memperlihatkan bahwa konsep dependensi dapat memberikan kontribusi wawasan tidak hanya dari segi linguistik, namun juga secara multidisipliner.

%-----------------------------------------------------------------------------%
\subsubsection{Kognisi Manusia dan Produksi Ujaran}
%-----------------------------------------------------------------------------%
Panjang dan jarak dependensi antarkonstituen mengindikasikan upaya memori kerja manusia untuk menyimpan aktif satu konstituen hingga konstituen lain direalisasikan sehingga maknanya dapat tersampaikan secara utuh (\citealp{hudson2003psychological, liu2008dependency}). Pada penelitian-penelitian terdahulu dalam ruang lingkup ilmu kognitif, memori kerja manusia memiliki batasan (\textit{threshold}) terhadap waktu sehingga proses produksi ataupun pemahaman bahasa tidak bertahan lama dan bersifat sekilas untuk saat itu saja \citep{christiansen2016now}. Oleh karena itu, memori akan bekerja lebih banyak untuk memproduksi ataupun memahami ujaran dengan nilai panjang dan jarak dependensi yang lebih besar karena kedua indikator tersebut diasumsikan linear dengan waktu yang dibutuhkan untuk menyimpan konstituen secara aktif. Dapat dikatakan juga bahwa panjang dan jarak dependensi mengindikasikan tingkat kesulitan dari sebuah ujaran. 

Melalui pendekatan linguistik kuantitatif, beberapa penelitian telah menunjukkan bukti-bukti kecenderungan bahwa semakin besar nilai panjang dan jarak dependensi, pemrosesan dan analisis struktur sintaktis semakin sulit (\citealp{gibson1998linguistic, hiranuma1999syntactic, jiang2015effects, temperley2007minimization}). Meskipun penelitian ini tidak menelaah lebih dalam dari aspek kognisi manusia, temuan penelitian ini dapat dikaitkan dengan temuan penelitian-penelitian tersebut dan memberikan wawasan dari aspek linguistik dalam bahasa Indonesia untuk membentuk hipotesis lanjutan secara lintas bahasa.

%-----------------------------------------------------------------------------%
\subsubsection{Karakter Bahasa dan Ketatabahasaan}
%-----------------------------------------------------------------------------%
Beberapa penelitian lain yang menggunakan pendekatan linguistik kuantitatif terhadap korpus data teks telah memulai upaya untuk melihat keterkaitan panjang dan jarak dependensi dengan tipe bahasa (Hiranuma, 1999; Eppler, 2005; Liu & Xu, 2012), panjang kalimat (Oya, 2011; Ferrer-i-Cancho & Liu, 2014; Jiang & Liu, 2015), dan tata bahasa (Liu, 2008; Gildea & Temperley, 2010). Penelitian-penelitian ini belum menunjukkan konsensus terhadap perbedaan mendasar maupun besar dampak dari pengaruh tersebut terhadap dependensinya. Tetapi, beberapa penelitian ini menjadi pondasi untuk memberikan wawasan dan gambaran secara lintas bahasa mengenai kualitas dependensi yang dimiliki semua bahasa maupun yang menjadi keunikan bahasa masing-masing. \cite{jiang2015effects} menemukan bahwa distribusi jarak dependensi tidak dipengaruhi oleh tipe bahasa maupun panjang kalimat. Tetapi, kedua korpus data menunjukkan karakter yang sama, yaitu trend jarak dependensi yang semakin naik seiring dengan semakin panjangnya kalimat dan rata-rata jarak dependensi dalam bahasa Mandarin lebih tinggi dibandingkan bahasa Inggris. Sebelumnya, Oya (2013) mendemonstrasikan adanya perbedaan antara jarak dependensi pada teks fiksi dan non fiksi dalam data bahasa Inggris di mana nilai rata-rata jarak dependensi pada teks fiksi lebih kecil. Namun, \cite{wang2017effects} mengungkapkan bahwa temuan ini dihasilkan tanpa investigasi kemungkinan pengaruh panjang kalimat terhadap jarak dependensi. Hal ini signifikan karena perbedaan rata-rata panjang kalimat antara kedua teks dapat berkontribusi terhadap rata-rata jarak dependensi secara keseluruhan.

Dalam penelitian \cite{hiranuma1999syntactic} dan Liu et al., (2009b), teks dialog yang diperuntukkan sebagai ujaran lisan memiliki rata-rata jarak dependensi yang berbeda antara teks imajinatif dan teks informatif. Sehingga, \cite{hiranuma1999syntactic} dan Liu et al., (2009b) beranggapan bahwa teks yang semakin formal (dalam konteks ini adalah bahasa Inggris ragam tulis) memiliki tingkat sintaktis yang lebih sulit. Sejauh ini belum ada wawasan linguistik yang dapat memberikan informasi terhadap hal tersebut pada kasus bahasa Indonesia. Bahasa Indonesia memiliki konstruksi kalimat umum  SVO atau Subyek-Predikat-Obyek, tetapi memiliki aturan urutan kata yang cukup bebas terkait frasa nomina dan frasa verba (Irmawati dkk, 2015). K�bler dkk (2009) menyatakan bahwa analisis sintaktis berdasarkan teori dependensi dapat membantu memahami hubungan antarkonstituen pada bahasa-bahasa dengan aturan urutan kata yang bebas. Munculnya pertanyaan seberapa besar pengaruh kebebasan urutan kata terkait pengurangan jarak dependensi ini juga disebabkan oleh pernyataan dalam penelitian Futrell dkk (2015) yang menyebutkan bahwa temuan dalam bahasa Indonesia menunjukkan hasil pengurangan jarak dependensi yang tertinggi dibandingkan 36 bahasa lain yang diteliti.

%-----------------------------------------------------------------------------%
\section{Efisiensi Ujaran dari Segi Dependensi}
%-----------------------------------------------------------------------------%
Hingga abad ke-20, para linguis mengajukan hipotesis pengurangan jarak Euclidean (\textit{Euclidean distance minimization}) untuk jarak antara konstituen-konstituen ujaran yang bertautan secara sintaktis dan fenomena urutan konstituen lainnya (Ferrer-i-Cancho, 2004, 2008). Hal ini berkaitan dengan usaha penutur dalam menghasilkan ujaran yang efisien sesuai dengan penelitian yang dilakukan Gildea dan Jaeger (2015). Gildea dan Jaeger (2015) menguji hipotesis bahwa penutur cenderung menyusun konstituen dalam ujaran untuk menyampaikan informasi secara efisien dan menemukan bahwa semua bahasa yang menjadi obyek penelitiannya memiliki urutan konstituen yang lebih mudah diproses dan dimengerti. Dalam penelitian tersebut, Gildea dan Jaeger menarik korelasi antara kesaratan informasi dan jarak dependensi dengan menitikberatkan pada urutan konstituen dalam ujaran. Saat ini, istilah \textit{Euclidean distance minimization} digantikan dengan istilah pengurangan memori dalam jaringan atau \textit{online memory minimization} karena para peneliti bahasa melihat keterkaitan yang signifikan antara pengurangan jarak antarkonstituen dalam sebuah ujaran dengan kemudahan proses memori kerja (Ferrer-i-Cancho, 2015). Rata-rata (\textit{mean}) jarak dependensi menjadi ukuran penting dalam memprediksikan tingkat kesulitan atau kemudahan sintaktis ini (Hudson, 1995). Hipotesis bahwa penutur cenderung mengurangi atau meminimalkan jarak dependensi kemungkinan dikarenakan oleh keterbatasan memori kerja manusia yang beradaptasi terhadap faktor tata bahasa (Ferrer-i-Cancho, 2004, 2016; Buch-Kromann, 2006; Liu, 2008; Gildea and Temperley, 2010, Futrell dkk, 2015). 

%-----------------------------------------------------------------------------%
\subsection{Pengurangan Panjang dan Jarak Dependensi Terkait Memori Kerja}
%-----------------------------------------------------------------------------%
Beberapa studi berbasis empiris telah dilakukan untuk mengeksplorasi kesulitan pemahaman sebuah bahasa, terutama dalam ranah psikolinguistik dan linguistik kognitif (Jay, 2004). Pendekatan kuantitatif pada ilmu linguistik dalam penelitian seperti ini terus berkembang hingga saat ini dan penggunaan pendekatan-pendekatan tersebut memberikan kontribusi terhadap keakuratan metode linguistik kuantitatif yang dikembangkan. Yngve (1960) merupakan salah satu linguis yang mengajukan konsep jumlah 'simbol' maksimum yang dapat disimpan dalam memori kerja saat mengkonstruksi sebuah ujaran. Hipotesis Kedalaman atau \textit{Depth Hypothesis} yang dikembangkan Yngve (1960) diterjemahkan sebagai konsep untuk menganalisis kesulitan pemahaman sebuah ujaran. Isi hipotesis ini adalah bahwa (a) meskipun semua bahasa memiliki tata bahasa, (b) sebuah ujaran memiliki kedalaman yang tidak melebihi angka tertentu (c) sejumlah atau mendekati jumlah yang merepresentasikan memori kerja cepat dan (d) tata bahasa dari semua bahasa tersebut akan menyertakan metode-metode untuk membatasi konstruksi regresif sehingga ujaran tidak akan melewati batas kedalaman (Yngve, 1960, 1996). Dari hipotesis ini dapat diambil kesimpulan, yaitu meskipun tata bahasa memungkinkan adanya ujaran-ujaran yang lebih dalam secara teoretis, dalam prakteknya kedalaman tersebut tidak dapat melewati batas tertentu yang setara dengan kapasitas memori kerja manusia (Miller, 1956; Cowan, 2001, 2005). Dengan hipotesis ini, Yngve mencoba membangun indikator universal untuk kesulitan pemahaman sebuah ujaran. 

Miller dan Chomsky (1963) mengajukan indikator kompleksitas sintaktis berdasarkan rasio dari simpai akhir dan simpai bukan akhir pada diagram pohon sintaktis sebuah ujaran. Pendekatan ini dilanjutkan oleh Frazier (1985) yang menggunakan penghitungan dengan menyesuaikan konteks bahasa lokal untuk mengganti penghitungan global yang bersifat lintas bahasa milik Miller dan Chomsky agar indikator tersebut lebih sensitif dan kontekstual. Hawkins (1994) kemudian muncul dengan asumsi mengenai keterkaitan antara tata bahasa dan urutan kata. Untuk mengukur dan memperkirakan kesulitan sintaktis, Hawkins mengembangkan prinsip Konstituen Terdekat Awal atau \textit{Early Immediate Constituents} (EIC) yang menyatakan bahwa manusia cenderung memilih urutan linear yang memaksimalkan rasio Konstituen Terdekat (IC) terhadap non-IC dalam jangkauan pemahaman ujaran. Hawkins (2004) memperbaharui EIC menjadi Meminimalkan Jangkauan atau \textit{Minimize Domain} (MiD) dengan memanfaatkan teori dependensi. Prinsip ini menyatakan bahwa manusia cenderung meminimalkan deret unit linguistik yang memiliki relasi dan/atau meminimalkan proses dependensi yang terlibat (Hawkins, 2004). Prinsip ini secara langsung memperlihatkan keterkaitan antara urutan konstituen linear dengan pengolahan bahasa yang dilakukan dalam memori kerja manusia.

Hipotesis yang muncul dari penelitian-penelitian tersebut menggambarkan ketertarikan dunia linguistik atas indikator kompleksitas ujaran yang melibatkan keterkaitan antara urutan konstituen linear, kesulitan sintaktis serta bagaimana dependensi antarunit linguistik berperan di dalam konteks indikator kompleksitas tersebut. Penekanan dependensi dalam area penelitian ini adalah jika struktur kognitif manusia berbentuk seperti jejaring (Hudson, 2007), analisis terhadap jejaring dependensi sintaktis juga merupakan langkah penting menuju pemetaan jejaring kognitif konseptual yang secara langsung menggambarkan kerja kognisi manusia (Liu, 2008b). Liu (2008a) menyimpulkan bahwa dibandingkan dengan struktur frasa, keterkaitan struktur dependensi dengan struktur kognitif dan jejaring bahasa lebih dekat. Terdapat kesamaan dalam prinsip-prinsip dari berbagai penelitian tersebut, yaitu indikasi adanya konstituen-konstituen yang memiliki relasi semantis akan cenderung berdekatan dalam ujaran. Liu (2008b, 2017) mengajukan \textbf{Jarak Dependensi} atau \textit{Dependency Length} (yang selanjutnya akan disingkat menjadi DD) sebagai indikator penghitungan tersebut dan telah melakukan penelitian terhadap data lintas bahasa berskala besar. Di sisi lain, Gildea dan Temperley (2010) mengembangkan teori \textbf{Panjang Dependensi} atau \textit{Dependency Length} (yang selanjutnya akan disingkat menjadi DL), yang seperti Liu (2008b, 2017), berkembang dari studi-studi milik Gibson (1998, 2000) dan Hawkins (1994). Penggunaan istilah 'jarak' dan 'panjang' secara prinsip serupa karena keduanya merefleksikan sejauh apa sebuah konstituen terpisahkan dengan konstituen lain tempatnya bergantung atau yang dikendalikannya, tetapi memiliki sedikit perbedaan dalam penghitungan dan aspek dependensi yang ditekankannya. 

Liu, Xu, dan Liang (2017) mengembangkan proposal indikator yang menggunakan nilai DL menjadi hipotesis \textbf{Pengurangan Jarak Dependensi} atau \textit{Dependency Distance Minimization} (yang selanjutnya akan disingkat menjadi DDM). Sebagai contoh, istilah 'jarak' sering digunakan jika seseorang bergerak menuju sebuah tempat hingga orang tersebut berhenti. Dalam proses produksi dan pemahaman bahasa, 'jarak' dependensi baru akan dapat terukur apabila dependensi kedua konstituen telah terbentuk (konstituen kedua sudah diujarkan, dibaca, ditulis, atau didengar). Liu dkk (2017) berargumen bahwa 'panjang' merefleksikan ukuran dengan karakteristik statis dari sebuah entitas dan 'jarak' lebih menggambarkan proses dinamis sehingga dalam konteks dependensi sebuah ujaran, para peneliti tersebut beranggapan istilah 'jarak' lebih tepat digunakan. Untuk mendapatkan gambaran DDM, Liu (2008b, 2017) menjabarkan rumus untuk menghitung \textbf{Rata-rata Jarak Dependensi} atau \textit{Mean Dependency Distance} (yang selanjutnya akan disingkat menjadiMDD) sebuah ujaran dengan menggunakan formula:

\todo{insert formula}

di mana n adalah jumlah kata dalam ujaran dan DDi adalah jarak dependensi dari tautan sintaktis ke-i dalam ujaran. Hudson (2010) dan Ferrer-i-Cancho (2004) menggunakan rumus yang kurang lebih sama untuk menghitung MDD dalam sebuah ujaran. Studi matematis juga telah dilakukan untuk memperlihatkan bahwa jejaring sintaksis merupakan contoh 'dunia-dunia kecil' yang berarti angka rata-rata dari tautan dalam simpai ditemukan sangat kecil (Ferrer-i-Cancho, Sol�, & K�hler, 2004). Hipotesis utama dari penelitian DDM dijabarkan Liu (2008b, 2017) sebagai berikut:
\begin{itemize}
\item Manusia cenderung memilih urutan kata yang dapat mengurangi angka rata-rata jarak dependensi.
\item Terdapat batasan angka rata-rata jarak dependensi pada hampir semua ujaran dalam bahasa manusia.
\item Tata bahasa dan kognisi manusia bekerja sama untuk menekan jarak dependensi agar berada di dalam batasan tersebut.
\end{itemize}

Dalam penelitian lintas bahasa berskala besar lain, Futrell, Mahowald, dan Gibson (2015) menggunakan hipotesis \textbf{Pengurangan Panjang Dependensi} atau \textit{Dependency Length Minimization} (yang selanjutkan akan disingkat menjadi DLM). Penelitian lintas bahas ini memanfaatkan korpus data sebanyak 37 bahasa, termasuk bahasa Indonesia. Hipotesis ini sebelumnya muncul dalam penelitian Temperley dan Gildea (Temperley, 2007, 2008; Gildea & Temperley, 2010). Gildea dan Temperley (2010) menggunakan istilah 'panjang' dalam DLM karena menitikberatkan pada keseluruhan kompleksitas kalimat. Penghitungan yang digunakan untuk melihat indikasi DLM didapatkan dengan menjumlahkan keseluruhan nilai tautan-tautan dependensi dalam sebuah ujaran seperti yang terlihat pada Gambar 1.1 di Bab 1. Liu dkk (2017) dan Ferrer-i-Cancho (2004) beranggapan bahwa pendekatan ini cukup sensitif terhadap panjang kalimat atau ujaran sehingga perbedaan rata-rata yang muncul kurang bisa mewakili kesulitan sintaktis karena ada kecenderungan bahwa kalimat yang semakin panjang akan menghasilkan angka lebih besar. 

%-----------------------------------------------------------------------------%
\subsection{Perubahan Valensi Akar (\textit{Root}) Verbal}
%-----------------------------------------------------------------------------%
Penjabaran konsep dependensi juga mencakup seberapa kuat konstituen-konstituen dalam kalimat mengikat satu sama lain. Dalam teori dependensi (Tesni�re, 1959) dan \textit{Word Grammar} (Hudson, 2007), kemampuan sebuah induk (sebagai contoh: verba) menarik konstituen di sekitarnya (sebagai contoh: aktor pelaku yang umumnya adalah nomina) disebut dengan \textbf{valensi}. Sebagai ilustrasi, agar penerapan konstituen 'memandangi' dalam ujaran menjadi gramatikal, konstituen tersebut (yang memiliki fungsi sebagai proses) menuntut keterlibatan dua konstituen lain seperti pada ujaran: Budi_1 memandangi lukisan_2. Verba 'memandangi' mengalokasikan fungsi 'Budi' sebagai 'yang memandang' (aktor pelaku atau \textit{agent}) pada posisi pertama, dan 'sepatu' sebagai 'yang dipandang' (obyek penderita atau \textit{patient}) (Welke, 2005). Kebutuhan fungsional verba tersebut menentukan kategori dan posisi yang harus diisi untuk sebuah ujaran menjadi gramatikal. Tidak hanya pelengkap, verba juga dapat berkombinasi dengan keterangan (\textit{adjunct}) seperti pada ujaran: Budi_1 memandangi lukisan_2 di museum_3. Pelengkap dapat bersifat tidak wajib seperti pada contoh 'dia_1 membaca buku_2' dan 'dia_1 membaca'. 'Buku' merupakan obyek penderita langsung yang tidak wajib muncul agar pendengar memahami proses 'membaca'. 

Hingga saat ini, belum ada pendekatan linguistik kuantitatif ataupun perangkat komputasional dan inventori leksikon valensi untuk bahasa Indonesia. Oleh karena itu, analisis valensi dalam penelitian ini dilakukan secara manual atau bersifat kualitatif dengan menggunakan referensi dari anotasi yang ada. Analisis valensi dalam penelitian ini mengambil obyek berupa akar verbal pada simpai pusat karena diduga merupakan bentuk rangka valensi yang paling umum. Perlu ditekankan bahwa analisis valensi ini bukan untuk mengidentifikasi ujaran yang bersifat gramatikal dan tidak gramatikal, namun menitikberatkan pada fenomena linguistik yang muncul dalam \textit{real utterance} sesuai dengan korpus data yang dikumpulkan. Meskipun kemunculan akar verbal pada simpai pusat diduga memiliki frekuensi tertinggi, beberapa nomina dan adjektiva juga dapat muncul pada simpai pusat dan memiliki valensi tertentu (?ezn�?kov�, 2003;  Haji?, Panevov�, Ure�ov�, B�mov�, & Kol�?ov�, 2003).

%-----------------------------------------------------------------------------%
\section{Dependensi dan Urutan Kata dalam Bahasa Indonesia}
%-----------------------------------------------------------------------------%
\todo{Dalam tipologi berbagai penelitian bahasa, ternyata terdapat kecenderungan kuat dalam banyak bahasa untuk mengatur tata bahasanya agar menghasilkan ujaran dengan processing domain yang minimal (Hawkins, 2001). Sebagai contoh, Hiranuma (1999) menemukan bahwa aturan pilihan konstituen terikat (optionality of dependents) dalam bahasa Jepang memberikan kompensasi terhadap letak induk yang selalu berada di akhir dalam urutan kata (head-final). Tanpa kompensasi ini, maka rata-rata jarak dependensi akan menjadi jauh lebih besar dibandingkan dengan bahasa Inggris yang letak induknya bervariasi (mixed-order). Hasil dari pilihan ini adalah konstituen terikat per kata induk dalam bahasa Jepang lebih sedikit dibandingkan dengan bahasa Inggris, sehingga rata-rata jarak dependensi pada data yang diteliti Hiranuma antara bahasa Jepang dan Inggris hampir sama. Di sisi lain, terdapat juga bukti empiris bahwa beberapa bahasa mentolerir derajat kesulitan tersebut seperti perbandingan jarak dependensi antara bahasa Inggris dan Jerman yang dilakukan oleh Eppler (2004). Eppler menemukan bahwa jarak dependensi dalam bahasa Jerman lebih tinggi kemungkinan karena teekanan untuk meminimalkan jarak dependensi bertentangan dengan tekanan fungsional lainnya.}

%-----------------------------------------------------------------------------%
\subsection{Urutan Kata dalam Tata Bahasa Indonesia}
%-----------------------------------------------------------------------------%
\todo{Tantangan utama dalam bahasa Indonesia adalah bahwa batas antara kelas kata yang satu dengan yang lain tidak sejelas yang diharapkan (Kridalaksana, 2002; Achmad, 2012). Namun, Kridalaksana (2002) juga mengungkapkan bahwa dalam menentukan kelas kata dalam bahasa Indonesia, perilaku sintaktis dijadikan ciri dasar. Sesuai saran Robins (1985), paradigma morfologis juga mendukung penentuan kelas kata ini. Salah satu perilaku sintaktis yang disebut Kridalaksana adalah dependensi kata tersebut dengan kata lain dalam konstruksi. Verba tidak hanya ditempatkan sebagai dasar untuk menjelaskan proses dalam bahasa Indonesia (deMena Travis, 2006), tetapi juga menjadi dasar dalam mayoritas pembentukan kata (Kridalaksana, 1999).
Wasow (2002) beranggapan bahwa karena perbedaan makna antara tiap bentuk (alternant) dapat diabaikan, preferensi terhadap satu ujaran dibandingkan yang lain didorong oleh  pertimbangan pengolahan sebagai motivasi utama. Wasow (2002) mengajukan Prinsip Bobot Akhir atau Principle of End Weight (PEW) sebagai penjelasan tendensi tersebut. PEW menyatakan bahwa saat dihadapkan dengan pilihan, manusia cenderung menempatkan konstituen dengan bobot terberat seakhir mungkin dalam sebuah ujaran. Rosenbach (2005) meneliti peran bobot kata atau frasa dalam alternasi genitif seperti pada contoh the boy?s face dan the face of the boy. Dalam PEW, ?pemilik? memiliki bobot lebih tinggi oleh karena itu bentuk ujaran yang kedua seharusnya lebih banyak dipilih penutur karena menempatkan konstituen dengan bobot lebih besar lebih dekat ke akhir ujaran. Francis (2010) dan Rosenbach (2005) juga melihat efek bobot tersebut dalam ranah produksi (penutur) maupun pemahaman (pendengar). Francis (2010) melihat kecenderungan pemilihan bentuk ekstraposisi saat klausa relatif memiliki bobot lebih seperti pada contoh three people who were from Chicago arrived here yesterday dan three people arrived here yesterday who were from Chicago.
Dalam bahasa Indonesia, terdapat gambaran umum mengenai urutan kata sehingga sebuah kalimat menjadi gramatikal. Seperti yang dibahas dalam Indonesian Reference Grammar (Sneddon, 2010), beberapa urutan kata untuk frasa nomina adalah:
\begin{itemize}
\item Nomina yang memodifikasi langsung mengikuti induknya
\item Pemilik mengikuti adjektiva
\item Nomina atributif dalam frasa nomina perbuatan mengikuti adjektiva
\item Klausa relatif mengikuti nomina yang memodifikasi, adjektiva, dan pemilik
\item Demonstrativa mengikuti konstituen-konstituen lain dalam frasa
\end{itemize}
Sneddon (2010) juga mengungkapkan bahwa beberapa pertukaran posisi urutan kata tidak mengubah maknanya seperti sudah harus dan harus sudah. Urutan kata yang paling umum dalam klausa adalah subyek + predikat dan subyek + predikat + obyek dalam klausa verba transitif. Obyek primer akan menempati posisi lebih depan dibandingkan obyek sekunder. Secara umum, klausa pasif memiliki dua tipe, (i) bentuk subyek + verba + pelaku dan (ii) subyek + pelaku + verba (Sneddon, 2010). Meskipun keterangan cenderung menempati posisi tertentu di dalam klausa, pada kenyataannya jauh lebih bebas dibandingkan dengan komponen nukleus. Dilihat secara gramatikal, perubahan urutan kata dilakukan untuk memfokuskan perhatian terhadap komponen tertentu (Sneddon, 2010), namun dalam penelitian linguistic performance hal ini tidak selalu menggambarkan konsep gramatikal yang dituju.}

%-----------------------------------------------------------------------------%
\subsection{Simpai Sentral (\textit{Central Node}) dalam Bahasa Indonesia}
%-----------------------------------------------------------------------------%
\todo{Hingga saat ini, belum ada teori atau penelitian terkait teori dependensi dan simpai sentral dalam bahasa Indonesia, namun teori-teori terkait tata bahasa yang ada dapat menjadi referensi awal untuk melihat perilaku simpai sentral yang gramatikal dalam bahasa Indonesia. Berkaitan dengan kebebasan urutan kata dalam bahasa Indonesia (terutama posisi keterangan), simpai sentral pada sebuah kalimat menempati posisi yang tidak tentu seperti pada contoh kalau pulang, tolong belikan nasi bungkus dan tolong belikan nasi bungkus kalau pulang (Sneddon, 2010). Perbedaan posisi klausa pada kedua kalimat tersebut berkaitan dengan perbedaan posisi simpai sentral serta keterlibatan jeda (pause) dalam kalimat yang diwakilkan oleh tanda baca koma (,). Secara lisan maupun tulisan, jeda ini berpengaruh terhadap proses konstruksi kalimat sehingga patut diperhitungkan juga dalam mengkonstruksikan struktur dependensi. 
Bahasa Indonesia memiliki beragam klausa tanpa verba yang banyak ditemukan dalam pemakaian sehari-hari baik secara lisan maupun tulisan (Sneddon, 2010) seperti klausa nomina, klausa adjektiva, klausa preposisional, dan penggunaan kopula. Teori dependensi dapat digunakan untuk menganalisis karena simpai sentral tidak selalu melibatkan verba (Tesniere, 1959). Dalam kasus-kasus seperti tersebut, identifikasi induk dilakukan dengan melihat tautan terbanyak (Tesniere, 1959) dan mempertimbangkan jenis klausa itu sendiri (Sneddon, 2010).}