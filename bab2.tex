%-----------------------------------------------------------------------------%
\chapter{\babDua}

%-----------------------------------------------------------------------------%
\section{Konteks Penelitian}
%-----------------------------------------------------------------------------%
Kalimat merupakan sebuah susunan terorganisir yang memiliki konstituen berupa unit-unit kata dan tanda baca. Makna semantis sebuah konstituen di dalam ujaran atau kalimat tidak berdiri sendiri seperti kata-kata dan maknanya yang tertulis di dalam kamus. Melainkan, dalam penggunaan bahasa nyata (\textit{real utterance}), makna konstituen-konstituen tersebut baru akan terbentuk utuh setelah memiliki relasi dengan konstituen lain dalam sebuah frasa, klausa, kalimat ataupun antarkalimat. Pembentukan relasi dengan konstituen lain ini merupakan salah satu hasil penerapan fungsi konstituen atau kata \citep{tesniere1959elements}. Dalam ruang lingkup dependensi, fungsi yang membentuk relasi antarkonstituen dalam struktur ujaran tersebut terbagi dua: fungsi sintaksis (struktural) dan fungsi semantik (makna) \citep{tesniere1959elements}. Meskipun sintaksis dan semantik adalah dua bidang yang independen dan berbeda, keduanya masih berjalan sejajar dan saling berhubungan dalam teori dependensi \citep{tesniere1959elements}. Hal ini terjadi karena dependensi merupakan tautan struktural langsung yang menghubungkan unit-unit linguistik yang memiliki relasi semantis. Dalam perkembangan teorinya, konsep dependensi ini dapat dilacak jejaknya hingga pada akar tata bahasa Panini \citep{bharati1995natural}, tata bahasa Yunani dan Latin kuno (\citealp{covington1984syntactic, percival1990reflections}), serta bahasa Arab \citep{owens1988foundations}. 

Konsep dependensi yang ditemukan pada beberapa bahasa kuno di atas menunjukkan bahwa kata merupakan konstitituen atau unit utama sintaksis dan konstituen-konstituen dalam ujaran tersebut memiliki relasi struktural secara langsung. Terkait dengan peran kata, kesejajaran, dan hubungan antara sintaksis (struktur) dan semantik (makna), (\citealp{hudson1984word, hudson2007language}) mengembangkan teori \textit{Word Grammar} atau Tata Bahasa Kata. \textit{Word Grammar} mengadopsi konsep dependensi sebagai dasar untuk menelaah struktur kalimat dan melihat bahasa sebagai sebuah jejaring \citep{hudson2007language}. Kata sebagai unit utama sintaksis dan penerapannya dalam teori dependensi juga banyak menjadi bahan diskusi dalam pengembangan metode komputasional. Salah satu teori utama terkait linguistik komputasional yang bertitik tolak dari teori dependensi tersebut adalah \textit{Meaning-Text Theory} atau Teori Makna-Teks (MTT) yang dikembangkan oleh Igor Mel??uk (1988) dan penguraian kalimat multilingual berdasarkan teori dependensi yang dibangun dari pengumpulan data diagram pohon dependensi dari berbagai bahasa, proyek ini disebut dengan \textit{Universal Dependencies} (\citealp{mcdonald2013universal, nivre2016universal, nivre2017universal}). 


%-----------------------------------------------------------------------------%
\subsection{Kemampuan (\textit{Competence}) dan Penampilan (\textit{Performance})}
%-----------------------------------------------------------------------------%
Agar dapat menggunakan \latex~(pada konteks hanya sebagai pengguna), Anda 
tidak perlu banyak tahu mengenai hal-hal didalamnya. 
Seperti halnya pembuatan dokumen secara visual (contohnya Open Office (OO) 
Writer), Anda dapat menggunakan \latex~dengan cara yang sama. 
Orang-orang yang menggunakan \latex~relatif lebih teliti dan terstruktur 
mengenai cara penulisan yang dia gunakan, \latex~memaksa Anda untuk seperti 
itu.  

Kembali pada bahasan utama, untuk mencoba \latex~Anda cukup mendownload 
kompiler dan IDE. Saya menyarankan menggunakan Texlive dan Texmaker. 
Texlive dapat didownload dari \url{http://www.tug.org/texlive/}. 
Sedangkan Texmaker dapat didownload dari 
\url{http://www.xm1math.net/texmaker/}. 
Untuk pertama kali, coba buka berkas thesis.tex dalam template yang Anda miliki 
pada Texmaker. 
Dokumen ini adalah dokumen utama. 
Tekan F6 (PDFLaTeX) dan Texmaker akan mengkompilasi berkas tersebut menjadi 
berkas PDF. 
Jika tidak bisa, pastikan Anda sudah menginstall Texlive. 
Buka berkas tersebut dengan menekan F7. 
Hasilnya adalah sebuah dokumen yang sama seperti dokumen yang Anda baca saat 
ini. 


%-----------------------------------------------------------------------------%
\subsection{Efisiensi dan Urutan Kata (\textit{Word Order}) dalam Ranah Ilmu Sintaksis}
%-----------------------------------------------------------------------------%
Hal pertama yang mungkin ditanyakan adalah bagaimana membuat huruf tercetak 
tebal, miring, atau memiliki garis bawah. 
Pada Texmaker, Anda bisa melakukan hal ini seperti halnya saat mengubah dokumen 
dengan OO Writer. 
Namun jika tetap masih tertarik dengan cara lain, ini dia: 

\begin{itemize}
	\item \bo{Bold} \\
		Gunakan perintah \bslash textbf$\lbrace\rbrace$ atau 
		\bslash bo$\lbrace\rbrace$. 
	\item \f{Italic} \\
		Gunakan perintah \bslash textit$\lbrace\rbrace$ atau 
		\bslash f$\lbrace\rbrace$. 
	\item \underline{Underline} \\
		Gunakan perintah \bslash underline$\lbrace\rbrace$.
	\item $\overline{Overline}$ \\
		Gunakan perintah \bslash overline. 
	\item $^{superscript}$ \\
		Gunakan perintah \bslash $\lbrace\rbrace$. 
	\item $_{subscript}$ \\
		Gunakan perintah \bslash \_$\lbrace\rbrace$. 
\end{itemize}

Perintah \bslash f dan \bslash bo hanya dapat digunakan jika package 
uithesis digunakan. 


%-----------------------------------------------------------------------------%
\subsection{Relasi dan Struktur Ujaran}
%-----------------------------------------------------------------------------%
Setiap gambar dapat diberikan caption dan diberikan label. Label dapat 
digunakan untuk menunjuk gambar tertentu. 
Jika posisi gambar berubah, maka nomor gambar juga akan diubah secara 
otomatis. 
Begitu juga dengan seluruh referensi yang menunjuk pada gambar tersebut. 
Contoh sederhana adalah \pic~\ref{fig:testGambar}. 
Silahkan lihat code \latex~dengan nama bab2.tex untuk melihat kode lengkapnya. 
Harap diingat bahwa caption untuk gambar selalu terletak dibawah gambar. 

\begin{figure}
	\centering
	\includegraphics[width=0.50\textwidth]
		{pics/creative_common.png}
	\caption{\license.}
	\label{fig:testGambar}
\end{figure}


%-----------------------------------------------------------------------------%
\subsection{Penelitian Terdahulu dan \textit{Positioning}}
%-----------------------------------------------------------------------------%
Seperti pada gambar, tabel juga dapat diberi label dan caption. 
Caption pada tabel terletak pada bagian atas tabel. 
Contoh tabel sederhana dapat dilihat pada \tab~\ref{tab:tab1}.

\begin{table}
	\centering
	\caption{Contoh Tabel}
	\label{tab:tab1}
	\begin{tabular}{| l | c r |}
		\hline
		& kol 1 & kol 2 \\ 
		\hline
		baris 1 & 1 & 2 \\
		baris 2 & 3 & 4 \\
		baris 3 & 5 & 6 \\
		jumlah  & 9 & 12 \\
		\hline
	\end{tabular}
\end{table}

Ada jenis tabel lain yang dapat dibuat dengan \latex~berikut 
beberapa diantaranya. 
Contoh-contoh ini bersumber dari 
\url{http://en.wikibooks.org/wiki/LaTeX/Tables}

\begin{table}
	\centering
	\caption{An Example of Rows Spanning Multiple Columns}
	\label{row.spanning}
	\begin{tabular}{|l|l|*{6}{c|}}
  		\hline % create horizontal line
  		No & Name & \multicolumn{3}{|c|}{Week 1} & \multicolumn{3}{|c|}{Week 2} \\
  		\cline{3-8} % create line from 3rd column till 8th column
  		& & A & B & C & A & B & C\\
  		\hline
  		1 & Lala & 1 & 2 & 3 & 4 & 5 & 6\\
  		2 & Lili & 1 & 2 & 3 & 4 & 5 & 6\\
  		3 & Lulu & 1 & 2 & 3 & 4 & 5 & 6\\
  		\hline
	\end{tabular}
\end{table}

\begin{table}
	\centering
	\caption{An Example of Columns Spanning Multiple Rows}
	\label{column.spanning}
	\begin{tabular}{|l|c|l|}
		\hline
		Percobaan & Iterasi & Waktu \\
		\hline
		Pertama & 1 & 0.1 sec \\ \hline
		\multirow{2}{*}{Kedua} & 1 & 0.1 sec \\
 		& 3 & 0.15 sec \\ 
 		\hline
		\multirow{3}{*}{Ketiga} & 1 & 0.09 sec \\
 		& 2 & 0.16 sec \\
 		& 3 & 0.21 sec \\ 
 		\hline
	\end{tabular}
\end{table}

\begin{table}
	\centering
	\caption{An Example of Spanning in Both Directions Simultaneously}
	\label{mix.spanning}
	\begin{tabular}{cc|c|c|c|c|}
		\cline{3-6}
		& & \multicolumn{4}{|c|}{Title} \\ \cline{3-6}
		& & A & B & C & D \\ \hline
		\multicolumn{1}{|c|}{\multirow{2}{*}{Type}} &
		\multicolumn{1}{|c|}{X} & 1 & 2 & 3 & 4\\ \cline{2-6}
		\multicolumn{1}{|c|}{}                        &
		\multicolumn{1}{|c|}{Y} & 0.5 & 1.0 & 1.5 & 2.0\\ \cline{1-6}
		\multicolumn{1}{|c|}{\multirow{2}{*}{Resource}} &
		\multicolumn{1}{|c|}{I} & 10 & 20 & 30 & 40\\ \cline{2-6}
		\multicolumn{1}{|c|}{}                        &
		\multicolumn{1}{|c|}{J} & 5 & 10 & 15 & 20\\ \cline{1-6}
	\end{tabular}
\end{table}


%-----------------------------------------------------------------------------%
\section{Konsep Dependensi}
%-----------------------------------------------------------------------------%
Setiap gambar dapat diberikan caption dan diberikan label. Label dapat 
digunakan untuk menunjuk gambar tertentu. 
Jika posisi gambar berubah, maka nomor gambar juga akan diubah secara 
otomatis. 


%-----------------------------------------------------------------------------%
\subsection{Konstituen Induk (\textit{Root/Head}) dan Simpai (\textit{Node})}
%-----------------------------------------------------------------------------%
Setiap gambar dapat diberikan caption dan diberikan label. Label dapat 
digunakan untuk menunjuk gambar tertentu. 
Jika posisi gambar berubah, maka nomor gambar juga akan diubah secara 
otomatis. 


%-----------------------------------------------------------------------------%
\subsection{Panjang Dependensi (\textit{Dependency Length}) dan Jarak Dependensi (\textit{Dependency Distance})}
%-----------------------------------------------------------------------------%
Setiap gambar dapat diberikan caption dan diberikan label. Label dapat 
digunakan untuk menunjuk gambar tertentu. 
Jika posisi gambar berubah, maka nomor gambar juga akan diubah secara 
otomatis. 


%-----------------------------------------------------------------------------%
\subsection{Faktor-faktor yang Mempengaruhi Tautan Dependensi}
%-----------------------------------------------------------------------------%
Setiap gambar dapat diberikan caption dan diberikan label. Label dapat 
digunakan untuk menunjuk gambar tertentu. 
Jika posisi gambar berubah, maka nomor gambar juga akan diubah secara 
otomatis. 

%-----------------------------------------------------------------------------%
\subsubsection{Kognisi Manusia dan Produksi Ujaran}
%-----------------------------------------------------------------------------%
Setiap gambar dapat diberikan caption dan diberikan label. Label dapat 
digunakan untuk menunjuk gambar tertentu. 
Jika posisi gambar berubah, maka nomor gambar juga akan diubah secara 
otomatis. 

%-----------------------------------------------------------------------------%
\subsubsection{Karakter Bahasa dan Ketatabahasaan}
%-----------------------------------------------------------------------------%
Setiap gambar dapat diberikan caption dan diberikan label. Label dapat 
digunakan untuk menunjuk gambar tertentu. 
Jika posisi gambar berubah, maka nomor gambar juga akan diubah secara 
otomatis. 

%-----------------------------------------------------------------------------%
\section{Efisiensi Ujaran dari Segi Dependensi}
%-----------------------------------------------------------------------------%
Setiap gambar dapat diberikan caption dan diberikan label. Label dapat 
digunakan untuk menunjuk gambar tertentu. 
Jika posisi gambar berubah, maka nomor gambar juga akan diubah secara 
otomatis. 

%-----------------------------------------------------------------------------%
\subsection{Pengurangan Panjang dan Jarak Dependensi Terkait Memori Kerja}
%-----------------------------------------------------------------------------%
Setiap gambar dapat diberikan caption dan diberikan label. Label dapat 
digunakan untuk menunjuk gambar tertentu. 
Jika posisi gambar berubah, maka nomor gambar juga akan diubah secara 
otomatis. 

%-----------------------------------------------------------------------------%
\subsection{Perubahan Valensi Konstituen Induk (\textit{Root}) Verbal}
%-----------------------------------------------------------------------------%
Setiap gambar dapat diberikan caption dan diberikan label. Label dapat 
digunakan untuk menunjuk gambar tertentu. 
Jika posisi gambar berubah, maka nomor gambar juga akan diubah secara 
otomatis. 

%-----------------------------------------------------------------------------%
\section{Dependensi dan Urutan Kata dalam Bahasa Indonesia}
%-----------------------------------------------------------------------------%
Setiap gambar dapat diberikan caption dan diberikan label. Label dapat 
digunakan untuk menunjuk gambar tertentu. 
Jika posisi gambar berubah, maka nomor gambar juga akan diubah secara 
otomatis. 

%-----------------------------------------------------------------------------%
\subsection{Urutan Kata dalam Tata Bahasa Indonesia}
%-----------------------------------------------------------------------------%
Setiap gambar dapat diberikan caption dan diberikan label. Label dapat 
digunakan untuk menunjuk gambar tertentu. 
Jika posisi gambar berubah, maka nomor gambar juga akan diubah secara 
otomatis. 

%-----------------------------------------------------------------------------%
\subsection{Simpai Sentral (\textit{Central Node}) dalam Bahasa Indonesia}
%-----------------------------------------------------------------------------%
Setiap gambar dapat diberikan caption dan diberikan label. Label dapat 
digunakan untuk menunjuk gambar tertentu. 
Jika posisi gambar berubah, maka nomor gambar juga akan diubah secara 
otomatis. 
